%\ProvidesPackage{UPIITA}
\usepackage[spanish,es-tabla]{babel}
\usepackage[utf8]{inputenc} 
\usepackage{amsmath}
\usepackage{amssymb}
\usepackage{graphicx}
%\PassOptionsToPackage{table,svgnames,xcdraw}{xcolor}
\usepackage{color}
%\usepackage{subfigure}
\usepackage{capt-of}
\usepackage{sidecap}
\sidecaptionvpos{figure}{c}
\usepackage{caption}
\usepackage{commath}
\usepackage{cancel}
\usepackage{anysize}
\usepackage{csquotes}
\usepackage{tcolorbox}
\usepackage{lipsum}
\usepackage{answers}
\usepackage{nccmath}
\usepackage[shortlabels]{enumitem}
\usepackage[acronym,toc]{glossaries}
\captionsetup[table]{position=top}
\usepackage{filecontents}
\tcbuselibrary{skins,breakable}
\usetikzlibrary{shadings,shadows}
\usepackage{mathtools, amsfonts, amssymb, amsthm}
\usepackage{makeidx}
\usepackage{float,graphicx}
\usepackage{lmodern}
\usepackage{gensymb}
\usepackage{nccmath}
\usepackage{tikz}
%\usepackage{newclude}

%=============Packages agregados por LERS==============
\usepackage {ulem} 		% Paquete ulem para subrayado y tachado de texto
\usepackage{rotating} 
\usepackage{booktabs}
\usepackage{colortbl}
\usepackage{xurl}         		% Permite añadir enlaces
\usepackage{cite}
\usepackage{longtable}
\usepackage{graphicx} 
\usepackage{tabularx}
\usepackage{multicol}   		% Múltiples columnas
\usepackage{multirow}      % Agrega nuevas opciones a las tablas
\usepackage{tabto}
\usepackage{pdfpages}
\usepackage{subcaption}
% Segmento de código que permite redefinir \paragraph{title} para agregarlo como subtítulo
{\makeatletter
\renewcommand\paragraph{\@startsection{paragraph}{4}{\z@}%
	{-2.5ex\@plus -1ex \@minus -.25ex}%
	{1.25ex \@plus .25ex}%
	{\normalfont\normalsize\bfseries}}
\makeatother
\setcounter{secnumdepth}{4} % how many sectioning levels to assign numbers to
\setcounter{tocdepth}{4}}    % how many sectioning levels to show in ToC

\usepackage{chngcntr}
\counterwithin{figure}{section}
\counterwithin{table}{section}

\usepackage{tocloft}
\setlength{\cftfignumwidth}{3.5em} % o 4em si aún se desbordan


%==================== Márgenes ==================
\marginsize{2cm}{2cm}{2cm}{2cm} % Izquierda, derecha, arriba, abajo
%=================== Insertar PDFs ==============
\usepackage{pdfpages} 
%================== Apéndices ==================
\usepackage[toc,page]{appendix}
\usepackage{chngcntr} % Para controlar numeración de figuras/tablas
% Cambiar "Appendices" por "Anexos"
\renewcommand{\appendixname}{Anexos}
\renewcommand{\appendixtocname}{Anexos}
\renewcommand{\appendixpagename}{Anexos}
%%%%%%%%%%%%%%%%%% Hipervínculos %%%%%%%%%%%%%%%%%%%%%%%%%%%%%%%%%%%%%%%%%%
\usepackage[colorlinks=true,plainpages=true,citecolor=black,linkcolor=black]{hyperref}
%\usepackage{hyperref} 

% Encabezado y pie de página
\usepackage{fancyhdr} 
\pagestyle{fancy}
\fancyhf{}
\fancyhead[L]{\footnotesize UPIITA}
\fancyhead[R]{\footnotesize IPN}
\fancyfoot[R]{\footnotesize Trabajo Terminal II}
\fancyfoot[C]{\thepage}
%\fancyfoot[C]{\ifnum\value{page}>2 \thepage\fi}
\fancyfoot[L]{\footnotesize Ing. Mecatrónica}
\renewcommand{\footrulewidth}{0.4pt}

%============= Para introducir código MATLAB ========
% configuración para el lenguaje que queramos utilizar
\usepackage{listings}
\definecolor{dkgreen}{rgb}{0,0.6,0}
\definecolor{gray}{rgb}{0.5,0.5,0.5} 
\lstset{language=Matlab,
	keywords={break,case,catch,continue,else,elseif,end,for,function,
		global,if,otherwise,persistent,return,switch,try,while},
	basicstyle=\ttfamily,
	keywordstyle=\color{blue},
	commentstyle=\color{red},
	stringstyle=\color{dkgreen},
	numbers=left,
	numberstyle=\tiny\color{gray},
	stepnumber=1,
	numbersep=10pt,
	backgroundcolor=\color{white},
	tabsize=4,
	showspaces=false,
	showstringspaces=false}
%======================= Del Inglés al Español =======================
\newcommand{\sen}{\operatorname{\sen}}	% Definimos el comando \sen para el seno
%en español
\newcommand{\asignatura}[1]{
	\def\@asignatura{#1}
}
\newcommand{\carrera}[1]{
	\def\@carrera{#1}
}
\newcommand{\titulo}[1]{
	\def\@titulo{#1}
}
\newcommand{\grupo}[1]{
	\def\@grupo{#1}
}
\newcommand{\asesores}[1]{
	\def\@asesores{#1}
}
\newcommand{\alumno}[1]{
	\def\@alumno{#1}
}