\chapter{Resumen}
\textbf{Resumen:}
Este proyecto propone el desarrollo de una ortesis robótica diseñada para asistir en la rehabilitación de pacientes con hemiplejia en la extremidad inferior derecha. La hemiplejia, causada por accidentes cerebrovasculares, afecta gravemente la movilidad, lo que requiere un proceso intensivo de fisioterapia para prevenir complicaciones como la atrofia muscular y contracturas articulares. La ortesis robótica tiene como objetivo replicar los movimientos de flexión, extensión, abducción y aducción en las articulaciones de cadera (coxofemoral) y rodilla, ajustando los parámetros en función de las necesidades de cada paciente. Este dispositivo automatizado aliviará la carga física del fisioterapeuta través de una rehabilitación optimizada y personalizada. El sistema emplea motores para controlar los movimientos, junto con un sistema de retroalimentación y seguimiento de trayectoria monitoreados a través de un interfaz humano máquina. \\ 

\textbf{Palabras clave:} Ortesis robótica, rehabilitación, hemiplejia, coxofemoral, rodilla, seguimiento de trayectoria.

\endinput 