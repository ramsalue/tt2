En esta sección se describe la construcción física de cada uno de los sistemas que conforman el proyecto. Así mismo, se detallan los cambios realizados al diseño planteado en Trabajo Terminal I con el objetivo de obtener un funcionamiento adecuado al integrar los distintos sistemas del proyecto. Además, se realizan pruebas para verificar el funcionamiento de cada uno y de forma integral. 
\section{Hoja de especificaciones del usuario}
\label{sec:hoja_usuario}
El sistema de ortesis robótica está diseñado para un rango de usuarios que cumpla con las siguientes características antropométricas, obtenidas de estudios de población mexicana \cite{TT2_01}, \cite{TT2_02}:

\begin{table}[h!]
	\centering
	\caption{Hoja de especificaciones antropométricas (resumen).}
	\begin{tabular}{|l|c|c|c|c|c|}
		\hline
		\textbf{Parámetro} &\textbf{Unidad} & \textbf{ID} & \textbf{P5} & \textbf{P95} & \textbf{Mín / Máx (diseño)} \\
		\hline
		Longitud total de pierna	 & cm &1&81 & 94 & 81-94 $\pm$ 10mm \\ \hline
		Longitud a altura rodilla	 & cm &2& 43.4 & 52.6 & 43.4-52.6 $\pm$ 10mm \\ \hline
		Perímetro pantorrilla 		 & cm &3& 31.5 & 42 & 31.5-42 $\pm$ 10mm \\ \hline
		Altura sentado 				 & cm &4& 82.5 & 95 & 82.5-95 $\pm$ 10mm \\ \hline
		Longitud nalga-poplíteo 	 & cm &5& 43.2 & 52.6 & 43.2-52.6 $\pm$ 10mm \\ \hline
		Longitud nalga-rodilla 		 & cm &6& 53.7 & 64 & 53.7-64 $\pm$ 10mm \\ \hline
		Altura poplítea				 & cm &7& 37.4 & 45.3&37.4-45.3 $\pm$ 10mm\\ \hline
		Altura rodilla sentado		 & cm &8&47.3&55.6&47.3-55.6 $\pm$ 10mm\\ \hline
		Anchura codos				 & cm &9& 44.3 & 62 & 44.3-62 $\pm$ 10mm \\ \hline
		Anchura cadera (sentado) 	 & cm &10& 32.8 & 42.3 & 32.8-42.3 $\pm$ 10mm \\ \hline
		Peso corporal (P95) 		 & kg && 55 & 97.3 & \textbf{Carga diseño = 92} \\
		\hline
	\end{tabular}
\end{table}
\begin{figure}[h!]
	\centering
	\includegraphics[width=0.2\textwidth]{figure/P06.png}
	\caption[Dimensión antropométrica de longitud de pierna.]{Dimensión antropométrica de longitud de pierna.\label{fig:P06}}
\end{figure}
\begin{figure}[h!]
	\centering
	\subcaptionbox{Longitud a altura de rodilla y perímetro de pantorrilla.\label{fig:P07}}
	{\includegraphics [trim = 0 0cm 0 0, clip,width=8cm]{figure/P07.png}}\\
	\subcaptionbox{Longitudes y alturas en posición sentado.\label{fig:P08}}
	{\includegraphics [trim = 0 0cm 0 0, clip,width=8cm]{figure/P08.png}}
	\caption[Dimensiones antropométricas contempladas para diseño del sistema.]{Dimensiones antropométricas contempladas para diseño del sistema.\label{fig:P07P08}}
\end{figure}
Estas dimensiones determinan:
\begin{itemize}
	\item La distancia y ajuste de los puntos de sujeción mecánica.
	\item La fuerza nominal requerida de los actuadores lineales y rotativos.
	\item La configuración de las rutinas de movimiento, para mantener seguridad y comodidad.
\end{itemize}

El diseño mecánico y los mecanismos de sujeción (véase Anexo \ref{sec:planos de manufactura}, \nameref{sec:planos de manufactura}) se dimensionaron considerando el percentil 95 para garantizar compatibilidad con la mayoría de los usuarios.
\clearpage
\subsection{Adquisición y verificación de componentes}\label{subsec:adquisicion}
Parte importante para la construcción del proyecto recayó en la adquisición de los componentes a utilizar, tomando en cuenta su capacidad para desempeñar su función esperada, así como los tiempos de envío y entrega una vez realizados los pedidos.
%% DESCRIBIR LO QUE SE COMPRÓ CON BASE EN LA LISTA DE MATERIALES EN LA TABLA DE COSTOS
\subsubsection{Gestión de proveedores y materiales}\label{subsubsec:gestion_proveedores}
La fase de implementación dio inicio con la adquisición de los componentes y materiales estructurales definidos en el diseño detallado desarrollado en Trabajo Terminal I. La selección de los componentes se basó en el cumplimiento de las especificaciones técnicas, mientras que para la selección de proveedores  se tomó en cuenta la disponibilidad en el mercado nacional, costos y tiempos de entrega.\\
Se gestionó la compra de todos los elementos, desde los perfiles estructurales PTR y de aluminio, hasta los componentes necesarios como los drivers \texttt{HSS57} y \texttt{HSS86}, los motores a pasos Nema 23 y Nema 34, y la unidad de procesamiento (Raspberry Pi 4).
\subsection{Inspección de componentes críticos}\label{subsec:inspeccion_componentes}
Una vez recibidos los materiales, se realizó una inspección visual y técnica para verificar que correspondieran a las especificaciones requeridas para el proyecto.

Se verificó que la unidad de procesamiento [Fig. \ref{fig:comp_rpi}] corresponde al modelo Raspberry Pi 4 Model B con 4GB de RAM. Para la HMI, se recibió la pantalla táctil Waveshare de 7 pulgadas [Fig. \ref{fig:comp_waveshare}], confirmando su compatibilidad de conexión HDMI y USB. Los sensores y los interruptores de límite fueron inspeccionados visualmente. 
\begin{figure}[h!]
	\centering
	\subcaptionbox{Unidad de procesamiento Raspberry Pi 4.\label{fig:comp_rpi}}
	{\includegraphics[width=0.48\textwidth]{figure/img_componentes/Untitled.png}}
	\subcaptionbox{Pantalla HMI Waveshare de 7 pulgadas.\label{fig:comp_waveshare}}
	{\includegraphics[width=0.48\textwidth]{figure/img_componentes/Untitled.png}}
	\caption{Componentes principales del sistema de control y comunicación.}
\end{figure}
\clearpage
\section{Sistema estructural (S1)}\label{Implementacion S1}
La construcción del sistema estructural fue dividido en secciones comenzando por la base de la cama que soportaría a los mecanismos de flexión-extensión, abducción-aducción, así como el montaje de componentes electrónicos y actuadores requeridos para los sistemas de control, seguridad eléctrica y comunicación humano-máquina.
\subsection{Construcción de la cama de soporte}
El soporte principal se construyó utilizando perfiles PTR soldados con la finalidad de obtener una base rígida ante la carga vertical del paciente y los mecanismos que soportará la estructura. En la Fig. \ref{fig:perfil_ptr} se muestra el perfil del PTR con dimensiones 2x2 pulgadas y grosor de calibre 12. 
\begin{figure}[h!]
	\centering
	\includegraphics[width=0.4\textwidth]{figure/img_estructura/PTR_perfil_.png}
	\caption{Perfil PTR de dimensión 2x2 pulgadas y calibre 12, para construcción de la cama soporte.\label{fig:perfil_ptr}}
\end{figure} 

Se integró una colchoneta de espuma de alta densidad que se utiliza como superficie de acojinamiento para comodidad del paciente con un espesor de XX mm. Dicha elevación permite alinear el eje de rotación de la articulación coxofemoral del paciente con el eje mecánico del actuador de abducción. Sin este ajuste de altura la discrepancia de alturas generaría un par de fuerzas no deseado en la cadera del paciente durante la sesión. 
\begin{figure}[h!]
	\centering
	\includegraphics[width=0.48\textwidth]{figure/img_componentes/Untitled.png}
	\caption{Implementación física con colchoneta de ajuste.\label{fig:cama_estructura}}
\end{figure}
\subsection{Construcción del mecanismo de flexión-extensión}
El mecanismo se manufacturó utilizando perfiles de aluminio 6061-T6 para reducir la dificultad del movimiento relacionada con la inercia del mecanismo. Las uniones se realizaron con tornillería de acero galvanizado.
\subsubsection*{Diseño de la placa base, manejo de tolerancias y deformaciones térmicas}
Durante el proceso de manufactura, las placas de acero fueron cortadas mediante láser. Se observó que el calor térmico generado al momento del corte provocó una curvatura, lo que dificultó el alineamiento de las placas con los rieles de la guía. Para corregir el ensamble se realizó una alineación activa, es decir, se montaron el eje, las chumaceras y las placas simultáneamente, y se ajustó la posición con esfuerzo manual antes de aplicar los puntos de soldadura y fijación final. Esto permitió que el eje del movimiento permaneciera paralelo a las guías lineales evitando atascamientos.

El mecanismo de flexión-extensión se colocó sobre una placa de acero al carbón de 1/2". La longitud de la placa fue acotada al área de transmisión de esfuerzos del actuador lineal en lugar de cubrir toda la base con la finalidad de reducir la deflexión del eje debido al peso de la pierna. Además, se implementaron múltiples puntos de apoyo (chumaceras) que fueron distribuidos a lo largo del eje de transmisión, reduciendo así el momento flector en los extremos.
\begin{figure}[h!]
	\centering
	\includegraphics[width=0.48\textwidth]{figure/img_componentes/Untitled.png}
	\caption{Implementación del mecanismo de flexión-extensión realizando el ensamble físico con placas de acero y perfil de aluminio.\label{fig:mecanismo_flexion_extension}}
\end{figure}

\subsubsection*{Discrepancias entre diseño e implementación del mecanismo de flexión-extensión}
\paragraph{Soporte distal (pie).}
Durante el ensamble del soporte distal (pie)\footnote{Se proporciona mayor información de la manufactura de los soportes en la subsección \nameref{Implementacion S2 y S3}.}, se encontró una interferencia dimensional en la unión de la impresión 3D plegada que actúa como soporte del pie con la estructura de aluminio, por lo que se aplicó un proceso de ajuste mecánico (desbaste de 3 mm) en las caras laterales para lograr un acople deslizante.
\paragraph*{Barreno invertido en clévis (fémur).}
En uno de los tubos telescópicos del fémur, un barreno para unión clévis fue soldado en sentido invertido al planteado en el diseño CAD, sin embargo, no afecta al funcionamiento ni el movimiento por lo que se decidió conservar dicho cambio debido a que el funcionamiento es aceptable.
\subsection{Construcción del mecanismo de abducción-aducción.}
El mecanismo de abducción-aducción se encarga de soportar la carga del subsistema de flexión y la pierna del paciente, y produce la rotación de un eje vertical. Para este mecanismo se utilizó el motor Nema 34 acoplado un eje de acero maquinado. Sobre el eje se tienen los puntos de apoyo distribuidos que reducen el momento flector en los extremos.
\subsubsection*{Selección de elementos de rotación}
Durante la validación del diseño, se analizó la selección de los elementos rodantes para las articulaciones superior e inferior.
\begin{enumerate}
	\item \textbf{Eje inferior (carga principal):} Se implementaron chumaceras de bolas y un rodamiento axial en la base. Esto para soportar la carga axial combinada (aprox. 25 kg) con mínima fricción.
	\item \textbf{Uniones superiores (nivel coxofemoral):} Para las articulaciones superiores del mecanismo de flexión-extensión, se emplearon las juntas Clevis seleccionadas mediante tablas de decisión, la cuales no tienen rodamientos de bolas, esto se debe a que la velocidad angular de operación es muy baja ($< 0.5$ rad/s) y el ciclo de trabajo no es continuo, lo que implica que la fricción generada por el contacto metal-metal lubricado es despreciable en comparación con el par disponible de los actuadores. Además, esta configuración reduce el volumen del mecanismo en la zona de la cadera. Esto reduce el volumen ocupado por el mecanismo en la articulación coxofemoral. 
\end{enumerate}

\begin{figure}[h!]
	\centering
	\includegraphics[width=0.6\textwidth]{figure/img_componentes/Untitled.png}
	\caption{Eje principal de abducción maquinado con cuñero para acople directo al motor Nema 34.\label{fig:mecanismo_abduccion_aduccion}}
\end{figure}
\subsection{Integración de mecanismos y montaje final}
Durante las fases finales del ensamble se realizó el acoplamiento del subensamble de flexión-extensión sobre el eje de abducción, y se realizaron los barrenos para la fijación de los interruptores de límite directamente sobre la estructura, ajustando su posición para permitir el movimiento de los mecanismos de flexión-extensión y abducción-aducción. Durante este proceso se encontró fricción excesiva con partes de la estructura por lo que se recurrió al desbaste de zonas de la estructura con el fin de permitir un desplazamiento suave y evitar atascamientos. 
% Debo mencionar más sobre esta subsubsección que aún está incompleta. Agregar la referencia a la imagen donde describa como se están montando los interruptores de límite
\begin{figure}[h!]
	\centering
	\includegraphics[width=0.7\textwidth]{figure/img_componentes/Untitled.png}
	\caption{Ajuste dimensional a estructura para montaje de interruptores de límite.\label{fig:limit_switch_montaje}}
\end{figure}

\section{Verificación del sistema estructural}

Una vez ensamblado el sistema S1, se procedió a su validación estática antes de integrar la electrónica sensible.

\subsection{Prueba S1-01: Integridad estructural bajo carga}
\textbf{Objetivo:} Verificar que la estructura soporta la carga de diseño sin deformaciones plásticas ni fallos en las soldaduras.
\\
\textbf{Procedimiento:} Se aplicó una carga estática distribuida de 90 kg sobre la cama de soporte y una carga puntual de 20 kg sobre el mecanismo de pierna extendida, simulando el peso de un paciente en el percentil P95.
\\
\textbf{Resultados:} La estructura no presentó deformaciones visibles ni ruidos estructurales. El mecanismo de flexión mantuvo su alineación bajo carga, validando la corrección de las placas de acero.



