\subsection{Implementación del sistema de seguridad mecánico (S3)}\label{Implementacion S3}
La seguridad mecánica constituye la segunda capa de protección del proyecto, actuando como una barrera física pasiva que protege la integridad del paciente y del mecanismo en caso de fallo del sistema de control, ya que, debido a la alta capacidad de par del actuador lineal seleccionado (Nema 23 con driver HSSS57) en relación con la rigidez estructural del mecanismo, existía un riesgo de daño mecánico en caso de colisión por fallo de control.

El sistema se compone de dos módulos: M3 y M4, cuya implementación se describe a continuación.
\subsubsection{Implementación de topes mecánicos (M3)}
Se integraron al sistema estructural los topes mecánicos que impiden el descarrilamiento del mecanismo de flexión-extensión, o la apertura excesiva del ángulo de abducción-aducción más allá de los límites permitidos ($0^\circ$ a $40^\circ$). Los topes fueron realizados con el material PTR, mismo que comparte con la estructura de la cama soporte, cuyo perfil se mostró en la Fig. \ref{fig:perfil_ptr}.

Para el movimiento de flexión-extensión se cuenta con dos topes mecánicos que limitan su rango de movimiento. 

\begin{figure}[h!]
	\centering
	\subcaptionbox{Topes para movimiento lineal.}
	{\includegraphics [trim = 0 0cm 0 0, clip,width=6cm]{figure/img_estructura/topes_flexion-extension.png}}
	\subcaptionbox{Tope para movimiento angular.}
	{\includegraphics [trim = 0 0cm 0 0, clip,width=6cm]{figure/img_estructura/Union_rod3.jpg}}
	\caption{Topes para mecanismo de flexión-extensión.\label{fig:union_rodilla}}
\end{figure}

\begin{figure}[h!]
	\centering
	\includegraphics[width=0.6\textwidth]{figure/img_componentes/Untitled.png}
	\caption{Tope mecánico para mecanismo de flexión-extensión.\label{fig:tope_mecanico_flexion}}
\end{figure}
\begin{figure}[h!]
	\centering
	\includegraphics[width=0.6\textwidth]{figure/img_componentes/Untitled.png}
	\caption{Tope mecánico para mecanismo de abducción-aducción.\label{fig:tope_mecanico_abduccion}}
\end{figure}
% Debo insertar imágenes sobre los topes mecánicnos implementados.

\subsubsection{Implementación de módulos de sujeción (M4)}
Los soportes directos para la pierna y pie se fabricaron mediante impresión 3D (manufactura aditiva) utilizando PLA para permitir geometrías ergonómicas complejas que no serían viables por métodos sustractivos. Estos soportes se acoplaron a los tubos telescópicos de aluminio que forman parte del mecanismo de flexión-extensión, permitiendo el ajuste longitudinal.\\
Aunque el diseño contempla el uso de cintas de velcro para la sujeción final, durante la fase de pruebas funcionales se priorizó la validación del mecanismo de ajuste telescópico y la rigidez de los soportes impresos.

\subsection{Verificación de seguridad mecánica}

Se validó el funcionamiento de los topes mecánicos como medida de última instancia ante una falla de control.

\subsubsection{Prueba S3-01: Resistencia al impacto controlado}
\textbf{Objetivo:} Verificar que la estructura y los topes soportan el torque del motor sin sufrir daño permanente en caso de que el software no detenga el movimiento.
\\
\textbf{Justificación de Seguridad:} Debido al alto par del actuador Nema 23 seleccionado, existe el riesgo de que el motor venza la estructura si no se limita su avance.
\\
\textbf{Procedimiento:} Se deshabilitaron intencionalmente los límites de software y se comandó el motor hacia el extremo del riel a baja velocidad, provocando una colisión controlada contra el tope mecánico.
\\
\textbf{Resultados:} El tope mecánico detuvo el avance del mecanismo sin deformar la estructura ni descarrilarlo, validando la robustez de la barrera física como segunda capa de protección.

