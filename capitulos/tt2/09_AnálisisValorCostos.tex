\section{Análisis de valor y costos}
\label{sec:costos}

Para la determinación del costo total del prototipo, se realizó un desglose de los recursos materiales y componentes adquiridos durante la fase de implementación. Los costos se clasifican en tres categorías principales: dispositivos electrónicos, componentes mecánicos e insumos de manufactura.

\subsection{Costos de dispositivos electrónicos y control}

La Tabla \ref{tab:costos_electronicos} enlista los elementos activos encargados del procesamiento, control de movimiento e interfaz con el usuario. Se destaca la inversión en la interfaz HMI y los sistemas de actuación de lazo cerrado.

\begin{table}[h!]
	\centering
	\caption{Costos de dispositivos electrónicos y de control.}
	\label{tab:costos_electronicos}
	\renewcommand{\arraystretch}{1.2}
	\begin{tabular}{|l|c|r|}
		\hline
		\rowcolor[HTML]{C5D9F1} 
		\textbf{Componente} & \textbf{Cantidad} & \textbf{Costo Total (MXN)} \\ \hline
		Pantalla HMI Táctil 7" (Industrial) & 1 & \$ 5,000.00 \\ \hline
		Microcontrolador Raspberry Pi 4 & 1 & \$ 1,920.00 \\ \hline
		Kit Motor Nema 23 + Driver HSS57 & 1 & \$ 3,238.00 \\ \hline
		Fuente de Poder 48V (LRS-600-48) & 1 & \$ 1,039.00 \\ \hline
		Fuente de Poder 5V (LRS-50-5) & 1 & \$ 409.00 \\ \hline
		Sensores y Finales de Carrera & 1 & \$ 675.00 \\ \hline
		Componentes de Seguridad (Paro/Relé) & 1 & \$ 200.00 \\ \hline
		\textbf{Subtotal Electrónica} & & \textbf{\$ 12,481.00} \\ \hline
	\end{tabular}
\end{table}

\subsection{Costos de componentes mecánicos y estructurales}

En la Tabla \ref{tab:costos_mecanicos} se detallan los costos asociados a la estructura de soporte, mecanismos de transmisión y el actuador principal de abducción.

\begin{table}[h!]
	\centering
	\caption{Costos de componentes mecánicos.}
	\label{tab:costos_mecanicos}
	\renewcommand{\arraystretch}{1.2}
	\begin{tabular}{|l|c|r|}
		\hline
		\rowcolor[HTML]{F2DCDB} 
		\textbf{Componente} & \textbf{Cantidad} & \textbf{Costo Total (MXN)} \\ \hline
		Kit Motor Nema 34 + Driver HSS86 & 1 & \$ 6,000.00 \\ \hline
		Actuador Lineal (Mecánica) & 1 & \$ 6,098.00 \\ \hline
		Perfiles Estructurales (PTR y Aluminio) & Lote & \$ 5,800.00 \\ \hline
		Placas de Acero y Maquinados & Lote & \$ 3,800.00 \\ \hline
		Rodamientos y Chumaceras & 4 & \$ 1,140.00 \\ \hline
		Elementos de Transmisión (Coples/Ejes) & Lote & \$ 2,200.00 \\ \hline
		\textbf{Subtotal Mecánica} & & \textbf{\$ 25,038.00} \\ \hline
	\end{tabular}
\end{table}

\subsection{Costos de insumos y manufactura}

Finalmente, la Tabla \ref{tab:costos_insumos} contempla los materiales consumibles utilizados para el ensamble, cableado y acabados estéticos, así como los elementos de sujeción (tornillería).

\begin{table}[h!]
	\centering
	\caption{Costos de insumos, herramientas y manufactura.}
	\label{tab:costos_insumos}
	\renewcommand{\arraystretch}{1.2}
	\begin{tabular}{|l|c|r|}
		\hline
		\rowcolor[HTML]{EBF1DE} 
		\textbf{Concepto} & \textbf{Descripción} & \textbf{Costo (MXN)} \\ \hline
		Tornillería Diversa & M3, M4, M5, M8 & \$ 1,200.00 \\ \hline
		Cableado y Conectores & Calibre 16/18, UTP, Terminales & \$ 800.00 \\ \hline
		Insumos de Soldadura y Pintura & Electrodos, Lijas, Aerosol & \$ 600.00 \\ \hline
		Material de Impresión 3D & Filamento PLA/PETG & \$ 800.00 \\ \hline
		Servicios Externos (Estimado) & Costura, Cortes especiales & \$ 1,500.00 \\ \hline
		\textbf{Subtotal Insumos} & & \textbf{\$ 4,900.00} \\ \hline
	\end{tabular}
\end{table}

\subsection{Costo total del proyecto}

El costo total de implementación del prototipo funcional asciende a:

\begin{table}[h!]
	\centering
	\caption{Resumen del costo total del prototipo.}
	\label{tab:resumen_total}
	\begin{tabular}{|l|r|}
		\hline
		\rowcolor[HTML]{DAEEF3} 
		\textbf{Categoría} & \textbf{Monto (MXN)} \\ \hline
		Sistemas Electrónicos y Control & \$ 12,481.00 \\ \hline
		Sistemas Mecánicos y Actuación & \$ 25,038.00 \\ \hline
		Insumos y Manufactura & \$ 4,900.00 \\ \hline
		\textbf{COSTO TOTAL} & \textbf{\$ 42,419.00} \\ \hline
	\end{tabular}
\end{table}

Este monto representa el costo de materiales y servicios directos para la fabricación de una unidad prototipo, excluyendo los costos de ingeniería y diseño (horas-hombre).