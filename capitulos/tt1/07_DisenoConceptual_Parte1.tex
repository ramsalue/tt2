\section{Diseño conceptual}
\begin{table}[h!]
	\centering
	\caption{Necesidades funcionales y no funcionales para el proyecto de la ortesis robótica.\label{tab01:tabla_necesidades}}
	\scalebox{0.7}{\begin{tabular}{|m{10.5cm}|m{8.5cm}|}
		\hline
		\multicolumn{1}{|c|}{\textbf{Necesidades   funcionales}}                                                                                       & \multicolumn{1}{c|}{\textbf{Necesidades no funcionales}}                                                         \\ \hline
		{Realizar movimientos de las   articulaciones de coxofemoral y rodilla para rehabilitar la región inferior   derecha.} & {Contar con modularidad en estructura mecánica   para adaptarla a diferentes pacientes.} \\
		&                                                                                                         \\ \hline
		Ajustar   y asignar los movimientos que determine el fisioterapeuta, de acuerdo con   cada sesión.                                    & Visualizar parámetros de salida   con ayuda de una interfaz humano-máquina.                                    \\ \hline
		Ingresar los parámetros de movimiento por medio de una interfaz humano-máquina.                                                            & Contar   con ergonomía y comodidad para el paciente.                                                                                                         \\ \hline
		Realizar   personalización de parámetros como la velocidad y rango de movimiento de la   ortesis.                                     &                                                                                                         \\ \hline
		Registrar   información por cada sesión de rehabilitación.                                                                            &                                                                                                         \\ \hline
		Establecer   límites al sistema para la seguridad del paciente.                                                                       &                                                                                                         \\ \hline
		Implementar   un botón de paro general en caso de emergencia.                                                                         &                                                                                                         \\ \hline
		Mantener   una postura rígida en el tobillo.                                                                                          &                                                                                                         \\ \hline
	\end{tabular}}
\end{table}
\subsection{Necesidades y requerimientos}
En la tabla \ref{tab01:tabla_necesidades} se clasifican las necesidades en funcionales y no funcionales para el sistema que se plantea en el presente proyecto.

La información descrita en la tabla \ref{tab01:tabla_necesidades} fue identificada con base en la entrevista realizada al Licenciado en Terapia Física Mario Sánchez Aguilar, de
la Unidad de Medicina Física y Rehabilitación del Norte–IMSS en la alcaldía Gustavo
A. Madero de la Ciudad de México.
\begin{table}[h!]
	\centering
	\caption{Requerimientos para el proyecto de la ortesis robótica. \label{tab02:requerimientos}}
	\scalebox{0.9}{\begin{tabular}{|m{5.5cm}|m{5.5cm}|m{4.5cm}|}
			\hline
			\textbf{Requerimiento}                                       & \textbf{Variable}                                    & \textbf{Valor(es)}                                       \\ \hline
			\multirow{4}{*}{Cama}                      & Largo (X)                                            & 180 cm - 250 cm                                             \\ \cline{2-3} 
			& Ancho   (Y)                                          & 80 cm - 100 cm                                               \\ \cline{2-3} 
			& Alto (Z)                                             & 60 cm – 100 cm                                            \\ \cline{2-3} 
			& Carga   máxima                                       & 100 kg                                                   \\ \hline
			\multirow{11}{*}{Mecanismo flexión/extensión} & Dimensión fémur (D1)                                 & 34 cm – 56 cm                                            \\ \cline{2-3} 
			& Dimensión   tibia (D2)                               & 31 cm – 48 cm                                            \\ \cline{2-3} 
			& Largo   total (X1)                                   & 100 cm - 120 cm                                                   \\ \cline{2-3} 
			& Ancho   total (Y1)                                   & 25 cm – 50 cm                                            \\ \cline{2-3} 
			& Soporte   vertical de pie                            & 20cm - 35 cm                                             \\ \cline{2-3} 
			& Soporte   Horizontal de pie                          & 25 cm – 50 cm                                            \\ \cline{2-3} 
			& Angulo   de dorsiflexión                             & 0° -  20°                                                \\ \cline{2-3} 
			& Angulo   de plantiflexión                            & 0° -  30°                                                \\ \cline{2-3} 
			& Angulo   de flexión de cadera con rodilla flexionada & 0° - 50°                                                 \\ \cline{2-3} 
			& Carga   máxima                                       & 20 kg                                                    \\ \hline
			\multirow{2}{*}{Mecanismo abducción/aducción} & Ángulo máximo ($\theta$)                                    & 0° - 50°                                                 \\ \cline{2-3} 
			& Carga   máxima                                       & 30 kg - 50 kg                                            \\ \hline
			\multirow{2}{*}{Interfaz de usuario}                         & Tipo                                                 & Adecuado para el usuario                                 \\ \cline{2-3} 
			& Información   visible                                & {[}Definir{]}                                            \\ \hline
			Suministro eléctrico                                         & Voltaje                                              & 127 V                                                                                       \\ \hline
	\end{tabular}}
\end{table}

\begin{table}[h!]
	\centering
	\scalebox{0.9}{\begin{tabular}{|m{5.5cm}|m{5.5cm}|m{4.5cm}|}
			\hline
			\textbf{Requerimiento}                                       & \textbf{Variable}                                    & \textbf{Valor(es)}                                       \\ \hline
			Seguridad eléctrica                                          & Corriente máxima del sistema                         & Máximo permitido por el proveedor   de energía eléctrica \\ \hline
			Seguridad mecánica                                           & Presencia de topes mecánicos                         & Si                                                       \\ \hline
			\multirow{4}{*}{Características del paciente}                & Edad del paciente                                    & \textgreater 18 años                                     \\ \cline{2-3} 
			& Tipo de   Hemiplejia                                 & Derecha                                                  \\ \cline{2-3} 
			& Etapa de   Hemiplejia                                & Flacidez                                                 \\ \cline{2-3} 
			& Tono   muscular                                      & Escala de Ashworth = 0 \\ \hline
			\multirow{2}{*}{Modo de uso para el paciente}                & Subida a la cama                                     & Del lado izquierdo de la cama                     \\ \cline{2-3} 
			& Posición                                             & Totalmente recostado                                     \\ \hline
			\multirow{2}{*}{Sujeción de la pierna}                       & Tipo de fijación                                     & Cintas de velcro                                         \\ \cline{2-3} 
			& Posición   de cintas                                 & Arriba de rodilla, debajo de   rodilla, y tobillo.       \\ \hline
			\multirow{2}{*}{Control de movimientos}                      & Tipo de control                                      & Manual y/o programado                                    \\ \cline{2-3} 
			& Velocidad   de movimiento                            & 1 cm/s - 5 cm/s                                           \\ \hline	
		\multirow{2}{*}{Higiene}                                     & Materiales para soporte del pie                      & Plástico, aluminio o yeso                            \\ \cline{2-3} 
		& Materiales   de contacto con el pie                  & Algodón plisado, foami o colchón                          \\ \hline
	\end{tabular}}
\end{table}

A partir de las necesidades descritas, se plantean bajo un enfoque funcional los requerimientos del proyecto, los cuales se presentan en la tabla \ref{tab02:requerimientos}.

\subsection{Arquitectura funcional}
Con base en los requerimientos descritos se muestra a continuación la descomposición de funciones con las cuales se plantea la propuesta de solución, siendo la función global \textbf{mover la extremidad inferior derecha}.
\clearpage
\tabto{0.0\textwidth} \textbf{F0.0.} Mover extremidad inferior derecha.\\
\tabto{0.06\textwidth} \textbf{F1.0.} Administrar energía al sistema.\\
\tabto{0.12\textwidth} \textbf{F1.1.} Implementar etapa de potencia.\\
\tabto{0.12\textwidth} \textbf{F1.2.} Implementar acondicionamiento de energía para el sistema.\\
\tabto{0.12\textwidth} \textbf{F1.3.} Implementar paro de emergencia.\\
\tabto{0.06\textwidth} \textbf{F2.0.} Soportar peso del paciente.\\
\tabto{0.06\textwidth} \textbf{F3.0.} Ajustar ortesis a la pierna derecha del paciente.\\
\tabto{0.12\textwidth} \textbf{F3.1.} Ajustar la estructura al fémur del paciente.\\
\tabto{0.12\textwidth} \textbf{F3.2.} Ajustar la estructura a la tibia del paciente.\\
\tabto{0.12\textwidth} \textbf{F3.3.} Ajustar ángulo de tobillo del paciente.\\
\tabto{0.06\textwidth} \textbf{F4.0.} Sujetar pierna del paciente.\\
\tabto{0.12\textwidth} \textbf{F4.1.} Sujetar fémur.\\
\tabto{0.12\textwidth} \textbf{F4.2.} Sujetar tibia.\\
\tabto{0.12\textwidth} \textbf{F4.3.} Sujetar tobillo.\\
\tabto{0.06\textwidth} \textbf{F5.0.} Comunicar al usuario con el sistema (HMI).\\
\tabto{0.12\textwidth} \textbf{F5.1.} Acceder a modos trabajo.\\
\tabto{0.18\textwidth} \textbf{F5.1.1.} Realizar movimientos de forma manual.\\
\tabto{0.18\textwidth} \textbf{F5.1.2.} Realizar movimientos de forma programada.\\
\tabto{0.24\textwidth} \textbf{F5.1.2.1.} Programar ejercicios.\\
\tabto{0.30\textwidth} \textbf{F5.1.2.1.1.} Ingresar rutina de ejercicios.\\
\tabto{0.24\textwidth} \textbf{F5.1.2.2.} Iniciar rutina de ejercicios.\\
\tabto{0.24\textwidth} \textbf{F5.1.2.3.} Monitorear progreso de ejercicios.\\
\tabto{0.24\textwidth} \textbf{F5.1.2.4.} Mostrar trayectoria de actuadores.\\
\tabto{0.24\textwidth} \textbf{F5.1.2.5.} Finalizar rutina de ejercicios.\\
\tabto{0.06\textwidth} \textbf{F6.0.} Comunicar internamente los datos de control del sistema.\\
\tabto{0.12\textwidth} \textbf{F6.1.} Recibir datos de la interfaz humano-máquina.\\
\tabto{0.12\textwidth} \textbf{F6.2.} Enviar datos a la interfaz humano-máquina.\\
\tabto{0.12\textwidth} \textbf{F6.3.} Enviar señales a actuadores del sistema.\\
\tabto{0.12\textwidth} \textbf{F6.4.} Recibir señales de sensores del sistema.\\
\tabto{0.06\textwidth} \textbf{F7.0.} Implementar seguimiento de trayectorias.\\
\tabto{0.06\textwidth} \textbf{F8.0.} Realizar movimientos de articulaciones.\\
\tabto{0.12\textwidth} \textbf{F8.1.} Realizar movimientos de articulaciones en el plano sagital.\\
\tabto{0.18\textwidth} \textbf{F8.1.1.} Accionar motor lineal.\\
\tabto{0.24\textwidth} \textbf{F8.1.1.1.} Realizar flexión de la articulación coxofemoral.\\
\tabto{0.24\textwidth} \textbf{F8.1.1.2.} Realizar extensión de la articulación coxofemoral.\\
\tabto{0.24\textwidth} \textbf{F8.1.1.3.} Realizar flexión de la articulación de la rodilla.\\
\tabto{0.24\textwidth} \textbf{F8.1.1.4.} Realizar extensión de la articulación de la rodilla.\\
\tabto{0.12\textwidth} \textbf{F8.2.} Realizar movimientos de articulaciones en el plano coronal.\\
\tabto{0.18\textwidth} \textbf{F8.2.1.} Accionar motor rotativo.\\
\tabto{0.24\textwidth} \textbf{F8.2.1.1.} Realizar abducción de la articulación coxofemoral.\\
\tabto{0.24\textwidth} \textbf{F8.2.1.2.} Realizar aducción de la articulación coxofemoral.\\


Se identifican diversas funciones principales para el funcionamiento de la ortesis robótica. La función F0.0 implica la capacidad del sistema para mover la extremidad inferior derecha del paciente de acuerdo con las instrucciones recibidas. La función F1.0 se encarga de administrar la energía necesaria para todas las operaciones del sistema, incluyendo la implementación de la etapa de potencia (F1.1) y el acondicionamiento de energía (F1.2), así como la implementación del paro de emergencia (F1.3) para mantener la seguridad del usuario. La función F2.0 asegura que la ortesis pueda soportar el peso del paciente durante el uso.

En cuanto a la función F3.0, se refiere a ajustar la ortesis específicamente a la pierna derecha del paciente, incluyendo el ajuste estructural al fémur (F3.1) y a la tibia (F3.2), así como el ajuste del ángulo del tobillo (F3.3).

La función F4.0 implica la sujeción adecuada de la pierna del paciente mediante el soporte del fémur (F4.1), tibia (F4.2) y tobillo (F4.3) de forma que pueda proporcionarse estabilidad y seguridad durante el uso. La función F5.0 implica la comunicación entre el usuario y el sistema a través de una interfaz humano-máquina (HMI), permitiendo acceso a diversos modos de trabajo (F5.1) que incluyen movimientos manuales (F5.1.1) y programados (F5.1.2), con la capacidad de programar ejercicios (F5.1.2.1), iniciar y monitorear rutinas de ejercicios (F5.1.2.2, F5.1.2.3), y mostrar trayectorias de actuadores (F5.1.2.2).

Internamente, la función F6.0 gestiona la comunicación de datos de control del sistema, recibiendo datos de la interfaz HMI (F6.1), enviando datos a la misma (F6.2), y controlando actuadores (F6.3) y recibiendo información de sensores (F6.4), para que de esta forma, la ortesis pueda implementar el seguimiento de trayectorias (F7.0) y realizar movimientos específicos de articulaciones en los planos sagital (F8.1) y coronal (F8.2), utilizando motores lineales (F8.1.1) y rotativos (F8.2.1) para flexión y extensión de la articulación coxofemoral y de la rodilla, así como abducción y aducción de la articulación coxofemoral respectivamente.

En la Figura \ref{fig:Diagrama_fbs} se presenta el modelo FBS que representa de forma gráfica y jerárquica las funciones descritas. 
\begin{figure}[h!]
	\centering
	\includegraphics [trim = 0 0cm 0 0, clip, angle=90, width=0.55\textwidth]{figure/P01_fbs.png}
	\caption[Estructura FBS.]{Estructura FBS para la representación gráfica y jerárquica de las funciones de la arquitectura funcional del proyecto de la ortesis robótica.\label{fig:Diagrama_fbs}}
\end{figure}

\subsubsection{IDEF-0}
Esta herramienta se utiliza para analizar las funciones del sistema, lo que permite comprender las interacciones entre ellas, así como descomponerlas y representarlas en diagramas sencillos que muestran entradas, salidas, controles y mecanismos en cada función. En la Fig. \ref{fig:Diagrama_nodoA0} se presenta el diagrama del nodo A0 mientras que en la Fig. \ref{fig:Diagrama_nodoA0_Ext} se tiene el diagrama del nodo A0 extendido.
\begin{figure}[h!]
	\centering
	\includegraphics [trim = 0 0cm 0 0, clip, angle = 90, width=0.38\textwidth]{figure/P02.jpg}
	\caption[Diagrama del nodo A0.]{Diagrama del nodo A0 que representa de forma general las entradas y salidas, controles y mecanismos requeridos para cada función.\label{fig:Diagrama_nodoA0}}
\end{figure}
\begin{figure}[h!]
	\centering
	\includegraphics [angle = 90, trim = 0 0cm 0 0, clip, width=0.65\textwidth]{figure/P03.png}
	\caption[Diagrama del nodo A0 extendido.]{Diagrama del nodo A0 extendido que muestra la interacción interna de entradas, salidas, controles, y mecanismos requeridos para cada función.\label{fig:Diagrama_nodoA0_Ext}}
\end{figure}
\clearpage
\subsection{Arquitectura física}
En esta sección se proponen los distintos sistemas y módulos que se asocian a las funciones.
\begin{itemize}
	\item S1. Sistema estructural.
	\item S2. Sistema de seguridad eléctrico.
	\begin{itemize}
		\item M1. Módulo de protección por sobrecorriente.
		\item M2. Módulo de protección por sobrevoltaje.
	\end{itemize}
	\item S3. Sistema de seguridad mecánico.
	\begin{itemize}
		\item M3. Módulo de topes mecánicos.
		\item M4. Módulo de sujeción y ajuste.
	\end{itemize}
	\item S4. Sistema de energía.
	\begin{itemize}
		\item M5. Módulo de paro de emergencia.
		\item M6. Módulo de etapa de potencia.
		\item M7. Módulo de acondicionamiento de energía.
	\end{itemize}
	\item S5. Sistema de movimiento.
	\begin{itemize}
		\item M8. Módulo de actuador lineal.
		\item M9. Módulo de actuador rotativo.
	\end{itemize}
	\item S6. Sistema de comunicación Humano–Máquina.
	\begin{itemize}
		\item M10. Módulo E/S.
		\item M11. Módulo de almacenamiento.
	\end{itemize}
	\item S7. Sistema de control.
	\begin{itemize}
		\item M12. Módulo de sensores.
		\item M13. Módulo de acondicionamiento de señales.
		\item M14. Módulo de procesamiento.
	\end{itemize}
\end{itemize}

Para la arquitectura física de la ortesis robótica se han definido siete sistemas principales que organizan y agrupan los diferentes componentes del dispositivo. A continuación se describen en párrafos separados los diferentes sistemas de la arquitectura física, detallando sus módulos, entradas, salidas y funciones principales:

\paragraph{Sistema estructural (S1).} 
El sistema estructural constituye la base del conjunto, proporcionando soporte y estabilidad a todos los subsistemas. Aunque no presenta módulos específicos definidos en el esquema, su función principal es soportar las cargas mecánicas y garantizar que la integridad física del sistema completo se mantenga, permitiendo que los demás sistemas operen. Sus salidas se reflejan en la capacidad de sostener y distribuir adecuadamente las fuerzas generadas y recibidas en el entorno operacional.

\paragraph{Sistema de seguridad eléctrico (S2).}  
Este sistema se encarga de proteger los componentes eléctricos ante sobrecargas y anomalías en el suministro de energía. Está conformado por módulos como la protección por sobrecorriente (M1) y la protección por sobrevoltaje (M2), los cuales reciben energía eléctrica como entrada. Sus funciones principales son prevenir daños en los equipos eléctricos y buscar la estabilidad del flujo energético. Las salidas del sistema consisten en la activación de medidas de protección y la señalización de fallos, procurando que el sistema opere dentro de los parámetros que se establezcan.

\paragraph{Sistema de seguridad mecánico (S3).}  
Diseñado para asegurar la integridad física del paciente y operatividad del conjunto, este sistema integra módulos como los topes mecánicos (M3) y los mecanismos de sujeción y ajuste (M4). La función más importante es evitar que el paciente sufra alguna lesión o molestia causada por fallas en el control del sistema,, actuando como una barrera protectora contra impactos y vibraciones no deseados. De igual manera, este sistema se encarga de ajustar la ortesis al tamaño de la pierna de cada paciente y la mantiene sujetada a la ortesis para evitar algún movimiento no deseado.

\paragraph{Sistema de energía (S4).}
El sistema de energía es el encargado de suministrar, regular y acondicionar la electricidad necesaria para el funcionamiento de todos los subsistemas. Recibe energía eléctrica como entrada y está compuesto por módulos como el paro de emergencia (M5), la etapa de potencia (M6) y el acondicionamiento de energía (M7). Su función principal es garantizar un suministro estable, protegiendo el sistema de posibles fluctuaciones y optimizando el uso de la energía. Como salidas, este sistema produce energía condicionada y controlada para su distribución a los sistemas de acción y control.

\paragraph{Sistema de movimiento (S5).}  
Este sistema está diseñado para generar y controlar los desplazamientos de los elementos activos del conjunto. Funciona mediante la integración de módulos como el de actuador lineal (M8) para movimientos en línea recta y el de actuador rotativo (M9) para movimientos de giro. Las entradas para este sistema son la energía eléctrica y, en ciertos casos, materia o elementos de fricción que permiten la transmisión del movimiento. Su función principal es ejecutar y regular los desplazamientos de los mecanismos, produciendo como salidas el movimiento de los actuadores.

\paragraph{Sistema de comunicación humano-máquina (S6).}
Orientado a facilitar la interacción entre el usuario y el sistema, este subsistema emplea módulos dedicados a las operaciones de entrada/salida (M10) y al almacenamiento de la información (M11). Sus entradas están compuestas principalmente por datos e información provenientes del usuario, mientras que las salidas incluyen respuestas del sistema, señales de retroalimentación e informes de estado. La función principal es traducir las intenciones y comandos humanos en acciones técnicas operativas, asegurando que la información fluya de manera bidireccional.

\paragraph{Sistema de control (S7).}  
El sistema de control es el núcleo encargado de coordinar, supervisar y regular el funcionamiento integral de la arquitectura. Está compuesto por módulos que incluyen sensores (M12), acondicionamiento de señales (M13) y procesamiento (M14). Recibe entradas en forma de información derivada de los diversos sensores y de energía eléctrica para alimentar sus procesos de cómputo. Las salidas de este sistema se expresan en comandos de control y ajustes operativos que influyen en todos los subsistemas para lograr el control del sistema.

En el diagrama de la Fig. \ref{fig:Arquitectura_Fisica} se muestra la interacción entre los sistemas y módulos que realizarán las funciones generales de la ortesis.
\begin{figure}[h!]
	\centering
	\includegraphics [trim = 0 0cm 0 0, clip, width=0.92\textwidth]{figure/P04.png}
	\caption[Arquitectura física.]{Arquitectura física. Se muestra la interacción de los módulos y sistemas propuestos para el proyecto de la ortesis robótica.\label{fig:Arquitectura_Fisica}}
\end{figure}

