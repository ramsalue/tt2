\section{Diseño detallado}\label{Diseño detallado}
El desarrollo detallado del proyecto, comenzó con el diseño de la estructura sobre la cual el paciente sería posicionado, además de que sobre este sistema serían montados todos los demás sistemas, por lo que se tomó como el punto de partida. 
\subsection{S1. Sistema estructural}
A continuación, se presenta el diseño detallado de la estructura y selección de materiales.
	 \paragraph*{Estructura para soporte de región inferior derecha}
	 Para la selección del perfil estructural del sistema se evaluaron cuatro alternativas: tubos circulares, tubos cuadrados, soleras y perfil Bosch. En la tabla \ref{ms_perfiles} se presenta la matriz de selección subjetiva, considerando criterios como costo, peso, maquinabilidad, ensamble, estética, protección, adaptabilidad, posibilidad telescópica y mantenimiento, los cuales fueron ponderados con base en una escala:
	 \begin{enumerate}
	 	\item Pésimo.
	 	\item Malo.
	 	\item Regular.
	 	\item Bueno.
	 	\item Excelente.
	 \end{enumerate}
	 \begin{table}[h!]
	 	\centering
	 	\caption{Matriz de selección subjetiva para perfiles de la estructura para soporte de región inferior derecha.\label{ms_perfiles}}
	 	\begin{tabular}{|
	 			>{\columncolor[HTML]{FFFFFF}}c 
	 			>{\columncolor[HTML]{FFFFFF}}c |cccc|}
	 		\hline
	 		\multicolumn{2}{|c|}{\cellcolor[HTML]{FFFFFF}}                                                                                     & \multicolumn{4}{c|}{\cellcolor[HTML]{C0C0C0}\textbf{Conceptos}}                                                                                                                                                                                                                                                                                      \\ \cline{3-6} 
	 		\multicolumn{2}{|c|}{\multirow{-2}{*}{\cellcolor[HTML]{FFFFFF}\textbf{Criterios}}}                                                 & \multicolumn{1}{c|}{\cellcolor[HTML]{FFFFFF}\textbf{\begin{tabular}[c]{@{}c@{}}Tubos \\ circulares\end{tabular}}} & \multicolumn{1}{c|}{\cellcolor[HTML]{FFFFFF}\textbf{\begin{tabular}[c]{@{}c@{}}Tubos \\ cuadrados\end{tabular}}} & \multicolumn{1}{c|}{\cellcolor[HTML]{FFFFFF}\textbf{Soleras}} & \cellcolor[HTML]{FFFFFF}\textbf{Perfil Bosch} \\ \hline
	 		\multicolumn{1}{|c|}{\cellcolor[HTML]{FFFFFF}\textbf{A}} & Costo                                                                   & \multicolumn{1}{c|}{Bueno}                                                                                        & \multicolumn{1}{c|}{Bueno}                                                                                       & \multicolumn{1}{c|}{Excelente}                                & Malo                                          \\ \hline
	 		\multicolumn{1}{|c|}{\cellcolor[HTML]{FFFFFF}\textbf{B}} & Peso                                                                    & \multicolumn{1}{c|}{Excelente}                                                                                    & \multicolumn{1}{c|}{Excelente}                                                                                   & \multicolumn{1}{c|}{Malo}                                     & Bueno                                         \\ \hline
	 		\multicolumn{1}{|c|}{\cellcolor[HTML]{FFFFFF}\textbf{C}} & Maquinabilidad                                                          & \multicolumn{1}{c|}{Regular}                                                                                      & \multicolumn{1}{c|}{Excelente}                                                                                   & \multicolumn{1}{c|}{Excelente}                                & Excelente                                     \\ \hline
	 		\multicolumn{1}{|c|}{\cellcolor[HTML]{FFFFFF}\textbf{D}} & Ensamble                                                                & \multicolumn{1}{c|}{Malo}                                                                                         & \multicolumn{1}{c|}{Excelente}                                                                                   & \multicolumn{1}{c|}{Excelente}                                & Excelente                                     \\ \hline
	 		\multicolumn{1}{|c|}{\cellcolor[HTML]{FFFFFF}\textbf{E}} & Estética                                                                & \multicolumn{1}{c|}{Excelente}                                                                                    & \multicolumn{1}{c|}{Bueno}                                                                                       & \multicolumn{1}{c|}{Malo}                                     & Bueno                                         \\ \hline
	 		\multicolumn{1}{|c|}{\cellcolor[HTML]{FFFFFF}\textbf{F}} & Protección                                                              & \multicolumn{1}{c|}{Excelente}                                                                                    & \multicolumn{1}{c|}{Bueno}                                                                                       & \multicolumn{1}{c|}{Malo}                                     & Regular                                       \\ \hline
	 		\multicolumn{1}{|c|}{\cellcolor[HTML]{FFFFFF}\textbf{G}} & \begin{tabular}[c]{@{}c@{}}Adaptabilidad de\\  componentes\end{tabular} & \multicolumn{1}{c|}{Regular}                                                                                      & \multicolumn{1}{c|}{Excelente}                                                                                   & \multicolumn{1}{c|}{Excelente}                                & Excelente                                     \\ \hline
	 		\multicolumn{1}{|c|}{\cellcolor[HTML]{FFFFFF}\textbf{H}} & Telescópico                                                             & \multicolumn{1}{c|}{Excelente}                                                                                    & \multicolumn{1}{c|}{Excelente}                                                                                   & \multicolumn{1}{c|}{Pésimo}                                   & Regular                                       \\ \hline
	 		\multicolumn{1}{|c|}{\cellcolor[HTML]{FFFFFF}\textbf{I}} & Mantenimiento                                                           & \multicolumn{1}{c|}{Bueno}                                                                                        & \multicolumn{1}{c|}{Bueno}                                                                                       & \multicolumn{1}{c|}{Excelente}                                & Regular                                       \\ \hline
	 	\end{tabular}
	 \end{table}
	 
	 Posteriormente, la tabla \ref{mb_perfiles} muestra la matriz binaria en la que se compararon los criterios entre sí para determinar su importancia relativa, esto con la finalidad de evaluar la importancia de cada criterio contra los demás, es decir, en la primera fila tenemos al criterio A y se evaluó que tan importante es respecto a los demás criterios, de esta manera podemos ver que en la primera columna se colocan números 1 debido a que también es el criterio A, pero en la segunda columna encontramos al criterio B y aquí no se colocan números 1 porque el criterio A es de mayor importancia que el criterio B. Por lo tanto, la matriz quedó de la siguiente manera:
	 \begin{table}[h!]
	 	\centering
	 	\caption{Matriz binaria para perfiles de la estructura.\label{mb_perfiles}}
	 	\scalebox{0.95}{\begin{tabular}{|
	 			>{\columncolor[HTML]{FFFFFF}}c 
	 			>{\columncolor[HTML]{FFFFFF}}c |c|c|c|c|c|c|c|c|c|c|c|}
	 		\hline
	 		\multicolumn{2}{|c|}{\cellcolor[HTML]{FFFFFF}\textbf{Criterios}}                                                                  & \cellcolor[HTML]{FFFFFF}\textbf{A} & \cellcolor[HTML]{FFFFFF}\textbf{B} & \cellcolor[HTML]{FFFFFF}\textbf{C} & \cellcolor[HTML]{FFFFFF}\textbf{D} & \cellcolor[HTML]{FFFFFF}\textbf{E} & \cellcolor[HTML]{FFFFFF}\textbf{F} & \cellcolor[HTML]{FFFFFF}\textbf{G} & \cellcolor[HTML]{FFFFFF}\textbf{H} & \cellcolor[HTML]{FFFFFF}\textbf{I} & \cellcolor[HTML]{FFFFFF}\textbf{Total} & \cellcolor[HTML]{FFFFFF}\textbf{Rango} \\ \hline
	 		\multicolumn{1}{|c|}{\cellcolor[HTML]{FFFFFF}\textbf{A}} & Costo                                                                  & \cellcolor[HTML]{C0C0C0}           & 0                                  & 0                                  & 0                                  & 0                                  & 0                                  & 0                                  & 0                                  & 1                                  & 1                                      & 8                                      \\ \hline
	 		\multicolumn{1}{|c|}{\cellcolor[HTML]{FFFFFF}\textbf{B}} & Peso                                                                   & 1                                  & \cellcolor[HTML]{C0C0C0}           & 1                                  & 1                                  & 1                                  & 0                                  & 1                                  & 0                                  & 1                                  & 6                                      & 3                                      \\ \hline
	 		\multicolumn{1}{|c|}{\cellcolor[HTML]{FFFFFF}\textbf{C}} & Maquinabilidad                                                         & 1                                  & 0                                  & \cellcolor[HTML]{C0C0C0}           & 1                                  & 1                                  & 0                                  & 1                                  & 0                                  & 1                                  & 5                                      & 4                                      \\ \hline
	 		\multicolumn{1}{|c|}{\cellcolor[HTML]{FFFFFF}\textbf{D}} & Ensamble                                                               & 1                                  & 0                                  & 0                                  & \cellcolor[HTML]{C0C0C0}           & 1                                  & 0                                  & 1                                  & 0                                  & 1                                  & 4                                      & 5                                      \\ \hline
	 		\multicolumn{1}{|c|}{\cellcolor[HTML]{FFFFFF}\textbf{E}} & Estética                                                               & 1                                  & 0                                  & 0                                  & 0                                  & \cellcolor[HTML]{C0C0C0}           & 0                                  & 0                                  & 0                                  & 1                                  & 2                                      & 7                                      \\ \hline
	 		\multicolumn{1}{|c|}{\cellcolor[HTML]{FFFFFF}\textbf{F}} & Protección                                                             & 1                                  & 1                                  & 1                                  & 1                                  & 1                                  & \cellcolor[HTML]{C0C0C0}           & 1                                  & 1                                  & 1                                  & 8                                      & 1                                      \\ \hline
	 		\multicolumn{1}{|c|}{\cellcolor[HTML]{FFFFFF}\textbf{G}} & \begin{tabular}[c]{@{}c@{}}Adaptabilidad de\\ componentes\end{tabular} & 1                                  & 0                                  & 0                                  & 0                                  & 1                                  & 0                                  & \cellcolor[HTML]{C0C0C0}           & 0                                  & 1                                  & 3                                      & 6                                      \\ \hline
	 		\multicolumn{1}{|c|}{\cellcolor[HTML]{FFFFFF}\textbf{H}} & Telescópico                                                            & 1                                  & 1                                  & 1                                  & 1                                  & 1                                  & 0                                  & 1                                  & \cellcolor[HTML]{C0C0C0}           & 1                                  & 7                                      & 2                                      \\ \hline
	 		\multicolumn{1}{|c|}{\cellcolor[HTML]{FFFFFF}\textbf{I}} & Mantenimiento                                                          & 0                                  & 0                                  & 0                                  & 0                                  & 0                                  & 0                                  & 0                                  & 0                                  & \cellcolor[HTML]{C0C0C0}           & 0                                      & 9                                      \\ \hline
	 	\end{tabular}}
	 \end{table}
	 
	 Finalmente, en la tabla \ref{mp_perfiles} se aplicaron dichas ponderaciones para calcular los puntajes ponderados de cada alternativa estructural. Los resultados muestran que los tubos cuadrados representan la opción más equilibrada y favorable para la estructura de la ortesis.
	 
	 En la Fig. \ref{fig:estructura} se muestra la estructura que llevaría el perfil seleccionado a través de las matrices.
	 \begin{figure}[h!]
	 	\centering
	 	\includegraphics [trim = 0 0cm 0 0, clip, width=12cm]{figure/Imagenes conceptos/13.jpg}
	 	\caption[Estructura para soporte de región inferior derecha.]{Estructura para soporte de región inferior derecha.\label{fig:estructura}}
	 \end{figure}
	 \begin{table}[h!]
	 	\centering
	 	\caption{Matriz de ponderación para perfiles de la estructura.\label{mp_perfiles}}
	 	\scalebox{0.95}{\begin{tabular}{cc|c|c|c|c|c|c|}
	 		\hline
	 		\rowcolor[HTML]{FFFFFF} 
	 		\multicolumn{2}{|c|}{\cellcolor[HTML]{FFFFFF}\textbf{Criterios}}                                                                                          & Total                     & Ponderación                   & \textbf{\begin{tabular}[c]{@{}c@{}}Tubos\\ circulares\end{tabular}} & \textbf{\begin{tabular}[c]{@{}c@{}}Tubos\\ cuadrados\end{tabular}} & \textbf{Soleras} & \textbf{\begin{tabular}[c]{@{}c@{}}Perfil\\ Bosch\end{tabular}} \\ \hline
	 		\multicolumn{1}{|c|}{\cellcolor[HTML]{FFFFFF}\textbf{F}} & \cellcolor[HTML]{FFFFFF}Protección                                                             & \cellcolor[HTML]{FFFFFF}8 & \cellcolor[HTML]{FFFFFF}0.222 & 5                                                                   & 4                                                                  & 2                & 3                                                               \\ \hline
	 		\multicolumn{1}{|c|}{\cellcolor[HTML]{FFFFFF}\textbf{H}} & \cellcolor[HTML]{FFFFFF}Telescópico                                                            & \cellcolor[HTML]{FFFFFF}7 & \cellcolor[HTML]{FFFFFF}0.194 & 5                                                                   & 5                                                                  & 1                & 3                                                               \\ \hline
	 		\multicolumn{1}{|c|}{\cellcolor[HTML]{FFFFFF}\textbf{B}} & \cellcolor[HTML]{FFFFFF}Peso                                                                   & \cellcolor[HTML]{FFFFFF}6 & \cellcolor[HTML]{FFFFFF}0.167 & 5                                                                   & 5                                                                  & 2                & 4                                                               \\ \hline
	 		\multicolumn{1}{|c|}{\cellcolor[HTML]{FFFFFF}\textbf{C}} & \cellcolor[HTML]{FFFFFF}Maquinabilidad                                                         & \cellcolor[HTML]{FFFFFF}5 & \cellcolor[HTML]{FFFFFF}0.139 & 3                                                                   & 5                                                                  & 5                & 5                                                               \\ \hline
	 		\multicolumn{1}{|c|}{\cellcolor[HTML]{FFFFFF}\textbf{D}} & \cellcolor[HTML]{FFFFFF}Ensamble                                                               & \cellcolor[HTML]{FFFFFF}4 & \cellcolor[HTML]{FFFFFF}0.111 & 2                                                                   & 5                                                                  & 5                & 5                                                               \\ \hline
	 		\multicolumn{1}{|c|}{\cellcolor[HTML]{FFFFFF}\textbf{G}} & \cellcolor[HTML]{FFFFFF}\begin{tabular}[c]{@{}c@{}}Adaptabilidad de\\ componentes\end{tabular} & \cellcolor[HTML]{FFFFFF}3 & \cellcolor[HTML]{FFFFFF}0.083 & 3                                                                   & 5                                                                  & 5                & 5                                                               \\ \hline
	 		\multicolumn{1}{|c|}{\cellcolor[HTML]{FFFFFF}\textbf{E}} & \cellcolor[HTML]{FFFFFF}Estética                                                               & \cellcolor[HTML]{FFFFFF}2 & \cellcolor[HTML]{FFFFFF}0.056 & 5                                                                   & 4                                                                  & 2                & 4                                                               \\ \hline
	 		\multicolumn{1}{|c|}{\cellcolor[HTML]{FFFFFF}\textbf{A}} & \cellcolor[HTML]{FFFFFF}Costo                                                                  & \cellcolor[HTML]{FFFFFF}1 & \cellcolor[HTML]{FFFFFF}0.028 & 4                                                                   & 4                                                                  & 5                & 2                                                               \\ \hline
	 		\multicolumn{1}{|c|}{\cellcolor[HTML]{FFFFFF}\textbf{I}} & \cellcolor[HTML]{FFFFFF}Mantenimiento                                                          & \cellcolor[HTML]{FFFFFF}0 & \cellcolor[HTML]{FFFFFF}0.000 & 4                                                                   & 4                                                                  & 5                & 3                                                               \\ \hline
	 		\multicolumn{1}{l}{}                                     & \multicolumn{1}{l|}{}                                                                          & \multicolumn{1}{c|}{36}   & \multicolumn{1}{l|}{}         & 4.19                                                                & \cellcolor[HTML]{FFFFFF}4.69                                       & 2.89             & 3.86                                                            \\ \cline{3-8} 
	 	\end{tabular}}
	 \end{table}
	 
	\mbox{}\\
	 \subsubsection*{Uniones}
	 En la Fig. \ref{fig:uniones} se muestran perfiles de la estructura mostrada en la Fig. \ref{fig:estructura}. 
	 \begin{figure}[h!]
	 	\centering
	 	\includegraphics [trim = 0 0cm 0 0, clip, width=9cm]{figure/Imagenes conceptos/14.jpg}
	 	\caption[Uniones Clevis.]{Uniones Clevis (horquilla).\label{fig:uniones}}
	 \end{figure}
	 	 
	 Para determinar el tipo de unión más adecuado para unir el soporte de fémur, tubo telescópico de fémur y de tibia en la estructura, se consideraron cinco alternativas: rodamientos, bisagras, juntas Cardán, aplanado con pernos y juntas Clevis. En la tabla \ref{ms_uniones} se evaluaron cualitativamente en función de criterios como facilidad de rotación, estabilidad, ruido, mantenimiento, montaje y durabilidad. La escala tomada en cuenta fue: 
	 \begin{enumerate}
	 	\item Pésimo.
	 	\item Malo.
	 	\item Regular.
	 	\item Bueno.
	 	\item Excelente.
	 \end{enumerate}
	 \begin{table}[h!]
	 	\centering
	 	\caption{Matriz de selección subjetiva para las uniones de la estructura (parte superior).\label{ms_uniones}}
	 	\begin{tabular}{|
	 			>{\columncolor[HTML]{FFFFFF}}c 
	 			>{\columncolor[HTML]{FFFFFF}}c |ccccc|}
	 		\hline
	 		\multicolumn{2}{|c|}{\cellcolor[HTML]{FFFFFF}}                                     & \multicolumn{5}{c|}{\cellcolor[HTML]{C0C0C0}\textbf{Conceptos}}                                                                                                                                                                                                                                                                                                                                                                                                   \\ \cline{3-7} 
	 		\multicolumn{2}{|c|}{\multirow{-2}{*}{\cellcolor[HTML]{FFFFFF}\textbf{Criterios}}} & \multicolumn{1}{c|}{\cellcolor[HTML]{FFFFFF}\textbf{Rodamientos}} & \multicolumn{1}{c|}{\cellcolor[HTML]{FFFFFF}\textbf{Bisagras}} & \multicolumn{1}{c|}{\cellcolor[HTML]{FFFFFF}\textbf{\begin{tabular}[c]{@{}c@{}}Juntas \\ Cardán\end{tabular}}} & \multicolumn{1}{c|}{\cellcolor[HTML]{FFFFFF}\textbf{\begin{tabular}[c]{@{}c@{}}Aplanado\\ Pernos\end{tabular}}} & \cellcolor[HTML]{FFFFFF}\textbf{\begin{tabular}[c]{@{}c@{}}Juntas \\ Clevis\end{tabular}} \\ \hline
	 		\multicolumn{1}{|c|}{\cellcolor[HTML]{FFFFFF}\textbf{A}}      & Facil rotación     & \multicolumn{1}{c|}{Excelente}                                    & \multicolumn{1}{c|}{Regular}                                   & \multicolumn{1}{c|}{Excelente}                                                                                 & \multicolumn{1}{c|}{Regular}                                                                                    & Bueno                                                                                     \\ \hline
	 		\multicolumn{1}{|c|}{\cellcolor[HTML]{FFFFFF}\textbf{B}}      & Estabilidad        & \multicolumn{1}{c|}{Excelente}                                    & \multicolumn{1}{c|}{Regular}                                   & \multicolumn{1}{c|}{\cellcolor[HTML]{FFFFFF}Malo}                                                              & \multicolumn{1}{c|}{Regular}                                                                                    & Excelente                                                                                 \\ \hline
	 		\multicolumn{1}{|c|}{\cellcolor[HTML]{FFFFFF}\textbf{C}}      & Ruido              & \multicolumn{1}{c|}{\cellcolor[HTML]{FFFFFF}Bueno}                & \multicolumn{1}{c|}{\cellcolor[HTML]{FFFFFF}Regular}           & \multicolumn{1}{c|}{\cellcolor[HTML]{FFFFFF}Regular}                                                           & \multicolumn{1}{c|}{\cellcolor[HTML]{FFFFFF}Regular}                                                            & \cellcolor[HTML]{FFFFFF}Bueno                                                             \\ \hline
	 		\multicolumn{1}{|c|}{\cellcolor[HTML]{FFFFFF}\textbf{D}}      & Mantenimiento      & \multicolumn{1}{c|}{Regular}                                      & \multicolumn{1}{c|}{Regular}                                   & \multicolumn{1}{c|}{Regular}                                                                                   & \multicolumn{1}{c|}{Bueno}                                                                                      & \cellcolor[HTML]{FFFFFF}Regular                                                           \\ \hline
	 		\multicolumn{1}{|c|}{\cellcolor[HTML]{FFFFFF}\textbf{E}}      & Montaje            & \multicolumn{1}{c|}{Malo}                                         & \multicolumn{1}{c|}{Regular}                                   & \multicolumn{1}{c|}{Malo}                                                                                      & \multicolumn{1}{c|}{Bueno}                                                                                      & \cellcolor[HTML]{FFFFFF}Regular                                                           \\ \hline
	 		\multicolumn{1}{|c|}{\cellcolor[HTML]{FFFFFF}\textbf{F}}      & Durabilidad        & \multicolumn{1}{c|}{Excelente}                                    & \multicolumn{1}{c|}{Regular}                                   & \multicolumn{1}{c|}{\cellcolor[HTML]{FFFFFF}Regular}                                                           & \multicolumn{1}{c|}{Malo}                                                                                       & Excelente                                                                                 \\ \hline
	 	\end{tabular}
	 \end{table}
	 
	 A continuación, la tabla \ref{mb_uniones} presenta la matriz binaria que permitió jerarquizar los criterios mencionados, asignando mayor peso a aquellos considerados más relevantes como el montaje y la facilidad de rotación.
	 \begin{table}[h!]
	 	\centering
	 	\caption{Matriz binaria para las uniones de la estructura (unión superior).\label{mb_uniones}}
	 	\begin{tabular}{cccccccc|c|c}
	 		\hline
	 		\rowcolor[HTML]{FFFFFF} 
	 		\multicolumn{2}{|c|}{\cellcolor[HTML]{FFFFFF}\textbf{Criterios}}                                                       & \multicolumn{1}{c|}{\cellcolor[HTML]{FFFFFF}\textbf{A}} & \multicolumn{1}{c|}{\cellcolor[HTML]{FFFFFF}\textbf{B}} & \multicolumn{1}{c|}{\cellcolor[HTML]{FFFFFF}\textbf{C}} & \multicolumn{1}{c|}{\cellcolor[HTML]{FFFFFF}\textbf{D}} & \multicolumn{1}{c|}{\cellcolor[HTML]{FFFFFF}\textbf{E}} & \textbf{F}               & \textbf{Total}          & \multicolumn{1}{c|}{\cellcolor[HTML]{FFFFFF}\textbf{Rango}} \\ \hline
	 		\multicolumn{1}{|c|}{\cellcolor[HTML]{FFFFFF}\textbf{A}} & \multicolumn{1}{c|}{\cellcolor[HTML]{FFFFFF}Facil rotación} & \multicolumn{1}{c|}{\cellcolor[HTML]{C0C0C0}}           & \multicolumn{1}{c|}{1}                                  & \multicolumn{1}{c|}{1}                                  & \multicolumn{1}{c|}{1}                                  & \multicolumn{1}{c|}{0}                                  & 1                        & 4                       & \multicolumn{1}{c|}{2}                                      \\ \hline
	 		\multicolumn{1}{|c|}{\cellcolor[HTML]{FFFFFF}\textbf{B}} & \multicolumn{1}{c|}{\cellcolor[HTML]{FFFFFF}Estabilidad}    & \multicolumn{1}{c|}{0}                                  & \multicolumn{1}{c|}{\cellcolor[HTML]{C0C0C0}}           & \multicolumn{1}{c|}{1}                                  & \multicolumn{1}{c|}{1}                                  & \multicolumn{1}{c|}{0}                                  & 0                        & 2                       & \multicolumn{1}{c|}{4}                                      \\ \hline
	 		\multicolumn{1}{|c|}{\cellcolor[HTML]{FFFFFF}\textbf{C}} & \multicolumn{1}{c|}{\cellcolor[HTML]{FFFFFF}Ruido}          & \multicolumn{1}{c|}{0}                                  & \multicolumn{1}{c|}{0}                                  & \multicolumn{1}{c|}{\cellcolor[HTML]{C0C0C0}}           & \multicolumn{1}{c|}{1}                                  & \multicolumn{1}{c|}{0}                                  & 0                        & 1                       & \multicolumn{1}{c|}{5}                                      \\ \hline
	 		\multicolumn{1}{|c|}{\cellcolor[HTML]{FFFFFF}\textbf{D}} & \multicolumn{1}{c|}{\cellcolor[HTML]{FFFFFF}Mantenimiento}  & \multicolumn{1}{c|}{0}                                  & \multicolumn{1}{c|}{0}                                  & \multicolumn{1}{c|}{0}                                  & \multicolumn{1}{c|}{\cellcolor[HTML]{C0C0C0}}           & \multicolumn{1}{c|}{0}                                  & 0                        & 0                       & \multicolumn{1}{c|}{6}                                      \\ \hline
	 		\multicolumn{1}{|c|}{\cellcolor[HTML]{FFFFFF}\textbf{E}} & \multicolumn{1}{c|}{\cellcolor[HTML]{FFFFFF}Montaje}        & \multicolumn{1}{c|}{1}                                  & \multicolumn{1}{c|}{1}                                  & \multicolumn{1}{c|}{1}                                  & \multicolumn{1}{c|}{1}                                  & \multicolumn{1}{c|}{\cellcolor[HTML]{C0C0C0}}           & 1                        & 5                       & \multicolumn{1}{c|}{1}                                      \\ \hline
	 		\multicolumn{1}{|c|}{\cellcolor[HTML]{FFFFFF}\textbf{F}} & \multicolumn{1}{c|}{\cellcolor[HTML]{FFFFFF}Durabilidad}    & \multicolumn{1}{c|}{0}                                  & \multicolumn{1}{c|}{1}                                  & \multicolumn{1}{c|}{1}                                  & \multicolumn{1}{c|}{1}                                  & \multicolumn{1}{c|}{0}                                  & \cellcolor[HTML]{C0C0C0} & 3                       & \multicolumn{1}{c|}{3}                                      \\ \hline
	 		\multicolumn{1}{l}{}                                     & \multicolumn{1}{l}{}                                        &                                                         & \multicolumn{1}{l}{}                                    & \multicolumn{1}{l}{}                                    & \multicolumn{1}{l}{}                                    & \multicolumn{1}{l}{}                                    & \multicolumn{1}{l|}{}    & \multicolumn{1}{l|}{15} & \multicolumn{1}{l}{}                                        \\ \cline{9-9}
	 	\end{tabular}
	 \end{table}
	 
	 Finalmente, la tabla \ref{mp_uniones} contiene la matriz de ponderación, donde se aprecia que la opción más favorable fue el uso de juntas Clevis, al obtener la puntuación más alta combinando todos los criterios ponderados.
	 \begin{table}[h!]
	 	\centering
	 	\caption{Matriz de ponderación para las uniones de la estructura (unión superior).\label{mp_uniones}}
	 	\scalebox{0.9}{\begin{tabular}{cccc|c|c|c|c|c|}
	 			\hline
	 			\rowcolor[HTML]{FFFFFF} 
	 			\multicolumn{2}{|c|}{\cellcolor[HTML]{FFFFFF}\textbf{Criterios}}                                                       & \multicolumn{1}{c|}{\cellcolor[HTML]{FFFFFF}Rango} & Ponderación                   & \textbf{\begin{tabular}[c]{@{}c@{}}Roda-\\ mientos\end{tabular}} & \textbf{Bisagras} & \textbf{\begin{tabular}[c]{@{}c@{}}Juntas\\ Cardán\end{tabular}} & \textbf{\begin{tabular}[c]{@{}c@{}}Aplanado \\ Pernos\end{tabular}} & \textbf{\begin{tabular}[c]{@{}c@{}}Juntas\\ Clevis\end{tabular}} \\ \hline
	 			\multicolumn{1}{|c|}{\cellcolor[HTML]{FFFFFF}\textbf{E}} & \multicolumn{1}{c|}{\cellcolor[HTML]{FFFFFF}Montaje}        & \multicolumn{1}{c|}{\cellcolor[HTML]{FFFFFF}5}     & \cellcolor[HTML]{FFFFFF}0.333 & 2                                                                & 3                 & 2                                                                & 4                                                                   & 3                                                                \\ \hline
	 			\multicolumn{1}{|c|}{\cellcolor[HTML]{FFFFFF}\textbf{A}} & \multicolumn{1}{c|}{\cellcolor[HTML]{FFFFFF}Fácil rotación} & \multicolumn{1}{c|}{\cellcolor[HTML]{FFFFFF}4}     & \cellcolor[HTML]{FFFFFF}0.267 & 5                                                                & 3                 & 5                                                                & 3                                                                   & 4                                                                \\ \hline
	 			\multicolumn{1}{|c|}{\cellcolor[HTML]{FFFFFF}\textbf{F}} & \multicolumn{1}{c|}{\cellcolor[HTML]{FFFFFF}Durabilidad}    & \multicolumn{1}{c|}{\cellcolor[HTML]{FFFFFF}3}     & \cellcolor[HTML]{FFFFFF}0.200 & 5                                                                & 3                 & 3                                                                & 2                                                                   & 5                                                                \\ \hline
	 			\multicolumn{1}{|c|}{\cellcolor[HTML]{FFFFFF}\textbf{B}} & \multicolumn{1}{c|}{\cellcolor[HTML]{FFFFFF}Estabilidad}    & \multicolumn{1}{c|}{\cellcolor[HTML]{FFFFFF}2}     & \cellcolor[HTML]{FFFFFF}0.133 & 5                                                                & 3                 & 2                                                                & 3                                                                   & 5                                                                \\ \hline
	 			\multicolumn{1}{|c|}{\cellcolor[HTML]{FFFFFF}\textbf{C}} & \multicolumn{1}{c|}{\cellcolor[HTML]{FFFFFF}Ruido}          & \multicolumn{1}{c|}{\cellcolor[HTML]{FFFFFF}1}     & \cellcolor[HTML]{FFFFFF}0.067 & 4                                                                & 3                 & 3                                                                & 3                                                                   & 4                                                                \\ \hline
	 			\multicolumn{1}{|c|}{\cellcolor[HTML]{FFFFFF}\textbf{D}} & \multicolumn{1}{c|}{\cellcolor[HTML]{FFFFFF}Mantenimiento}  & \multicolumn{1}{c|}{\cellcolor[HTML]{FFFFFF}0}     & \cellcolor[HTML]{FFFFFF}0.000 & 3                                                                & 3                 & 3                                                                & 4                                                                   & 3                                                                \\ \hline
	 			\multicolumn{1}{l}{}                                     & \multicolumn{1}{l}{}                                        & \multicolumn{1}{l}{}                               & \multicolumn{1}{l|}{}         & 3.93                                                             & 3.00              & 3.07                                                             & 3.13                                                                & \cellcolor[HTML]{FFFFFF}4.00                                     \\ \cline{5-9} 
	 	\end{tabular}}
	 \end{table}
	 
	 De forma similar, se realizó un análisis para las unión del eje de rotación más próximo al coxofemoral, utilizando los mismos cinco tipos de uniones y criterios. En la tabla \ref{ms_uniones2} se muestra la evaluación subjetiva de estas alternativas.\clearpage
	 \begin{table}[h!]
	 	\centering
	 	\caption{Matriz de selección subjetiva para las uniones de la estructura (eje de giro).\label{ms_uniones2}}
	 	\begin{tabular}{|
	 			>{\columncolor[HTML]{FFFFFF}}c 
	 			>{\columncolor[HTML]{FFFFFF}}c |ccccc|}
	 		\hline
	 		\multicolumn{2}{|c|}{\cellcolor[HTML]{FFFFFF}}                                     & \multicolumn{5}{c|}{\cellcolor[HTML]{C0C0C0}\textbf{Conceptos}}                                                                                                                                                                                                                                                                                                                                                                                                 \\ \cline{3-7} 
	 		\multicolumn{2}{|c|}{\multirow{-2}{*}{\cellcolor[HTML]{FFFFFF}\textbf{Criterios}}} & \multicolumn{1}{c|}{\cellcolor[HTML]{FFFFFF}\textbf{Rodamientos}} & \multicolumn{1}{c|}{\cellcolor[HTML]{FFFFFF}\textbf{Bisagras}} & \multicolumn{1}{c|}{\cellcolor[HTML]{FFFFFF}\textbf{\begin{tabular}[c]{@{}c@{}}Juntas\\ Cardán\end{tabular}}} & \multicolumn{1}{c|}{\cellcolor[HTML]{FFFFFF}\textbf{\begin{tabular}[c]{@{}c@{}}Aplanado\\ Pernos\end{tabular}}} & \cellcolor[HTML]{FFFFFF}\textbf{\begin{tabular}[c]{@{}c@{}}Juntas\\ Clevis\end{tabular}} \\ \hline
	 		\multicolumn{1}{|c|}{\cellcolor[HTML]{FFFFFF}\textbf{A}}      & Facil rotación     & \multicolumn{1}{c|}{Excelente}                                    & \multicolumn{1}{c|}{Regular}                                   & \multicolumn{1}{c|}{Excelente}                                                                                & \multicolumn{1}{c|}{Regular}                                                                                    & Bueno                                                                                    \\ \hline
	 		\multicolumn{1}{|c|}{\cellcolor[HTML]{FFFFFF}\textbf{B}}      & Estabilidad        & \multicolumn{1}{c|}{Excelente}                                    & \multicolumn{1}{c|}{Regular}                                   & \multicolumn{1}{c|}{Malo}                                                                                     & \multicolumn{1}{c|}{Regular}                                                                                    & Excelente                                                                                \\ \hline
	 		\multicolumn{1}{|c|}{\cellcolor[HTML]{FFFFFF}\textbf{C}}      & Ruido              & \multicolumn{1}{c|}{Bueno}                                        & \multicolumn{1}{c|}{Regular}                                   & \multicolumn{1}{c|}{Bueno}                                                                                    & \multicolumn{1}{c|}{Regular}                                                                                    & Bueno                                                                                    \\ \hline
	 		\multicolumn{1}{|c|}{\cellcolor[HTML]{FFFFFF}\textbf{D}}      & Mantenimiento      & \multicolumn{1}{c|}{Regular}                                      & \multicolumn{1}{c|}{Regular}                                   & \multicolumn{1}{c|}{Regular}                                                                                  & \multicolumn{1}{c|}{Bueno}                                                                                      & Excelente                                                                                \\ \hline
	 		\multicolumn{1}{|c|}{\cellcolor[HTML]{FFFFFF}\textbf{E}}      & Montaje            & \multicolumn{1}{c|}{Malo}                                         & \multicolumn{1}{c|}{Regular}                                   & \multicolumn{1}{c|}{Regular}                                                                                  & \multicolumn{1}{c|}{Bueno}                                                                                      & Excelente                                                                                \\ \hline
	 		\multicolumn{1}{|c|}{\cellcolor[HTML]{FFFFFF}\textbf{F}}      & Durabilidad        & \multicolumn{1}{c|}{Excelente}                                    & \multicolumn{1}{c|}{Regular}                                   & \multicolumn{1}{c|}{Bueno}                                                                                    & \multicolumn{1}{c|}{Bueno}                                                                                      & Excelente                                                                                \\ \hline
	 	\end{tabular}
	 \end{table}
	
	 La tabla \ref{mb_uniones2} contiene la comparación binaria que establece el orden de importancia entre los criterios, destacando la facilidad de rotación y estabilidad como los más influyentes.
	 \begin{table}[h!]
	 	\centering
	 	\caption{Matriz binaria para las uniones de la estructura (eje de giro).\label{mb_uniones2}}
	 	\begin{tabular}{ccllllll|l|l}
	 		\hline
	 		\rowcolor[HTML]{FFFFFF} 
	 		\multicolumn{2}{|c|}{\cellcolor[HTML]{FFFFFF}\textbf{Criterios}}                                                       & \multicolumn{1}{c|}{\cellcolor[HTML]{FFFFFF}\textbf{A}} & \multicolumn{1}{c|}{\cellcolor[HTML]{FFFFFF}\textbf{B}} & \multicolumn{1}{c|}{\cellcolor[HTML]{FFFFFF}\textbf{C}} & \multicolumn{1}{c|}{\cellcolor[HTML]{FFFFFF}\textbf{D}} & \multicolumn{1}{c|}{\cellcolor[HTML]{FFFFFF}\textbf{E}} & \multicolumn{1}{c|}{\cellcolor[HTML]{FFFFFF}\textbf{F}} & \multicolumn{1}{c|}{\cellcolor[HTML]{FFFFFF}\textbf{Total}} & \multicolumn{1}{c|}{\cellcolor[HTML]{FFFFFF}\textbf{Rango}} \\ \hline
	 		\multicolumn{1}{|c|}{\cellcolor[HTML]{FFFFFF}\textbf{A}} & \multicolumn{1}{c|}{\cellcolor[HTML]{FFFFFF}Fácil rotación} & \multicolumn{1}{c|}{\cellcolor[HTML]{C0C0C0}}           & \multicolumn{1}{l|}{1}                                  & \multicolumn{1}{l|}{1}                                  & \multicolumn{1}{l|}{1}                                  & \multicolumn{1}{l|}{1}                                  & 1                                                       & 5                                                           & \multicolumn{1}{l|}{1}                                      \\ \hline
	 		\multicolumn{1}{|c|}{\cellcolor[HTML]{FFFFFF}\textbf{B}} & \multicolumn{1}{c|}{\cellcolor[HTML]{FFFFFF}Estabilidad}    & \multicolumn{1}{l|}{0}                                  & \multicolumn{1}{c|}{\cellcolor[HTML]{C0C0C0}}           & \multicolumn{1}{l|}{1}                                  & \multicolumn{1}{l|}{1}                                  & \multicolumn{1}{l|}{1}                                  & 1                                                       & 4                                                           & \multicolumn{1}{l|}{2}                                      \\ \hline
	 		\multicolumn{1}{|c|}{\cellcolor[HTML]{FFFFFF}\textbf{C}} & \multicolumn{1}{c|}{\cellcolor[HTML]{FFFFFF}Ruido}          & \multicolumn{1}{l|}{0}                                  & \multicolumn{1}{l|}{0}                                  & \multicolumn{1}{c|}{\cellcolor[HTML]{C0C0C0}}           & \multicolumn{1}{l|}{1}                                  & \multicolumn{1}{l|}{0}                                  & 0                                                       & 1                                                           & \multicolumn{1}{l|}{5}                                      \\ \hline
	 		\multicolumn{1}{|c|}{\cellcolor[HTML]{FFFFFF}\textbf{D}} & \multicolumn{1}{c|}{\cellcolor[HTML]{FFFFFF}Mantenimiento}  & \multicolumn{1}{l|}{0}                                  & \multicolumn{1}{l|}{0}                                  & \multicolumn{1}{l|}{0}                                  & \multicolumn{1}{c|}{\cellcolor[HTML]{C0C0C0}}           & \multicolumn{1}{l|}{0}                                  & 0                                                       & 0                                                           & \multicolumn{1}{l|}{6}                                      \\ \hline
	 		\multicolumn{1}{|c|}{\cellcolor[HTML]{FFFFFF}\textbf{E}} & \multicolumn{1}{c|}{\cellcolor[HTML]{FFFFFF}Montaje}        & \multicolumn{1}{l|}{0}                                  & \multicolumn{1}{l|}{0}                                  & \multicolumn{1}{l|}{1}                                  & \multicolumn{1}{l|}{1}                                  & \multicolumn{1}{c|}{\cellcolor[HTML]{C0C0C0}}           & 0                                                       & 2                                                           & \multicolumn{1}{l|}{4}                                      \\ \hline
	 		\multicolumn{1}{|c|}{\cellcolor[HTML]{FFFFFF}\textbf{F}} & \multicolumn{1}{c|}{\cellcolor[HTML]{FFFFFF}Durabilidad}    & \multicolumn{1}{l|}{0}                                  & \multicolumn{1}{l|}{0}                                  & \multicolumn{1}{l|}{1}                                  & \multicolumn{1}{l|}{1}                                  & \multicolumn{1}{l|}{1}                                  & \multicolumn{1}{c|}{\cellcolor[HTML]{C0C0C0}}           & 3                                                           & \multicolumn{1}{l|}{3}                                      \\ \hline
	 		\multicolumn{1}{l}{}                                     & \multicolumn{1}{l}{}                                        & \multicolumn{1}{c}{}                                    &                                                         &                                                         &                                                         &                                                         &                                                         & 15                                                          &                                                             \\ \cline{9-9}
	 	\end{tabular}
	 \end{table}
	 
	 Por último, en la tabla \ref{mp_uniones2}, mediante la ponderación de los criterios, se concluye que nuevamente las juntas Clevis representan la mejor alternativa para el eje de giro, con una puntuación final superior al resto de conceptos analizados.
	 \begin{table}[h!]
	 	\centering
	 	\caption{Matriz de ponderación para las uniones de la estructura (eje de giro).\label{mp_uniones2}}
	 	\scalebox{0.9}{\begin{tabular}{ccll|c|c|c|c|c|}
	 			\hline
	 			\rowcolor[HTML]{FFFFFF} 
	 			\multicolumn{2}{|c|}{\cellcolor[HTML]{FFFFFF}\textbf{Criterios}}                                                       & \multicolumn{1}{c|}{\cellcolor[HTML]{FFFFFF}Total} & \multicolumn{1}{c|}{\cellcolor[HTML]{FFFFFF}Ponderación} & \textbf{\begin{tabular}[c]{@{}c@{}}Roda-\\ mientos\end{tabular}} & \textbf{Bisagras}      & \textbf{\begin{tabular}[c]{@{}c@{}}Juntas\\ Cardán\end{tabular}} & \textbf{\begin{tabular}[c]{@{}c@{}}Aplanado\\ Pernos\end{tabular}} & \textbf{\begin{tabular}[c]{@{}c@{}}Juntas\\ Clevis\end{tabular}} \\ \hline
	 			\multicolumn{1}{|c|}{\cellcolor[HTML]{FFFFFF}\textbf{A}} & \multicolumn{1}{c|}{\cellcolor[HTML]{FFFFFF}Fácil rotación} & \multicolumn{1}{l|}{5}                             & 0.333                                                    & 5                                                                & 3                      & 5                                                                & 3                                                                  & 4                                                                \\ \hline
	 			\multicolumn{1}{|c|}{\cellcolor[HTML]{FFFFFF}\textbf{B}} & \multicolumn{1}{c|}{\cellcolor[HTML]{FFFFFF}Estabilidad}    & \multicolumn{1}{l|}{4}                             & 0.267                                                    & 5                                                                & 3                      & 2                                                                & 3                                                                  & 5                                                                \\ \hline
	 			\multicolumn{1}{|c|}{\cellcolor[HTML]{FFFFFF}\textbf{F}} & \multicolumn{1}{c|}{\cellcolor[HTML]{FFFFFF}Durabilidad}    & \multicolumn{1}{l|}{3}                             & 0.200                                                    & 5                                                                & 3                      & 4                                                                & 4                                                                  & 5                                                                \\ \hline
	 			\multicolumn{1}{|c|}{\cellcolor[HTML]{FFFFFF}\textbf{E}} & \multicolumn{1}{c|}{\cellcolor[HTML]{FFFFFF}Montaje}        & \multicolumn{1}{l|}{2}                             & 0.133                                                    & 2                                                                & 3                      & 3                                                                & 4                                                                  & 5                                                                \\ \hline
	 			\multicolumn{1}{|c|}{\cellcolor[HTML]{FFFFFF}\textbf{C}} & \multicolumn{1}{c|}{\cellcolor[HTML]{FFFFFF}Ruido}          & \multicolumn{1}{l|}{1}                             & 0.067                                                    & 4                                                                & 3                      & 4                                                                & 3                                                                  & 4                                                                \\ \hline
	 			\multicolumn{1}{|c|}{\cellcolor[HTML]{FFFFFF}\textbf{D}} & \multicolumn{1}{c|}{\cellcolor[HTML]{FFFFFF}Mantenimiento}  & \multicolumn{1}{l|}{0}                             & 0.000                                                    & 3                                                                & 3                      & 3                                                                & 4                                                                  & 5                                                                \\ \hline
	 			\multicolumn{1}{l}{}                                     & \multicolumn{1}{l}{}                                        &                                                    &                                                          & \multicolumn{1}{l|}{4.53}                                        & \multicolumn{1}{l|}{3} & \multicolumn{1}{l|}{3.66}                                        & \multicolumn{1}{l|}{3.33}                                          & \multicolumn{1}{l|}{\cellcolor[HTML]{FFFFFF}4.6}                 \\ \cline{5-9} 
	 	\end{tabular}}
	 \end{table}

\subsubsection*{Selección de material para mecanismos de flexión-extensión y abducción-aducción}
Se definió que el perfil adecuado para el diseño del mecanismo de abducción y aducción es el tubo cuadrado, asimismo los tipos de uniones seleccionados son la juntas clevis, pero ahora es importante seleccionar el material para la estructura porque se busca que no sea muy pesada y que sea capaz de soportar el peso de la pierna derecha del paciente principalmente, por lo que buscando diferentes alternativas encontramos que algunos materiales usados en los dispositivos médicos son el aluminio, acero inoxidable y titanio. El titanio es un material biocompatible por excelencia y puede estar en contacto con la piel del ser humano sin ningún problema, sin embargo, es muy costoso y no tan accesible, además no presenta buena maquinabilidad, de tal manera que este material queda descartado para el diseño de la ortesis, por otro lado, el aluminio y el acero inoxidable son materiales muy accesibles y presentan buena resistencia a la tensión y flexión, la maquinabilidad de estos materiales es buena y se pueden cortar, unir y ensamblar de manera más sencilla que el titanio, el costo del aluminio y el acero inoxidable es menor al costo del titanio y la biocompatibilidad de estos materiales es menor a la biocompatibilidad del titanio, sin embargo, en este caso no es muy necesario que el material para la construcción de la ortesis sea bio compatible por excelencia, debido a que la pierna del paciente no va a estar en contacto directo con el material y el tiempo de uso de la ortesis no es prolongado, por lo que, se decidió seleccionar el aluminio y el acero inoxidable para diseñar los mecanismos de la ortesis.

El aluminio se usará para el diseño del mecanismo de flexión y extensión, debido a que es menos pesado que el acero inoxidable y su resistencia a la tensión es buena, aunque el acero inoxidable presenta una mejor resistencia a la tensión por las características del mecanismo de flexión y extensión el aluminio es el más adecuado porque la carga que debe soportar no es tan elevada, por lo tanto, se seleccionó específicamente el aluminio 6061-T6.

Para el mecanismo de abducción y aducción se usará el acero inoxidable porque en este mecanismo necesitamos que el material sea más resistente estructuralmente, ya que sobre este mecanismo se va a montar el mecanismo de flexión y extensión. Para el mecanismo de abducción y aducción el peso no es muy importante y en el tema del costo es más barato el aluminio, pero lo que más nos interesa es que sea muy resistente, por eso se seleccionó el acero inoxidable, específicamente el AISI 304.

\subsubsection*{Selección de material para soportes de la pierna derecha}
Los soportes en donde va a reposar la pierna del paciente deben ser rígidos y ligeros, y al igual que los mecanismos de flexión-extensión y abducción-aducción no se necesita que el material sea biocompatible por excelencia, sin embargo, en este caso la pierna del paciente puede ser que sí entre en contacto directo con el material, por lo tanto, al buscar alternativas encontramos que algunos materiales usados para el diseño de férulas o soportes médicos son los plásticos, aluminio y yeso. 

El aluminio es una buena opción para el diseño de los soportes de la pierna, sin embargo, es más pesado que un plástico o que el yeso, la maquinabilidad del aluminio es menor en comparación con los otros dos materiales y para los soportes no se requiere de un material tan resistente como el aluminio por eso se descartó, ya que los soportes solo brindarán reposo y todo el peso de la pierna lo va a soportar la estructura de la ortesis.

El yeso es un material fácil de trabajar y se puede moldear de una manera sencilla y rápida, pero es muy frágil y no es duradero, además es poroso y podría causar alguna molestia para el paciente por lo que se descartó, además es un material de uso temporal y no es reutilizable.

La mejor opción para el diseño de los soportes de la pierna son los plásticos porque presentan buena rigidez, es cierto que el aluminio es más rígido y por eso se usará en el diseño de la estructura del mecanismo de flexión y extensión, sim embargo, en este caso no requerimos de un material que soporte cargas elevadas, de igual manera los plásticos son menos pesados que el aluminio y como los soportes se colocaran sobre la estructura de la ortesis el peso de estos debe ser mínimo, en este caso se seleccionó el PVC Rígido para el diseño de los soportes ya que tiene aplicaciones médicas y cumple con las características de rigidez y peso mencionadas.

\subsubsection*{Selección de rieles para desplazamiento}
Para que los movimientos de flexión y extensión se realicen de manera efectiva, el actuador lineal debe ser capaz de desplazarse a lo largo de la estructura con facilidad y estabilidad, por estas razones se decidió usar una base con rieles sobre la cual se monta un objeto móvil que contiene rodamientos lineales, estos permiten que el desplazamiento a lo largo de los rieles tenga poca fricción, y al estar apoyado sobre rieles el movimiento será estable y no presentará alguna desviación o vibración. La base se puede observar en la Fig. \ref{fig:rieles} y en el lado derecho se pueden acoplar motores Nema con diferentes características, esto permite una flexibilidad de uso de esta base, por eso se seleccionó para montarla sobre la estructura del mecanismo de flexión y extensión. El motor específico que se usará se va a detallar más adelante en el sistema de movimiento, módulo 8.  
 \begin{figure}[h!]
 	\centering
 	\includegraphics [trim = 0 0cm 0 0, clip, width=0.5\textwidth]{figure/s1_estructural/Rieles.png}
 	\caption[Base con rieles donde se montará el actuador lineal]{Base con rieles donde se montará el actuador lineal\label{fig:rieles}}
 \end{figure}

\subsubsection{Mecanismo de flexión-extensión}
Una vez seleccionado el perfil estructural, los tipos de uniones, el actuador lineal y los materiales con los que se van a diseñar los mecanismos y la estructura de la cama se comenzó a diseñar el mecanismo de flexión y extensión de la ortesis, contemplando las medidas propuestas en los requerimientos para el fémur y la tibia principalmente, debido a que este mecanismo deberá ajustarse al tamaño de pierna de cada persona. En la Fig. \ref{fig:Pina_Figura01} se observa el diseño del mecanismo con los materiales seleccionados, y se incluyeron chapas metálicas y soportes en diferentes puntos con la finalidad de que sobre ellos repose la pierna del paciente.
 \begin{figure}[h!]
 	\centering
 	\includegraphics [trim = 0 0cm 0 0, clip, width=0.6\textwidth]{figure/s1_estructural/Pina_Figura01.png}
 	\caption[Ensamble del mecanismo terminado de flexión y extensión]{Ensamble del mecanismo terminado de flexión y extensión\label{fig:Pina_Figura01}}
 \end{figure}

Para verificar que el mecanismo diseñado sea resistente y factible de implementar con los materiales correspondientes, se realizó un análisis estático en SolidWorks en donde se aplicó una carga sobre los soportes simulando el peso de la pierna del paciente, en este caso el mecanismo debe ser capaz de soportar un peso máximo de 20Kg, sin embargo, para tener un margen de seguridad se aumentó el peso a 23Kg en la simulación para observar si el mecanismo es seguro. En la Fig. \ref{fig:Pina_Figura02} se muestran las caras sobre las cuales se aplicaron las fuerzas y su magnitud respectivamente, la primera corresponde a la parte donde reposa el muslo, la segunda corresponde a la pantorrilla y la tercera al pie, las fuerzas aplicadas tienen esas magnitudes porque el peso del muslo es de 14 Kg aproximadamente, el de la pantorrilla es de 4 Kg, y el pie pesa 2 Kg \cite{37}, y al multiplicar esas masas por la gravedad (9.81$m/s^2$) obtenemos la fuerza equivalente en newtons, pero en este caso decidimos utilizar una magnitud de la gravedad de 10$m/s^2$ y agregamos 1 Kg extra a cada masa para aumentar el margen de seguridad.
 
 \begin{figure}[h!]
 	\centering
 	\includegraphics [trim = 0 0cm 0 0, clip, width=0.6\textwidth]{figure/s1_estructural/Pina_Figura02.png}
 	\caption[Caras del ensamble donde se simula una fuerza proporcional a una pierna humana.]{Caras del ensamble donde se simula una fuerza proporcional a una pierna humana.\label{fig:Pina_Figura02}}
 \end{figure}

 \subsubsection*{Resultado de tensiones}
Se presenta una tensión máxima de 24MPa aproximadamente, y los límites elásticos de los materiales utilizados se encuentran arriba de los 50MPa lo que demuestra que este mecanismo es capaz de soportar la carga aplicada y no va a ceder fácilmente [Fig. \ref{fig:Pina_Figura03}].
 \begin{figure}[h!]
 	\centering
 	\includegraphics [trim = 0 0cm 0 0, clip, width=0.6\textwidth]{figure/s1_estructural/Pina_Figura03.png}
 	\caption[Resultados de un estudio de tensiones en SolidWorks de acuerdo a las fuerzas de la Fig. \ref{fig:Pina_Figura02}.]{Resultados de un estudio de tensiones en SolidWorks de acuerdo a las fuerzas de la Fig. \ref{fig:Pina_Figura02}.\label{fig:Pina_Figura03}}
 \end{figure}
 
 \subsubsection*{Resultado de desplazamientos}
 El desplazamiento máximo del mecanismo al aplicarle la carga ocurre en la zona donde reposa el muslo del paciente y podemos observar que es de 0.213mm, el cual representa un valor muy pequeño y no representa algún peligro para el paciente, además este desplazamiento no interfiere en la geometría de la estructura y no afecta a los movimientos de flexión y extensión de la pierna del paciente [Fig. \ref{fig:Pina_Figura04}].
 \begin{figure}[h!]
 	\centering
 	\includegraphics [trim = 0 0cm 0 0, clip, width=0.6\textwidth]{figure/s1_estructural/Pina_Figura04.png}
 	\caption[Resultados de un estudio de desplazamientos en SolidWorks de acuerdo a las fuerzas de la Fig. \ref{fig:Pina_Figura02}.]{Resultados de un estudio de desplazamientos en SolidWorks de acuerdo a las fuerzas de la Fig. \ref{fig:Pina_Figura02}.\label{fig:Pina_Figura04}}
 \end{figure}

 \subsubsection*{Factor de seguridad}
El mecanismo tiene un factor de seguridad mínimo de 11 como se observa en la Fig. \ref{fig:Pina_Figura05}, esto nos permite saber que la estructura del mecanismo es resistente y puede soportar la carga aplicada sin inconvenientes, por lo que será factible su implementación con los materiales y perfiles seleccionados.
 \begin{figure}[h!]
 	\centering
 	\includegraphics [trim = 0 0cm 0 0, clip, width=0.6\textwidth]{figure/s1_estructural/Pina_Figura05.png}
 	\caption[Resultados de un estudio para conocer el Factor de Seguridad en SolidWorks de acuerdo a las fuerzas de la Fig. \ref{fig:Pina_Figura02}.]{Resultados de un estudio para conocer el Factor de Seguridad en SolidWorks de acuerdo a las fuerzas de la Fig. \ref{fig:Pina_Figura02}.\label{fig:Pina_Figura05}}
 \end{figure}

 \subsubsection{Mecanismo de abducción-aducción}
El mecanismo de abducción-aducción va a soportar toda la estructura del mecanismo de flexión-extensión junto con el peso de la pierna del paciente, por lo tanto, debe tener un diseño estructural muy resistente y adecuado para brindar seguridad. En las Figs. \ref{fig:Pina_Figura06} y \ref{fig:Pina_Figura07} podemos observar el diseño del mecanismo, el cual consta de un tubo apoyado sobre un rodamiento de carga axial y alineado con 2 chumaceras colocadas en los extremos, en la parte superior el tubo se acopla al mecanismo de flexión-extensión mediante un soporte y una placa perforada. El diseño de este mecanismo se enfoca en que el actuador rotativo no reciba una carga axial de forma directa y que sea capaz de realizar los movimientos de abducción y aducción fácilmente, por esa razón se decidió utilizar en la parte inferior del tubo un rodamiento de carga axial que está diseñado para soportar cargas axiales elevadas y los podemos encontrar de diferentes tamaños, además esto permitirá que el tubo gire libremente sobre su eje y con poca fricción. Las chumaceras se utilizan para poder alinear el tubo con el eje del motor y que no existan vibraciones durante el giro, debido a que las chumaceras evitan que el tubo se desplace hacia los lados. Al tener las chumaceras y el rodamiento axial nos aseguramos de que la tarea del motor sea únicamente proporcionar el giro de abducción y aducción de la pierna, por lo tanto, no necesitamos un motor robusto y resistente ya que no soportará alguna carga axial directamente.
 \begin{figure}[h!]
 	\centering
 	\includegraphics [trim = 0 0cm 0 0, clip, width=0.6\textwidth]{figure/s1_estructural/Pina_Figura06.png}
 	\caption[Ensamble del mecanismo terminado de abducción y aducción]{Ensamble del mecanismo terminado de abducción y aducción\label{fig:Pina_Figura06}}
 \end{figure}
 \begin{figure}[h!]
 	\centering
 	\includegraphics [trim = 0 0cm 0 0, clip, width=0.78\textwidth]{figure/s1_estructural/Pina_Figura07.png}
 	\caption[Mecanismo de abducción y aducción acoplado al de flexión y extensión]{Mecanismo de abducción y aducción acoplado al de flexión y extensión\label{fig:Pina_Figura07}}
 \end{figure}

 Se realizó un análisis estático en SolidWorks para verificar que el mecanismo diseñado sea capaz de soportar el peso de la estructura del mecanismo de flexión y extensión y el peso de la pierna de la persona, para ello se aplicó una carga total de 301N sobre 3 puntos diferentes como se observa en la Fig. \ref{fig:Pina_Figura08}. Del lado izquierdo sobre las uniones en donde se colocan los tubos que soportan el muslo de la pierna se aplicaron dos cargas con un valor de 93N cada una, esto es porque se contempló un peso máximo de la pierna de 230N y el peso de los componentes restantes del mecanismo de flexión y extensión es de 70N aproximadamente que se repartieron en 35N por lado, anteriormente se definió que el peso del muslo es de 150N y al sumarlo con los 35N de la estructura nos da un resultado de 185N que a su vez se reparten en 92.5N para cada unión, pero se decidió fijar la carga en 93N, por otro lado, el peso de la pantorrilla más el peso del pie suman 80N y al añadirle los 35N de la estructura nos da un total de 115N que se concentran en un solo punto que es la pieza móvil que se encuentra sobre el riel.
 \begin{figure}[h!]
 	\centering
 	\includegraphics [trim = 0 0cm 0 0, clip, width=0.6\textwidth]{figure/s1_estructural/Pina_Figura08.png}
 	\caption[Caras del ensamble donde se simula una fuerza proporcional a la pierna y al mecanismo de flexión y extensión.]{Caras del ensamble donde se simula una fuerza proporcional a la pierna y al mecanismo de flexión y extensión.\label{fig:Pina_Figura08}}
 \end{figure}

 Para simplificar el análisis estático en SolidWorks solo se consideraron las piezas mostradas en la Fig. \ref{fig:Pina_Figura09}, y se colocaron sujeciones de rodamiento en los puntos en donde van colocadas las chumaceras y una sujeción de rodillo deslizante que es la cara inferior del tubo porque ahí se coloca el rodamiento axial, de igual manera se agregaron tornillos de acero grado 8.8 para unir el soporte, la placa de acople y la base del mecanismo de flexión y extensión, estos tornillos se eligieron porque están diseñados para ensamblajes que requieren alta resistencia mecánica y precisión, también son accesibles y se pueden conseguir de diferentes tamaños y diámetros.
 \begin{figure}[h!]
 	\centering
 	\includegraphics [trim = 0 0cm 0 0, clip, width=0.6\textwidth]{figure/s1_estructural/Pina_Figura09.png}
 	\caption[Sujeciones del mecanismo de abducción y aducción para su análisis.]{Sujeciones del mecanismo de abducción y aducción para su análisis.\label{fig:Pina_Figura09}}
 \end{figure}

 \subsubsection*{Resultado de tensiones}
 El resultado arroja una tensión máxima de 105MPa y los materiales utilizados en este análisis contienen límites elásticos arriba de los 200MPa, por lo tanto, el mecanismo será capaz de soportar la carga aplicada y no va a presentar inconvenientes [Fig. \ref{fig:Pina_Figura10}].
 \begin{figure}[h!]
 	\centering
 	\includegraphics [trim = 0 0cm 0 0, clip, width=0.6\textwidth]{figure/s1_estructural/Pina_Figura10.png}
 	\caption[Resultados de un estudio de tensiones en SolidWorks de acuerdo a las fuerzas de la Fig. \ref{fig:Pina_Figura09}.]{Resultados de un estudio de tensiones en SolidWorks de acuerdo a las fuerzas de la Fig. \ref{fig:Pina_Figura09}.\label{fig:Pina_Figura10}}
 \end{figure}

 \subsubsection*{Resultado de desplazamientos}
 El desplazamiento máximo del mecanismo ocurre en el extremo derecho como se observa en la Fig. \ref{fig:Pina_Figura11} y tiene un valor de 1.42mm, en este caso este desplazamiento no representa algún peligro porque no interfiere en el funcionamiento del mecanismo y los movimientos de abducción y aducción se podrán realizar de manera efectiva [Fig. \ref{fig:Pina_Figura11}].
 \begin{figure}[h!]
 	\centering
 	\includegraphics [trim = 0 0cm 0 0, clip, width=0.6\textwidth]{figure/s1_estructural/Pina_Figura11.png}
 	\caption[Resultados de un estudio de desplazamientos en SolidWorks de acuerdo a las fuerzas de la Fig. \ref{fig:Pina_Figura09}.]{Resultados de un estudio de desplazamientos en SolidWorks de acuerdo a las fuerzas de la Fig. \ref{fig:Pina_Figura09}.\label{fig:Pina_Figura11}}
 \end{figure}

  \subsubsection*{Factor de seguridad}
El factor de seguridad del mecanismo es de 2, esto significa que el mecanismo puede ser capaz de soportar el doble de la carga aplicada antes de que el material falle, de esta manera aseguramos que la implementación de este mecanismo con el diseño mostrado es segura y va a resistir el peso de la pierna del paciente sin problemas [Fig. \ref{fig:Pina_Figura12}].
 \begin{figure}[h!]
 	\centering
 	\includegraphics [trim = 0 0cm 0 0, clip, width=0.6\textwidth]{figure/s1_estructural/Pina_Figura12.png}
 	\caption[Resultados de un estudio para conocer el Factor de Seguridad en SolidWorks de acuerdo a las fuerzas de la Fig. \ref{fig:Pina_Figura09}.]{Resultados de un estudio para conocer el Factor de Seguridad en SolidWorks de acuerdo a las fuerzas de la Fig. \ref{fig:Pina_Figura09}.\label{fig:Pina_Figura12}}
 \end{figure}

\subsubsection{Estructura de la cama}
Para la estructura de la cama se eligió un tubo cuadrado de PTR de 3”x3”x0.1875”, debido a su alta resistencia a la flexión y a la compresión, además el PTR es comúnmente utilizado en estructuras de muebles y en herrería gracias a que su forma facilita el corte, la soldadura y el ensamblaje, las caras planas de este tipo de tubos nos permiten acoplar o adaptar diferentes componentes con facilidad, ya sea a través de soldadura o tornillos y pernos, y en este caso necesitamos acoplarle los mecanismos de flexión y extensión, abducción y aducción, los diferentes componentes electrónicos, el botón de emergencia, etc., a la estructura, de esta manera el tubo es una opción ideal para realizar el diseño de la cama. Otro punto importante es que sobre la parte superior de la estructura se va a colocar una superficie plana que abarque todo el perímetro y con el perfil cuadrado esta tarea es más fácil de realizar.

La superficie plana que se va a colocar sobre la estructura de la cama es una tabla de triplay de 18mm de grosor, este material es muy accesible y no es costoso, además gracias a su estructura interna de capas de madera entrecruzadas es resistente a la flexión y torsión, lo que lo convierte en un material adecuado para utilizarse en la cama, por otro lado, sobre la tabla de triplay se va a colocar una colchoneta para terapia de 5cm de grosor para tener ergonomía y que el paciente este cómodo durante las sesiones de rehabilitación, de esta forma la combinación de los tubos de PTR y la tabla de triplay representa una opción viable, funcional y accesible para trabajar, sin embargo, para comprobar que el diseño de la cama sea seguro se realizó un análisis estático en SolidWorks. Para el análisis se aplicó una carga de 835N sobre la cara superior de la tabla de triplay como se observa en la Fig. \ref{fig:Pina_Figura13}, esta carga resulta del peso máximo de la persona que es de 80Kg (sin la pierna derecha) y un peso adicional de 5Kg para tener un margen de seguridad lo que nos da una carga de 833N, pero se decidió fijar en 835N.
 \begin{figure}[h!]
 	\centering
 	\includegraphics [trim = 0 0cm 0 0, clip, width=0.6\textwidth]{figure/s1_estructural/Pina_Figura13.png}
 	\caption[Caras del ensamble donde se simula una fuerza proporcional a un cuerpo recostado sobre la cama.]{Caras del ensamble donde se simula una fuerza proporcional a un cuerpo recostado sobre la cama.\label{fig:Pina_Figura13}}
 \end{figure}

 \subsubsection*{Resultado de tensiones, desplazamientos, y factor de seguridad}
 El resultado de la tensión máxima es de 3.73 MPa y el límite elástico mínimo de los materiales utilizados se encuentra arriba de los 30 MPa, de tal manera que la estructura de la cama junto con el triplay puede soportar el peso de la persona con total seguridad y no va a fallar [Fig. \ref{fig:Pina_Figura14}].
 \begin{figure}[h!]
 	\centering
 	\subcaptionbox[Resultados de un estudio de tensiones en SolidWorks de acuerdo a las fuerzas de
 	la Fig. \ref{fig:Pina_Figura13}]{Resultados de un estudio de tensiones en SolidWorks de acuerdo a las fuerzas de
 		la Fig. \ref{fig:Pina_Figura13}\label{fig:Pina_Figura14}}
 	{\includegraphics [trim = 0 0cm 0 0, clip, width=0.49\textwidth]{figure/s1_estructural/Pina_Figura14.png}}
 	\subcaptionbox[Resultados de un estudio de desplazamientos en SolidWorks de acuerdo a las fuerzas de la Fig. \ref{fig:Pina_Figura13}]{Resultados de un estudio de desplazamientos de acuerdo a las fuerzas de
 		la Fig. \ref{fig:Pina_Figura13}\label{fig:Pina_Figura15}}
 	{\includegraphics [trim = 0 0cm 0 0, clip, width=0.49\textwidth]{figure/s1_estructural/Pina_Figura15.png}}
 	\subcaptionbox[Resultados de un estudio para conocer el Factor de Seguridad en SolidWorks de acuerdo a las fuerzas de la Fig. \ref{fig:Pina_Figura13}]{Resultados de un estudio para conocer el Factor de Seguridad en SolidWorks de acuerdo a las fuerzas de
 		la Fig. \ref{fig:Pina_Figura13}\label{fig:Pina_Figura16}}
 	{\includegraphics [trim = 0 0cm 0 0, clip, width=0.49\textwidth]{figure/s1_estructural/Pina_Figura16.png}}
 	\caption[Resultados de estudios en SolidWorks realizados al ensamble de la cama.]{Resultados de estudios en SolidWorks realizados al ensamble de la cama.\label{fig:Pina_Figura141516}}
 \end{figure}
 
Se observa en la Fig. \ref{fig:Pina_Figura15} que el desplazamiento máximo es de 0.01 mm y se encuentra en la región de la pierna izquierda; sin embargo, este desplazamiento es mínimo y no representa ningún peligro para el paciente. De igual manera, tampoco afecta a la estructura metálica, por lo que es un resultado favorable.\\ 
El factor de seguridad de la estructura de la cama es de 80, de tal forma que este diseño puede soportar el peso del paciente sin presentar inconvenientes y con un margen de seguridad muy amplio [Fig. \ref{fig:Pina_Figura16}].

 
