%\setlength{\parskip}{10cm}
\thispagestyle{plain}			% Supress header 
%\topskip0pt
\vspace*{1.8cm}
%\setlength{\parskip}{0pt plus 1.0pt}
\section*{\hfil Resumen}
\addcontentsline{toc}{section}{Resumen}
Este proyecto propone el desarrollo de una ortesis robótica diseñada para asistir en la rehabilitación de pacientes con hemiplejia en la extremidad inferior derecha. La hemiplejia, causada por accidentes cerebrovasculares, afecta gravemente la movilidad, lo que requiere un proceso intensivo de fisioterapia para prevenir complicaciones como la atrofia muscular y contracturas articulares. La ortesis robótica tiene como objetivo replicar los movimientos de flexión, extensión, abducción y aducción en las articulaciones de cadera (coxofemoral) y rodilla, ajustando los parámetros en función de las necesidades de cada paciente. Este dispositivo automatizado aliviará la carga física del fisioterapeuta través de una rehabilitación optimizada y personalizada. El sistema emplea motores para controlar los movimientos, junto con un sistema de retroalimentación y seguimiento de trayectoria monitoreados a través de un interfaz humano máquina. \\ 
% KEYWORDS (MAXIMUM 10 WORDS)

\textbf{Palabras clave:} Ortesis robótica, rehabilitación, hemiplejia, coxofemoral, rodilla, seguimiento de trayectoria.

\addcontentsline{toc}{section}{Abstract}
\section*{\hfil Abstract}
This project aims to develop a robotic orthosis designed to assist in the rehabilitation of patients with hemiplegia affecting the lower right limb. Hemiplegia, caused by strokes, severely impacts mobility, requiring intensive physiotherapy to prevent complications such as muscle atrophy and joint contractures. The robotic orthosis aims to replicate flexion, extension, abduction, and adduction movements in the hip (hip joint) and knee joints, adjusting parameters based on each patient's needs. This automated device will reduce the physical burden on the physiotherapist through an optimized and personalized rehabilitation. The system employs motors to control movements, along with a feedback and trajectory tracking system monitored through a human machine interface.\\

\textbf{Key words}: Robotic orthosis, rehabilitation, hemiplegia, hip joint, knee joint, trajectory tracking.

\vspace*{\fill}