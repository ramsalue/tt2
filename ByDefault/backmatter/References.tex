% CREATED BY MAGNUS GUSTAVER, 2020
% ------------------------------------------------------------------------------
% REFERENCIAS,.% ------------------------------------------------------------------------------
\begin{thebibliography}{99}
\bibitem{01} G. Cioni, G. Sgandurra, S. Muzzini, P. B. Paolicelli y A. Ferrari, “Forms of hemiplegia”, en The Spastic Forms of Cerebral Palsy. Milano: Springer Milan, 2010, pp. 331–356. Accedido el 1 de octubre de 2024. [En línea]. Disponible: \url{https://doi.org/10.1007/978-88-470-1478-7_16}
\bibitem{02} 	INEGI, “Estadística de defunciones registradas de enero a junio de 2022 (preliminar)”, Ciudad de México, 29/23, 24 de enero de 2023. Accedido el 1 de octubre de 2024. [En línea]. Disponible: \url{https://www.inegi.org.mx/contenidos/saladeprensa/boletines/2023/DR/DR-Ene-jun2022.pdf}
\bibitem{03} 	Estela. “Efectos de la inmovilización y tratamiento con fisioterapia”. Fisalia. Accedido el 1 de octubre de 2024. [En línea]. Disponible: \url{https://fisalia.es/efectos-inmovilizacion-tratamiento-fisioterapia/}
\bibitem{03_01}
Ú. Costa y S. Díez, “Robótica para la rehabilitación”, Sobreruedas, vol. 102, pp. 16–20, 2020. Accedido el 22 de noviembre de 2024. [En línea]. Disponible: \url{https://siidon.guttmann.com/files/sr_102_robotica_costa_diez.pdf}
\bibitem{03_02}
B. Y. Noa, M. Torres y J. Nodarse, “Terapia robótica en la rehabilitación del miembro superior hemipléjico en pacientes con enfermedad cerebrovascular”, Medimay, p. 6, 2021. Accedido el 22 de noviembre de 2024. [En línea]. Disponible: \url{https://www.medigraphic.com/pdfs/revciemedhab/cmh-2021/cmh211n.pdf}
\bibitem{20}  	Patrick, D. (2014). Orthotics. En A comprehensive guide to geriatric rehabilitation (pp. 497–500). Elsevier. \url{https://doi.org/10.1016/b978-0-7020-4588-2.00069-3}
\bibitem{04} 	Instituto Nacional de Seguridad e Higiene en el Trabajo, “Riesgos en trabajos de fisioterapia”, Erga Form. Prof., n.º 73, 2011. Accedido el 1 de octubre de 2024. [En línea]. Disponible: \url{https://www.insst.es/documents/94886/160640/N%C3%BAmero+73.+RIESGOS+EN+TRABAJOS+DE+FISIOTERAPIA.pdf}
\bibitem{05} 	A. S. Muñoz Pinto, “Lokomat en la re-educación de la marcha en personas hemipléjicas post accidente cerebro vascular”, Trabajo de grado, Univ. Tec. Ambato, Ambato, Ecuador, 2016. Accedido el 1 de octubre de 2024. [En línea]. Disponible: \url{https://repositorio.uta.edu.ec/jspui/bitstream/123456789/22862/2/PROYECTO%20DE%20INVESTIGACION%20LOKOMAT.pdf} 
\bibitem{06} 	E. Jiménez Vázquez, “Diseño de exoesqueleto de apoyo a la motricidad para la articulación de cadera”, Trabajo de grado, Inst. Politec. Nac., Ciudad de México, 2014. Accedido el 28 de septiembre de 2024. [En línea]. Disponible: \url{https://tesis.ipn.mx/jspui/bitstream/123456789/17893/1/Diseno%20de%20exoesqueleto%20de%20apoyo%20a%20la%20motricidad%20para%20la%20articulacion%20de%20cadera.pdf}
\bibitem{07} 	C. H. Guzmán Valdivia, A. Blanco Ortega, M. A. Oliver Salazar, F. A. Gómez Becerra y J. L. Carrera Escobedo, “HipBot – The design, development and control of a therapeutic robot for hip rehabilitation”, Mechatronics, vol. 30, pp. 55–64, septiembre de 2015. Accedido el 1 de octubre de 2024. [En línea]. Disponible: \url{https://doi.org/10.1016/j.mechatronics.2015.06.007}
\bibitem{08} 	C. B. Sanz-Morère et al., “An active knee orthosis with a variable transmission ratio through a motorized dual clutch”, Mechatronics, vol. 94, p. 103018, octubre de 2023. Accedido el 1 de octubre de 2024. [En línea]. Disponible: \url{https://doi.org/10.1016/j.mechatronics.2023.103018}
\bibitem{09} 	López, R., Aguilar, H., Salazar, S., Lozano, R., \& Torres, J. A. (2014). Modelado y Control de un Exoesqueleto para la Rehabilitación de Extremidad Inferior con dos grados de libertad. Revista Iberoamericana de Automática e Informática Industrial RIAI, 11(3), 304–314. Accedido el 1 de octubre de 2024. [En línea]. Disponible: \url{https://doi.org/10.1016/j.riai.2014.02.008}
\bibitem{10} 	X. L. Albornoz Tepán, “Robotic orthosis for bilateral rehabilitation of left hand for patients with hemiplegia”, I+T+C Investig. Tecnol. Cienc., vol. 1, no. 13, pp. 10–15, Dec. 2019. Accedido el 1 de octubre de 2024. [En línea]. Disponible: \url{https://revistas.unicomfacauca.edu.co/ojs/index.php/itc/article/view/itc2019_pag_10_15 }
\bibitem{11} 	Alianza B@UNAM, CCH \& ENP ante la pandemia. (2024, 18 de febrero). Planos anatómicos y términos direccionales. https://alianza.bunam.unam.mx/enp/planos-anatomicos-y-terminos-direccionales/ 
\bibitem{12}	Articulaciones y ligamentos. (2024). Visible Body. \url{https://www.visiblebody.com/es/learn/skeleton/joints-and-ligaments}
\bibitem{13}	Azucas, R. (2023, 30 de octubre). Articulación coxofemoral. Kenhub. \url{https://www.kenhub.com/es/library/anatomia-es/articulacion-coxofemoral}
\bibitem{14} 	Saldivia Paredes, M. (2018). Descripción morfológica y biomecánica de la articulación de la rodilla del canino (Canis lupus familiaris). CES Medicina Veterinaria y Zootecnia, 13(3), 294–307. \url{https://doi.org/10.21615/cesmvz.13.3.1}
\bibitem{15} 	Navarro, B. (2023, 30 de octubre). Tipos de movimientos del cuerpo humano. Kenhub. \url{https://www.kenhub.com/es/library/anatomia-es/tipos-de-movimientos-del-cuerpo-humano}
\bibitem{16} 	Junquera, M. (2013, 18 de abril). Fases de la hemiplejia. FisioOnline. \url{https://www.fisioterapia-online.com/articulos/fases-de-la-hemiplejia}
\bibitem{17} 	American Stroke Association. (2019, 7 de abril). Hemiparesis. www.stroke.org. \url{https://www.stroke.org/en/about-stroke/effects-of-stroke/physical-effects/hemiparesis}
\bibitem{18}  	Con la EM. (s.f.). ¿Qué es la hemiparesia? Causas, síntomas y tratamiento. Esclerosis Múltiple: Todo lo que debes saber. \url{https://www.conlaem.es/esclerosis-multiple/glosario/hemiparesia}
\bibitem{19}  	Panzhinskiy, E., Culver, B., Ren, J., Bagchi, D., \& Nair, S. (2019). Role of mammalian target of rapamycin in muscle growth. En Nutrition and enhanced sports performance (pp. 251–261). Elsevier. https:\url{//doi.org/10.1016/b978-0-12-813922-6.00022-9}
%\bibitem{21} 	Edelstein, J. E. (2007). Orthotics. En Physical rehabilitation (pp. 897–917). Elsevier. \url{https://doi.org/10.1016/b978-072160361-2.50037-5}
%\bibitem{22}	Hovorka, C., \& Acker, D. (2019). Orthotic treatment considerations for arthritis and overuse syndromes in the upper limb. En Atlas of orthoses and assistive devices (pp. 176–197.e2). Elsevier. \url{https://doi.org/10.1016/b978-0-323-48323-0.00016-0}
%\bibitem{23} 	Supan, T. J. (2019). Principles of fabrication. En Atlas of orthoses and assistive devices (pp. 42–48.e1). Elsevier. \url{https://doi.org/10.1016/b978-0-323-48323-0.00003-2}
\bibitem{24} 	Muscular dystrophy - Symptoms \& causes. (2022, 11 de febrero). Mayo Clinic. \url{https://www.mayoclinic.org/diseases-conditions/muscular-dystrophy/symptoms-causes/syc-20375388}
\bibitem{25} 	Nagy, P. V. (1989). Trajectory tracking control for industrial robots. Journal of Mechanical Working Technology, 20, 273–281. \url{https://doi.org/10.1016/0378-3804(89)90037-5}
\bibitem{26} 	Siciliano, B., Villani, L., \& Sciavicco, L. (2009). Robotics: Modelling, planning and control. Springer.

\bibitem{30} 	Gausemeier, J., \& Moehringer, S. (2002). VDI 2206- A new guideline for the design of mechatronic systems. IFAC Proceedings Volumes, 35(2), 785–790. \url{ https://doi.org/10.1016/s1474-6670(17)34035-1}
\bibitem{27} D. Levack, B. DeHoff, and R. Rhodes, “Functional Breakdown Structure (FBS) and Its Relationship to Life Cycle Cost”, in 45th AIAA/ASME/SAE/ASEE Joint Propulsion Conf. Exhibit, Denver, Colorado. Reston, Virigina: Amer. Inst. Aeronaut. Astronaut., 2009. Accessed: Jun. 4, 2025. [Online]. \url{https://doi.org/10.2514/6.2009-5344}
\bibitem{28} “Federal Information Processing Standards Publication: integration definition for function modeling (IDEF0)”, National Institute of Standards and Technology, Gaithersburg, MD, 1993. Accessed: Jun. 4, 2025. [Online]. \url{https://doi.org/10.6028/nist.fips.183}
\bibitem{31} 	InformedHealth.org. (2024, 19 de marzo). In brief: Physical therapy. National Center for Biotechnology Information. \url{https://www.ncbi.nlm.nih.gov/books/NBK561514/}

\bibitem{32} 	Bonaldo, P., \& Sandri, M. (2012). Cellular and molecular mechanisms of muscle atrophy. Disease Models \& Mechanisms, 6(1), 25–39. \url{https://doi.org/10.1242/dmm.010389}
\bibitem{33} 	Panzhinskiy, E., Culver, B., Ren, J., Bagchi, D., \& Nair, S. (2019). Role of mammalian target of rapamycin in muscle growth. En Nutrition and enhanced sports performance (pp. 251–261). Elsevier. \url{https://doi.org/10.1016/b978-0-12-813922-6.00022-9}

\bibitem{34} 	Sun, X., Xu, K., Shi, Y., Li, H., Li, R., Yang, S., Jin, H., Feng, C., Li, B., Xing, C., Qu, Y., Wang, Q., Chen, Y., \& Yang, T. (2021). Discussion on the rehabilitation of stroke hemiplegia based on interdisciplinary combination of medicine and engineering. Evidence-Based Complementary and Alternative Medicine, 2021, 1–11. \url{https://doi.org/10.1155/2021/6631835}

\bibitem{37} Segments. (n.d.). ExRx.net. \url{https://exrx.net/Kinesiology/Segments}

\bibitem{35} Meyer, D. (n.d.). Emergency stop circuit. Eaton. \url{https://www.eaton.com/gb/en-gb/markets/machine-building/service-and-support-machine-building-moem-service-eaton/blogs/emergency-stop-circuit---blogs---eaton.html}
\bibitem{38} Littelfuse, "SMAJ Series Surface Mount 400W", [Revisado 22/07/2022]
\bibitem{36} Martínez, A. (2025, January 30). ISO 13850:2015 Categorías de parada de emergencia: Seguridad en acción para máquinas. \url{https://industriaaldia.com/2025/01/30/iso-138502015-categorias-de-parada-de-emergencia/}

%Referencia insertadas para TT2
\bibitem{TT2_01} R. Ávila, L. R. Prado y E. L. González, Dimensiones antropométricas de la población latinoamericana, 2a ed. Guadalajara, Jalisco: Cent. Univ. Arte, Arquit., Diseno, 2007.
\bibitem{TT2_02} EFE. “Dan a conocer cuánto mide y cuánto pesa el mexicano promedio”. FashionNetwork.com. Accedido el 22 de septiembre de 2025. [En línea]. Disponible: \url{https://n9.cl/b55tt}






\end{thebibliography}
