\clearpage
\section*{Conclusiones}
\addcontentsline{toc}{section}{Conclusiones}
En el desarrollo de Trabajo Terminal I para el proyecto de Ortesis robótica para asistencia del movimiento de coxofemoral y rodilla en adultos con hemiplejia derecha se realizó un replanteamiento de la arquitectura funcional y arquitectura física con base en una investigación más profunda sobre los requerimientos del proyecto, esto en principio debido al uso de la metodología VDI 2206 que fue implementada para el proyecto, la cual utiliza los requerimientos como base para desarrollo de proyectos mecatrónicos. A través de herramientas como el diagrama morfológico y matrices de selección subjetivas, binarias y de ponderación se plantearon posibles soluciones  y se concibieron conceptos solución del cual se eligió uno para desarrollar el diseño detallado del proyecto.

Se realizó el diseño del sistema estructural así como de los mecanismos tanto para los movimientos de flexión-extensión como de abducción-aducción a través de SolidWorks validando las propiedades con comportamientos mecánicos utilizando diferentes materiales y análisis por medio de cargas máximas establecidas. Dentro de la parte mecánica también se diseñaron y validaron topes mecánicos de seguridad para proteger al paciente en caso de fallos en el control del sistema, así como la implementación de un botón de paro de emergencia apegado a la norma ISO 13850:2015 que detalla las características esenciales que debe cumplir dicho botón (colores, mecanismo de desactivación). 

También se llevó a cabo la selección de una fuente conmutada de tensión  para la etapa de potencia de motores, mismas que se verificó a través de la revisión de sus hojas de especificaciones tuvieran la capacidad de proporcionar la alimentación tanto de corriente como de tensión para los motores, y una segunda fuente para el acondicionamiento de energía para el sistema que permitiera alimentar los componentes de baja potencia. Además, se diseñó el sistema de seguridad eléctrico integrando y validando componentes que desconectaran al sistema de la energía eléctrica en caso de picos de corriente o tensiones prolongados.

Así mismo, dentro del diseño del sistema de movimiento y de control se seleccionaron motores y drivers, y se validó su selección verificando que fueran capaces de generar los movimientos con las cargas establecidas y resultados de fuerzas y presiones obtenidos a través de la validación del sistema estructural.

Se realizó la propuesta de diseño del sistema de comunicación humano máquina en el cual se definió la forma de interacción del usuario con el sistema, y se validó su funcionamiento verificando su compatibilidad con los elementos seleccionados para el sistema de control así como la capacidad de estos, específicamente del microcontrolador seleccionado.

Finalmente, los componentes físicos del proyecto fueron integrados computacionalmente, generando un ensamble que sirve como referencia para su implementación y verificación en Trabajo Terminal II.

\section*{Recomendaciones y trabajo a futuro}
\addcontentsline{toc}{section}{Recomendaciones y trabajo a futuro}
\subsection*{Falta de energía eléctrica con paciente acostado en cama}
\label{subsec:falta_energia_paciente}
\addcontentsline{toc}{subsection}{Falta de energía eléctrica con paciente acostado en cama}
Durante la operación del sistema de la ortesis existe la posibilidad de una pérdida del suministro eléctrico mientras el paciente se encuentra utilizando el dispositivo. 
Este escenario representa un riesgo de inmovilización del usuario y posibles lesiones debido a la detención inesperada de los actuadores y la pérdida de control de los mecanismos de sujeción.
\subsubsection*{Consecuencias principales}
\addcontentsline{toc}{subsubsection}{Consecuencias principales}
\begin{itemize}
	\item Interrupción inmediata del movimiento, que puede provocar incomodidad o dolor al paciente si el dispositivo 
	se detiene en una posición desfavorable.
	\item Dificultad para retirar al paciente del sistema en caso de bloqueo mecánico.
\end{itemize}
\subsubsection*{Medidas de mitigación propuestas (recomendación a futuro)}
\addcontentsline{toc}{subsubsection}{Medidas de mitigación propuestas (recomendación a futuro)}
Aunque en la presente fase del proyecto (Trabajo Terminal II) no se implementará una solución definitiva, se recomienda considerar las siguientes acciones en el desarrollo posterior:
\begin{itemize}
	\item \textbf{Sistema de respaldo de energía}: Incorporar una unidad de alimentación ininterrumpida (UPS) 
	o batería de respaldo que permita, al menos, liberar los actuadores lineales y rotativos para 
	retirar al paciente con seguridad.
	\item \textbf{Mecanismo de liberación manual}: Diseñar un sistema mecánico fail-safe que permita desacoplar o destrabar los actuadores de forma manual, buscando con ello la movilidad libre del miembro ortopédico.
	\item \textbf{Alarma sonora y visual}: Implementar una señal acústica y luminosa que alerte inmediatamente al personal de la pérdida de alimentación eléctrica, facilitando una evacuación oportuna.
\end{itemize}

Estas recomendaciones deberán considerarse en el diseño final de la etapa de construcción o en trabajos de mejora posteriores, con el objetivo de aumentar la seguridad del paciente ante situaciones de interrupción eléctrica.