\subsection*{Antecedentes}
\addcontentsline{toc}{subsection}{Antecedentes}
En el campo de la rehabilitación asistida por tecnología, se han desarrollado diversas soluciones robóticas y de ortesis para mejorar la calidad de vida de los pacientes que presentan limitaciones en el movimiento.
Estos proyectos se han enfocado para asistir, corregir o potenciar el proceso de rehabilitación, utilizando desde sistemas mecánicos básicos hasta dispositivos mecatrónicos que integran sensores, actuadores y control inteligente.\\

A continuación, se presenta una revisión de algunos de los proyectos más relevantes que han servido como base y referencia para el desarrollo de la ortesis robótica que se propone en este trabajo.
\begin{table}[h!]
	\centering
	\caption{Antecedentes.\label{tab:01}}
	\rotatebox{90}{\scalebox{0.85}{\begin{tabular}{|m{0.8cm}|m{2.5cm}|m{5cm}|m{7cm}|m{1.4cm}|m{4cm}|m{2cm}|m{0.5cm}|}
			\hline
			Ítem               & Nombre                                                                                                                               & Descripción                                                                                                                                                                                                                                                          & Características                                                                                                                                                  & País                             & Instituto                                                                                                                                                                    & Tipo                                        & Ref                \\ \hline
			{1} & {Lokomat en la re-educación de la marcha en personas   hemipléjicas post accidente cerebro vascular.}                 & {Sistema robótico   diseñado para la rehabilitación funcional de la marcha en personas que sufren   secuelas producidas por un daño neurológico tanto a nivel cerebral como en la   médula espinal.}                                                  & {\begin{itemize}
					\item Módulo que mejora   la terapia al permitir movimientos laterales y rotacionales de la pelvis.
					\item	Motores sincronizados a una computadora.
					\item	Ajuste en parámetros de entrenamiento.
					\item	Interfaz de fácil operación para el terapeuta.
					
			\end{itemize}}                                                   & {Ecuador}         & {Universidad   Técnica de Ambato.}                                                                                                                            & {Informe de   investigación} & {\cite{05}}  \\ \hline
			{2} & {Diseño de   exoesqueleto de apoyo a la motricidad para la articulación de cadera.}                                   & {Prototipo exoesqueletico para apoyo en la   articulación de la cadera, para guiar el movimiento en partes inferiores   durante el ciclo de marcha y posición del usuario.}                                                                           & {\begin{itemize}
					\item Actuadores   lineales eléctricos.
					\item Diseño biomecanico.
					\item Piezas realizadas en nylamid.
					\item Sistema de sujección por arneses y correas.
					\item Piezas fabricadas en máquinas CNC.
					
			\end{itemize}}                                                                                                               & {México}          & {Instituto   Politécnico Nacional}                                                                                                                            & {Tesis}                      & {\cite{06}}  \\ \hline
		{3} & {HipBot}                                                                                                              & {Robot terapéutico   diseñado para la rehabilitación de la articulación de la cadera, siendo capaz   de realizar movimientos combinados de abducción/aducción y flexión/extensión   de la cadera, replicando movimientos necesarios en fisioterapia.} & {\begin{itemize}
		\item Posee 5 grados de libertad.
		\item Realiza movimientos combinados laterales y frontales.
		\item Emplea controlador PID para seguimiento de trayectorias.
		\item Cuenta con botones de emergencia y sensores de fuerza para detener el sistema en caso de anomalía.
		
	\end{itemize}}                                                                                                                   & {México}          &{Departamento de   Ingeniería Mecatrónica de la Universidad Politécnica de Zacatecas; Centro   Nacional de Investigación y Desarrollo Tecnológico (CENIDET).} &{Artículo   científico}      & {\cite{07}}  \\ \hline
	\end{tabular}}}
\end{table}
\clearpage
\begin{table}[h!]
	\centering
	\rotatebox{90}{\scalebox{0.85}{\begin{tabular}{|m{0.8cm}|m{2.5cm}|m{5cm}|m{7cm}|m{1.4cm}|m{4cm}|m{2cm}|m{0.5cm}|}
			\hline
			Ítem               & Nombre                                                                                                                               & Descripción                                                                                                                                                                                                                                                          & Características                                                                                                                                                  & País                             & Instituto                                                                                                                                                                    & Tipo                                        & Ref                \\ \hline
			
			{4} & {Ortesis activa de   rodilla con una relación de transmisión variable a través de un embrague   doble motorizado.}    & {Ortesis activa de   rodilla (AKO) destinada a asistir a personas con movilidad reducida.}                                                                                                                                                            & {\begin{itemize}
					\item Cuenta con un actuador que permite seleccionar entre dos modos: alta torsión y baja velocidad; y baja torsión y alta velocidad.
					\item El diseño es simétrico y puede ser utilizado en ambas piernas (izquierda o derecha).
					\item Incluye un resorte torsional como actuador elástico en serie diseñado para soportar una torsión de 50Nm, y rigidez de 150 Nm/rad.
					\item Tiene una masa de 3.8 kg incluyendo unidad de control.
					\item Posee un controlador adaptativo que ajusta el momento aplicado.
					
			\end{itemize}}             & {Italia}          & {Instituto de Bio   Robótica y Departamento de Excelencia en Robótica e Inteligencia Artificial   de la Escuela Superior Santa Ana}                           & {Artículo   científico}      & {\cite{08}}  \\ \hline
			{5} & {Modelado y Control   de un Exoesqueleto para la Rehabilitación de Extremidad Inferior con dos   grados de libertad.} & {Exoesqueleto de dos grados de libertad diseñado para   realizar ejercicios de rehabilitación de tobillo y rodilla, para las personas   que, a causa de algún accidente, o enfermedad tienen movilidad reducida o   nula.}                            & {\begin{itemize}
					\item Cuenta con actuadores tipo SEA (Series Elastic Actuator) que son utilizados para amplificar la fuerza humana con ayuda de algunos sensores.
					\item Utiliza sensores para medir la posición y velocidad angular de las articulaciones, que se utilizan para controlar el movimiento de la pierna.				
			\end{itemize}} & {México, Francia} &{Centro de Investigación y de Estudios Avanzados del   Instituto Politécnico Nacional, Université de Technologie de Compiegne}                                & {Artículo   científico}      & {\cite{09}} \\ \hline
	\end{tabular}}}
\end{table}
\clearpage
\begin{table}[h!]
	\centering
	\rotatebox{90}{\scalebox{0.85}{\begin{tabular}{|m{0.8cm}|m{2.5cm}|m{5cm}|m{7cm}|m{1.4cm}|m{4cm}|m{2cm}|m{0.5cm}|}
				\hline
				Ítem               & Nombre                                                                                                                               & Descripción                                                                                                                                                                                                                                                          & Características                                                                                                                                                  & País                             & Instituto                                                                                                                                                                    & Tipo                                        & Ref                \\ \hline
				{6} & {Ortesis robótica para rehabilitación bilateral para   la mano izquierda para pacientes con hemiplejia.}              & {Sistema para   realizar rehabilitación bilateral en pacientes sobrevivientes a un accidente   cerebro vascular o con dificultad de movimiento en la mano izquierda, basado   en terapia espejo.}                                                     & {\begin{itemize}
						\item Entrega retroalimentación neuronal al imitar el movimiento de flexión-extensión de los dedos de la mano sana en la afectada.
						\item Identifica el rango de movimiento de cada dedo de la mano derecha, midiendo la resistencia de sensores flex ubicados en todos ellos.
						\item Replica el movimiento de cada dedo en la mano afectada guiados por servomotores acoplados a un sistema mecánico
						
				\end{itemize}}                & {Ecuador}         & {Universidad de   Cuenca}                                                                                                                                     & {Artículo   científico}      &{\cite{10}}  \\ \hline
	\end{tabular}}}
\end{table}

En conclusión, los proyectos revisados en este apartado han sido fundamentales para el avance en la creación de ortesis y sistemas robóticos orientados a la rehabilitación, sin embargo, siempre existirá la necesidad de desarrollar soluciones más adaptables, personalizadas y eficientes para abordar de manera integral las diversas patologías que afectan la movilidad de los pacientes.\\

La ortesis robótica propuesta en este proyecto busca aprovechar los aprendizajes y avances de los trabajos previos para buscar un enfoque más preciso y eficaz en la rehabilitación del coxofemoral y rodilla. A través de la implementación de un diseño modular, motores, y un sistema de control. Este dispositivo pretende superar los retos existentes y ofrecer un apoyo para mejorar la calidad de vida de los pacientes.

\subsection*{Organización del documento}
\addcontentsline{toc}{subsection}{Organización del documento}
El presente documento para Trabajo Terminal I está estructurado en secciones y subsecciones en las cuales se desglosa el desarrollo conceptual del proyecto de una ortesis robótica para asistencia del movimiento de coxofemoral y rodilla en adultos con hemiplejia derecha. La primera sección que comienza con la introducción establece el contexto del proyecto, también se aborda el enfoque mecatrónico desde el cual se concibe la ortesis, se delimita el planteamiento del problema a resolver, se justifica la relevancia del trabajo a través de la justificación, y se definen los objetivos específicos que guiarán el desarrollo. Además, se presenta una la revisión de los antecedentes que sustentan el proyecto, analizando soluciones previas y tecnologías relacionadas.\\

Posteriormente, en la sección \nameref{Marco de referencia} desarrollan las bases teóricas y conceptuales fundamentales. Dentro de esta, la subsección del marco teórico profundiza en los principios científicos y de ingeniería esenciales para el diseño de la ortesis, incluyendo una revisión de los planos anatómicos y las articulaciones relevantes para la movilidad de la extremidad inferior afectada.\\

En la sección \nameref{Diseño del sistema} se presenta en primera instancia el diseño conceptual en el cual se declaran las necesidades, y con base en ellas se establecen los requerimientos del proyecto. Apoyados en dichos requerimientos se define la arquitectura funcional, se crea la estructura FBS del proyecto en torno a la función global de mover la extremidad inferior derecha. Seguido de esto, se presenta el IDEF-0 con sus diagramas del nodo A0 compacto y extendido para establecer la arquitectura física del proyecto. Posteriormente se presentan las propuestas solución para cada sistema, los diseño conceptuales basados en dichas propuestas para generar así conceptos solución de los cuales, a través de matrices de selección subjetiva, binaria y ponderación elegir el concepto más apropiado.\\

En la subsección \nameref{Diseño detallado} se detalla el proceso completo de diseño de cada sistema aplicando la metodología mecatrónica de forma modular. Con el propósito de organizar la documentación se estructuró de forma secuencial siguiendo la numeración de cada sistema y módulo en la arquitectura física, sin embargo, el diseño se realizó de forma simultánea de uno o más sistemas al mismo tiempo, es por ello que la redacción de la documentación presenta referencias cruzadas para comprender la mención de componentes, cálculos, factores que hayan sido necesarios para un módulo y el contexto y validación de su selección sean parte de otro módulo o sistema. En esta misma subsección se encuentra la integración de los sistemas de la ortesis así como el plan de pruebas y validación que dan pie al desarrollo de Trabajo Terminal II.\\

Después del diseño del sistema se presenta lo relacionado a la implementación del sistema como parte de Trabajo Terminal II. Se describen las especificaciones del usuario para los cuales el sistema está diseñado y se procede con la implementación de cada uno de los sistemas y sus respectivos módulos. \\

 Llegando al final del documento se tiene el análisis de resultados donde se presentan aspectos relacionados con la administración del proyecto, desde los costos realizados para el desarrollo del proyecto hasta el cronograma planteado para Trabajo Terminal II. Finalmente, se describen las conclusiones en función de los avances obtenidos y en relación con los objetivos propuestos, además se incluyen en anexos hojas de especificaciones de componentes utilizados, así como planos de manufactura del sistema estructural.
