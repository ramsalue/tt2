\subsection*{Justificación}
\addcontentsline{toc}{subsection}{Justificación}
En la fase flácida de la hemiplejia, donde la extremidad afectada carece de tono muscular y se encuentra sin movimiento activo, la intervención del fisioterapeuta se centra en evitar estas complicaciones mediante movilizaciones pasivas controladas. En este contexto, la fisioterapia tradicional implementada con estos pacientes es altamente demandante desde el punto de vista físico para el fisioterapeuta. Las sesiones de rehabilitación requieren una atención exhaustiva y un gran esfuerzo físico dado que la terapia se da con el paciente recostado, lo que genera molestias para el fisioterapeuta como dolor de cintura por estar inclinado, y la necesidad de supervisión para garantizar una adecuada ejecución. Esta demanda, sumada a la falta de recursos en las clínicas, subraya la necesidad de soluciones alternativas que permitan una rehabilitación más eficiente, con mejores resultados y un uso optimizado de los recursos\cite{04}.\\

Por ello, la ortesis robótica para pacientes hemipléjicos se presenta como una solución innovadora, que integra la mecánica, programación, electrónica, sistemas de control y medicina física, además de permitir al fisioterapeuta realizar las sesiones de rehabilitación con facilidad. Para lograr esta propuesta de solución se requiere un mecanismo que realice los movimientos de flexión y extensión en la rodilla y coxofemoral, un mecanismo para controlar la abducción y aducción en la coxofemoral, una interfaz humano-máquina para ingresar y visualizar los parámetros, sistemas de seguridad para el paciente y una estructura mecánica donde se puedan integrar los mecanismos y componentes de la ortesis robótica.\\

Finalmente, la combinación de estos mecanismos y componentes otorga una ventaja considerable sobre otros dispositivos meramente mecánicos, ya que, permite ajustar los parámetros de las sesiones de rehabilitación en términos de fuerza, duración y rangos de movimiento, lo que posibilita un enfoque personalizado para cada paciente. Al automatizar el proceso de rehabilitación, se alivia la carga física del fisioterapeuta y se asegura una terapia consistente, continua y apegada a los objetivos terapéuticos definidos por los especialistas.

