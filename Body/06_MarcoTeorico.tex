\section{Marco de referencia}\label{Marco de referencia}
\subsection{Marco teórico}
El desarrollo del proyecto abarca disciplinas más allá de los alcances de la ingeniería mecatrónica, de los cuales resulta importante tener una noción para poder ser aplicados en el desarrollo de la ortesis robótica.
\subsubsection{Planos anatómicos}
Al cuerpo humano se le realizan tres cortes imaginarios para poder ubicar las estructuras y órganos que los componen \cite{11}. Dichos cortes son conocidos como planos anatómicos y se observan en la Figura \ref{img:P01_PlanosA}. Estos planos son:
\begin{enumerate}
	\item Sagital o medio sagital, el cual divide al cuerpo humano en mitad derecha e izquierda.
	\item Frontal o coronal, que divide en mitad anterior y posterior.
	\item Transversal u horizontal, que divide al cuerpo en mitad superior e inferior.
\end{enumerate}
\begin{figure}[h!]
	\centering
	\includegraphics [trim = 0 0cm 0 0, clip, width=0.75\textwidth]{figure/P01_PlanosA.png}
	\caption[Planos anatómicos del cuerpo humano.]{Planos anatómicos del cuerpo humano. Recuperado de \cite{11}.\label{img:P01_PlanosA}}
\end{figure}

\subsubsection{Articulaciones}
Las articulaciones conectan los huesos del esqueleto y permiten el soporte y la ejecución de movimientos. Existen dos formas principales de clasificarlas. La primera es de acuerdo con su función, es decir, el rango de movimiento que permiten. La segunda clasificación se basa en el material que une los huesos \cite{12}.\\

Una de las articulaciones de interés para el proyecto es la articulación de la coxofemoral, también conocida como cadera, la cual es una articulación multiaxial sinovial que permite una gran variedad de movimientos, entre ellos la flexión, extensión, abducción, aducción, rotación interna y externa, así como la circunducción, lo que le otorga una amplia movilidad en diferentes direcciones \cite{13}.\\

Otra de las articulaciones con la cual se trabajará en el proyecto es la articulación de la rodilla, también conocida como la articulación femoro-tibio patelar, es una articulación de gran rango de movilidad, clasificada como sinovial o diartrodial, permitiendo principalmente los movimientos de flexión y extensión \cite{14}.

\subsubsection{Movimientos del cuerpo humano}
En anatomía, el concepto de movimiento involucra el desplazamiento de huesos o partes del cuerpo alrededor de articulaciones fijas, en relación con los principales ejes anatómicos (sagital, coronal, transversal) o planos paralelos a estos \cite{15}. 
Así, el esquema de los movimientos anatómicos se compone de lo siguiente:
\begin{enumerate}
	\item Estructuras anatómicas que participan en el movimiento.
	\item Ejes de referencia alrededor de los cuales ocurre el movimiento.
	\item Dirección del movimiento, que en anatomía suele vincularse con un plano estándar, como el mediano, medial, sagital, o frontal.
\end{enumerate}
\clearpage
Entre los movimientos del cuerpo humano se encuentra la flexión/extensión, los cuales son movimientos opuestos que tienen lugar en direcciones sagitales alrededor de un eje frontal/coronal. La flexión se refiere a la acción de reducir el ángulo entre dos estructuras que intervienen en el movimiento, como huesos o partes del cuerpo. En contraste, la extensión o el acto de enderezar implica aumentar el ángulo entre dichas estructuras. Este tipo de movimiento se presenta en la rodilla, donde la tibia de la pierna se mueve con relación al fémur del muslo, y ocurre en el plano sagital. En el movimiento de flexión, la pierna se mueve hacia atrás, y durante la extensión, se mueve hacia adelante \cite{15}. En la Fig. \ref{fig:M01} se representan los movimientos de flexión y extensión de la articulación de la rodilla.
\begin{figure}[h!]
	\centering
	\subcaptionbox{Flexión.\label{fig:F01_01}}
	{\includegraphics [trim = 0 0cm 0 0, clip,width=7cm]{figure/F02_01.jpg}}
	\subcaptionbox{Extensión.\label{fig:F01_02}}
	{\includegraphics [trim = 0 0cm 0 0, clip,width=7cm]{figure/F04_01.jpg}}
	\caption{Movimientos de flexión/extensión de rodilla.}\label{fig:M01}
\end{figure}

    En la Fig. \ref{fig:M02} se representan los movimientos de flexión y extensión que involucran únicamente la articulación del coxofemoral.
\begin{figure}[h!]
	\centering
	\subcaptionbox{Flexión.\label{fig:F05_01}}
	{\includegraphics [trim = 0 0cm 0 0, clip,width=7cm]{figure/F05_01.jpg}}
	\subcaptionbox{Extensión.\label{fig:F04_01}}
	{\includegraphics [trim = 0 0cm 0 0, clip,width=7cm]{figure/F04_01.jpg}}
	\caption{Movimientos de flexión/extensión de articulación de coxofemoral.}\label{fig:M02}
\end{figure}

Por su parte, los movimientos de abducción/aducción, los cuales están estrechamente relacionados con el plano medial del cuerpo. Ambos movimientos se desarrollan alrededor de un eje anteroposterior, lo que significa que se desplazan hacia adelante y hacia atrás. En términos anatómicos, estos movimientos son más fáciles de entender al observar las piernas y los brazos, ya que su dinámica es bastante similar. El brazo se mueve con respecto al tronco y al hombro, mientras que la pierna lo hace en relación con la articulación coxofemoral. El movimiento ocurre en el plano frontal \cite{15}. Los movimientos de abducción y aducción se representan en la Fig. \ref{fig:M03}.\\
\begin{figure}[h!]
	\centering
	\subcaptionbox{Abducción.\label{fig:F07_01}}
	{\includegraphics [trim = 0 0cm 0 0, clip,width=7cm]{figure/F07_01.jpg}}
	\subcaptionbox{Aducción.\label{fig:F06_01}}
	{\includegraphics [trim = 0 0cm 0 0, clip,width=7cm]{figure/F06_01.jpg}}
	\caption{Movimientos de abducción/aducción de articulación de coxofemoral.}\label{fig:M03}
\end{figure}

Los movimientos del cuerpo se ven afectados por condiciones neurológicas como la hemiplejia, la cual se define continuación.

\subsubsection{Hemiplejia}
La hemiplejia es un término general que se le otorga a una condición crónica que afecta al sistema nervioso central, provocando alteraciones principalmente en la sensibilidad y el control de la acción motora de un lado del cuerpo. Aunque esta afección impacta mayormente un hemicuerpo, también causa otros problemas en diferentes áreas del cuerpo que van más allá del lado afectado \cite{16}. \\

La hemiplejia consta de cuatro fases genéricas:
\begin{enumerate}
	\item Etapa inicial o de ictus: Tras el evento, el paciente puede estar en coma o semicoma. La duración de esta fase es variable, ya que, se identifica el hemisferio cerebral afectado, pero no su alcance funcional.
		\item Fase flácida: El hemisferio cerebral está inhibido, lo que provoca flacidez en el hemicuerpo afectado. El hombro cae, la cabeza se inclina, y el pie se arrastra. Los trastornos sensitivos como la hipoestesia y la hiperestesia también están presentes. Esta fase finaliza con el inicio de la hipertonía.
		\item	Etapa espástica: Aparece la hipertonía, lo que conduce a posturas fijas debido a la rigidez de los músculos. En el miembro inferior afecta la articulación del coxofemoral y el pie. También pueden presentarse alteraciones vegetativas y afasia.
		\item	Fase de secuelas: En torno a los dos años de la primera fase se ha producido toda la recuperación espontánea posible. En esta fase el paciente debe adaptarse a las secuelas buscando mejorar su funcionalidad a pesar de que no se esperan más avances significativos. Esta fase es sometida a tratamientos con el propósito de mitigar las secuelas y maximizar la autonomía del paciente.
\end{enumerate}
\subsubsection{Diferencias entre hemiplejia y hemiparesia}
La hemiplejia se caracteriza por una parálisis total de uno de los lados del cuerpo a causa de una lesión o alteración en el cerebro o sistema nervioso. Mientras que la hemiparesia se refiere a una debilidad o disminución del control muscular en la mitad del cuerpo, pero sin llegar a una parálisis completa. Una persona con hemiparesia aún conserva cierto grado de movilidad en la parte afectada \cite{18}. 
\subsubsection{Atrofia y distrofia muscular }
La atrofia muscular se refiere a una disminución en la masa del músculo, lo que puede llevar a una pérdida parcial o total del tejido muscular. Esta condición puede ser provocada por diversas enfermedades comunes como el cáncer, la diabetes y la insuficiencia renal, así como por quemaduras graves, desnutrición o la falta de uso de los músculos. También puede ser causada por lesiones en la médula espinal, como la paraplejia, que afecta la función motora o sensorial de las extremidades inferiores \cite{19}. Por su parte, la distrofia muscular es un conjunto de enfermedades que provocan una debilidad progresiva y pérdida de masa muscular. En esta condición, los genes anormales (mutaciones) afectan la producción de proteínas necesarias para la formación y el mantenimiento de músculos sanos \cite{19}. Las condiciones médicas descritas son atendidas por tecnologías de rehabilitación y asistencia, entre los cuales se encuentra las ortesis y fisioterapias. 
\subsubsection{Fisioterapia}
La fisioterapia, también conocida como terapia física, se enfoca en aliviar el dolor, mejorar la movilidad y fortalecer los músculos debilitados a través de ejercicios, masajes y tratamientos con estímulos físicos como calor, frío, corrientes eléctricas y ultrasonido. Además de su aplicación en la clínica, uno de sus objetivos clave es enseñar a los pacientes a mejorar su salud de manera independiente, fomentando la práctica de ejercicios en casa \cite{24}. Esta terapia incluye tanto movimientos activos realizados por el paciente, como movimientos pasivos guiados por el terapeuta, y utiliza diversas técnicas para tratar síntomas y prevenir problemas futuros.
\subsubsection{Seguimiento de trayectoria}
Es el proceso de diseñar un sistema de control que guíe a un objeto, máquinas o robot para que siga una trayectoria dada \cite{25}. Es comúnmente utilizado en aplicaciones de robótica, sistemas de control de vehículos, brazos mecánicos, y particularmente para este proyecto, busca ser aplicado en la ortesis robótica. A través del seguimiento de trayectoria el sistema alcanza una serie de puntos, o trayectoria cartesiana, minimizando el error entre la posición deseada y la posición real. Para lograr esto, se utilizan controladores como el control PID (Proporcional Integral Derivativo) \cite{26}. 

\subsection{Marco procedimental}
\subsubsection{Metodología mecatrónica}
La metodología seleccionada para el desarrollo del proyecto es la metodología VDI 2206, la cual es una guía flexible diseñada específicamente para el desarrollo de sistemas mecatrónicos, que integra disciplinas como la mecánica, electrónica, control y tecnologías de la información, ayudando a gestionar la complejidad y heterogeneidad de diseños mecatrónicos a través de un modelo adaptable a las necesidades del proyecto \cite{30}. 
La metodología consta de un diseño en dos niveles:
\begin{itemize}
	\item Micro nivel: Centrado en el proceso de resolución de problemas a nivel individual, apoyando en tareas específicas del diseño.
	\item Macro nivel: Utiliza un modelo en “V”, que combina un enfoque de arriba hacia abajo para el diseño del sistema (descomponiendo en funciones), y de abajo hacia arriba para la integración del sistema, lo que permite la validación y verificación continua.
\end{itemize}

\subsubsection{Implementación de metodología VDI-2206}
En la Fig. \ref{fig:P02_DiagramaVDI2206} se muestra el diagrama de modelo VDI-2206 con etapas enfocadas en el desarrollo de la ortesis robótica.
\begin{figure}[h!]
	\centering
	\includegraphics [trim = 0 0cm 0 0, clip, width=15cm]{figure/P01_VDI.png}
	\caption[Diagrama de modelo del VDI-2206.]{Diagrama de modelo del VDI-2206 con etapas para el desarrollo de la ortesis robótica.\label{fig:P02_DiagramaVDI2206}}
\end{figure}

\subsubsection{Esquema FBS}
El esquema FBS (Functional Breakdown Structure) es un enfoque de descomposición funcional que organiza todas las actividades necesarias para cumplir una función global, separándolas de una estructura basada en productos, como la Work Breakdown Structure (WBS). A diferencia de la WBS, la FBS se centra en los procesos y funciones requeridas para alcanzar los objetivos de una arquitectura sin estar ligada a una implementación específica. Este enfoque permite evaluar que tan completos son los diseños y optimizar la integración de disciplinas desde una perspectiva holística, evitando la segmentación en niveles individuales de componentes. Además, la FBS ayuda a establecer comparaciones más precisas entre opciones arquitectónicas al identificar funciones redundantes o faltantes \cite{27}.

\subsubsection{IDEF0}
IDEF0 (Integration Definition for Function Modeling) es una metodología de modelado de procesos utilizada para representar gráficamente funciones dentro de un sistema, facilitando su análisis y optimización. En el contexto de la mecatrónica, IDEF0 permite estructurar la interacción entre componentes mecánicos, electrónicos y de control, en la búsqueda de una integración eficiente de los subsistemas. Su enfoque jerárquico y modular ayuda a definir entradas, controles, mecanismos y salidas de cada función \cite{28}.