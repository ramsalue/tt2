\subsection{Integración de sistemas de ortesis} \label{Integración de sistemas}
Después de haber desarrollado y detallado los diversos sistemas y módulos que componen la ortesis robótica, es fundamental visualizar cómo estos elementos se integran para formar el dispositivo completo. Esta sección presenta la interconexión de los componentes electrónicos y de control, así como el ensamble físico de la estructura y los mecanismos.
\begin{figure}[h!]
	\centering
	\includegraphics[width=0.95\textwidth]{figure/S7_control/Integracion_Comp.jpg}
	\caption[Diagrama de Integración de Componentes Electrónicos y de Control.]{Diagrama de Integración de Componentes Electrónicos y de Control.\label{fig:int_comp}}
\end{figure}

La integración de los sistemas electrónicos y de control se visualiza en un diagrama que muestra las conexiones entre la unidad de procesamiento principal, los módulos de sensado, los drivers de motor, los actuadores, la interfaz humano-máquina y el almacenamiento. La Fig. \ref{fig:int_comp} ilustra este esquema de interconexión. En este diagrama se observa cómo la Raspberry Pi actúa como eje central, recibiendo información de los sensores óptico y limit switch, interactuando con la pantalla Waveshare para la interfaz con el usuario, y enviando señales de comando a los drivers HSS86 y HSS57 para controlar los motores paso a paso Nema 34 y Nema 23. La tarjeta Micro-SD, insertada en la Raspberry Pi, es el medio de almacenamiento principal para el sistema operativo y los datos. Este arreglo electrónico y de control es la base para ejecutar la lógica de movimiento y las rutinas de rehabilitación.



A nivel físico, la integración de los sistemas se materializa en el ensamble de la ortesis. Después de tener todos los sistemas diseñados a detalle, se realizó un modelo tridimensional en SolidWorks para visualizar la disposición física de los componentes y validar su integración espacial. La Fig. \ref{fig:int_assbly} muestra una vista donde se pueden observar los mecanismos de movimiento (flexión-extensión y abducción-aducción) integrados a la estructura de la cama, así como los soportes para la pierna, los topes mecánicos y los puntos de apoyo como las chumaceras. De igual manera, se identifica el espacio destinado para colocar la caja de componentes electrónicos, la ubicación de la interfaz HMI y la ubicación del botón de paro de emergencia.

\begin{figure}[h!]
	\centering
	\includegraphics[width=1\textwidth]{figure/S7_control/Integracion_ASSBLY.jpg}
	\caption[Ensamble de la Ortesis Robótica en SolidWorks.]{Ensamble de la Ortesis Robótica en SolidWorks.\label{fig:int_assbly}}
\end{figure}

Este ensamble en SolidWorks nos ofrece una visión clara de la estructura general de la ortesis robótica y de cómo se integran físicamente los diferentes sistemas que la componen. Además de ser una herramienta de validación del diseño, sirve como referencia detallada para poder implementar la ortesis físicamente en la siguiente fase del proyecto (Trabajo Terminal II).

\subsection{Plan de pruebas y validación}
\label{sec:plan_pruebas}

El objetivo de las pruebas es verificar el funcionamiento integral del sistema,
incluyendo los subsistemas eléctrico, mecánico y de control, buscando
la seguridad del usuario.

\subsubsection{Entorno de prueba}
\begin{itemize}
	\item Laboratorio de Trabajo Terminal en instalaciones de la unidad académica UPIITA, así como en taller donde se realizará parte de la construcción del prototipo.
	\item Alimentación eléctrica de 120 V~CA regulada.
\end{itemize}

\subsubsection{Pruebas a realizar}
\begin{enumerate}
	\item \textbf{Prueba de movimiento completo:} verificar el rango de flexión y extensión del
	actuador lineal y rotativo en ciclos repetidos.
	\item \textbf{Prueba de respuesta del paro de emergencia:} verificar que la etapa de potencia se desconecta al presionar el botón de emergencia.
	\item \textbf{Prueba de carga máxima:} aplicar un peso de 90 kg en la estructura para comprobar la resistencia y detectar deformaciones.
	\item \textbf{Prueba de protecciones eléctricas:} poner en marcha motores y medir corriente para verificar que el consumo se encuentra dentro del rango esperado.
	\item \textbf{Prueba de comunicación y control:} evaluar la interfaz táctil y el microcontrolador con los motores NEMA 23 y 34, verificando que existe comunicación entre ambos al ingresar instrucciones en la pantalla táctil para ser transmitidas al microcontrolador y éste a su vez a los actuadores.
\end{enumerate}

\subsubsection{Criterios de aceptación}
\begin{itemize}
	\item Los movimientos y repeticiones realizados por la ortesis sean los que se ingresan a través de la pantalla táctil.
	\item Corriente máxima en actuadores dentro del rango de 8 A pico.
	\item Ortesis soporte el peso del sujeto de prueba\footnote{Se contempla como sujeto de prueba a integrante del equipo que se encuentre dentro de las dimensiones establecidas para el usuario}.
\end{itemize}

Los resultados de cada prueba se registrarán en una bitácora técnica con
fotografías, gráficas y conclusiones para validación final del prototipo.

