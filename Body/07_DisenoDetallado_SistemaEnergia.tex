\clearpage
\subsubsection{S4. Sistema de energía}\label{S4. Sistema de energía}
Para diseñar el sistema de energía se tomó en consideración todos los dispositivos y componentes que consumen energía eléctrica dentro de cada componente, se consideraron sus especificaciones técnicas, para considerar su etapa de potencia y acondicionamiento de energía. Además, con base en la distribución de energía eléctrica se establece la mejor posición para posicionar y hacer efectivo un paro de emergencia.
\paragraph{M5. Módulo de paro de emergencia}\mbox{}\\
El módulo de paro de emergencia permite la desconexión inmediata de la alimentación en caso de una situación de riesgo para el paciente o el equipo. Con base en nuestro concepto solución se propone para este módulo el uso de un botón normalmente cerrado (NC), que al ser presionado interrumpe el paso de corriente hacia la bobina de un relevador de seguridad, deteniendo el funcionamiento del sistema.\\

El módulo de paro debe cumplir las siguientes funciones:
\begin{itemize}
	\item Interrumpir inmediatamente la energía de potencia a los actuadores del sistema (S5) ante una pulsación del botón.
	\item Retener el estado de paro hasta que se realice un rearme manual seguro.
	\item Emitir una señal de diagnóstico al sistema de control (S7) para el registro del evento.
\end{itemize}

Para obtener la conmutación del sistema de energía a apagado se propone el uso de un relevador posicionado entre la fuente de tensión y los actuadores. En la Fig. \ref{fig:paro_emergencia} se muestra el esquema de conexión para posición del módulo de paro de emergencia. La elección de un botón normalmente cerrado está relacionado con las normas ISO 13850:2015 e IEC 60204-1:2018, las cuales indican que el botón de paro de emergencia debe actuar por apertura de contacto (NC), y su activación debe cortar la energía de manera directa \cite{35}.

\begin{figure}[h!]
	\centering
	\includegraphics[width=0.54\textwidth]{figure/s4_energía/diagrama_paro_emergencia.png}
	\caption[Diagrama de conexión para el módulo de paro de emergencia (M5).]{Diagrama de conexión para el módulo de paro de emergencia (M5).}
	\label{fig:paro_emergencia}
\end{figure}
%https://articulo.mercadolibre.com.mx/MLM-1460436346-paro-de-emergencia-con-caja-amarilla-_JM#polycard_client=search-nordic&position=7&search_layout=grid&type=item&tracking_id=197af8db-fa7e-4862-9139-640489def5b1&wid=MLM1460436346&sid=search
%https://sieeg.com.mx/producto/modulo-rele-1-canal-a-5v-10a-low-level-relay/?srsltid=AfmBOoqWVT-qG8RPXjIHprqa2cuZwyFcEMMtZDoWk0_yPYwQCf59j_Xu

El botón normalmente cerrado que se propone cuenta con una base amarilla y el botón rojo, lo cual permite su fácil identificación en el sistema al maximizar el contraste entre los colores, además de contar con un mecanismo de liberación que requiera un giro para ser liberado, y debido a que se trata de un botón, el paro de emergencia requeriría de un solo movimiento humano para poder activarlo. Esta selección se fundamenta en la norma ISO 13850:2015, la cual establece los principios y requisitos funcionales para el diseño y la implementación de la función de parada de emergencia en mecanismos, sin importar el tipo de energía utilizada \cite{36}. En la Fig. \ref{fig:boton_paroemergencia} se muestra el paro de emergencia propuesto. 
\begin{figure}[h!]
	\centering
	\includegraphics[width=0.6\textwidth]{figure/s4_energía/boton_paro_emergencia.png}
	\caption[Botón de paro de emergencia.]{Boton Paro De Emergencia Con caja empotrable NC Modelo TP-BBM1C.En la imagen izquierda se muestran los colores del botón, mientras que en la imagen derecha se tienen las dimensiones del mismo.}
	\label{fig:boton_paroemergencia}
\end{figure}
%https://articulo.mercadolibre.com.mx/MLM-1480054772-boton-paro-de-emergencia-con-caja-empotrable-1nc-tp-bbm1c-_JM

Sobre el relevador de seguridad se propone el módulo relé relevador de 1 canal a 5V y 10A, el cual se observa en la Fig. \ref{fig:modulo_relevador}. Para proyectos de electrónica y automatización que requieren la gestión de cargas de alta potencia con un control de bajo voltaje, el módulo relé de 1 canal a 5V es una opción eficaz. Este dispositivo electromagnético opera como un interruptor accionado eléctricamente, donde una bobina crea un campo magnético para manipular los contactos, facilitando la conmutación de circuitos eléctricos distintos.
\begin{figure}[h!]
	\centering
	\subcaptionbox{Partes del módulo y dimensiones.}
	{\includegraphics[width=0.4\textwidth]{figure/s4_energía/modulo_relevador.png}}
	\subcaptionbox{Diagrama del relevador.}
	{\includegraphics[width=0.5\textwidth]{figure/s4_energía/relevador.jpg}}
	\caption[Módulo relé-relevador de 1 canal a 5V y 10A.]{Módulo relé-relevador de 1 canal a 5V y 10A.}
	\label{fig:modulo_relevador}
\end{figure}

El componente elegido tiene dentro de sus ventajas se distingue por su compatibilidad con una amplia gama de microcontroladores y sistemas electrónicos, lo que simplifica su implementación en diversos proyectos. Ofrece una alta capacidad de conmutación, siendo capaz de manejar dispositivos con consumos de hasta 10 amperios y voltajes de hasta 250 voltios de corriente alterna. Para su supervisión incorpora un indicador visual LED que permite identificar el estado operativo del relé. 

\paragraph{M6. Módulo de etapa de potencia}\mbox{}\\
El módulo de etapa de potencia tiene como función principal suministrar la energía necesaria para el funcionamiento de los actuadores principales del sistema, que en este caso están conformados por dos motores paso a paso: un motor NEMA 34 y un motor NEMA 23. Ambos motores requieren un control, por lo que se emplean drivers específicos para cada uno, el modelo HSS86 para el NEMA 34 y el HSS57 para el NEMA 23 \footnote{La selección de estos motores y drivers se detalla en la subsección \nameref{S5_Sistema de movimiento} y \nameref{S7_Sistema de control}.}.

Para alimentar esta etapa se contempla el uso de una fuente conmutada, la cual ofrece ventajas como mayor eficiencia energética, menor peso y menor tamaño en comparación con fuentes lineales. La fuente deberá ser capaz de entregar una tensión y corriente adecuadas para ambos motores. En este caso, considerando las especificaciones de los drivers y la demanda de corriente pico de hasta 8A, se recomienda seleccionar una fuente conmutada de al menos 48V y 10A (480W), con margen de seguridad del 20\% para evitar sobrecargas y garantizar un funcionamiento continuo y confiable. 

Es importante que esta fuente cuente con protección contra sobrecorriente, sobrevoltaje y sobrecalentamiento. Se sugiere también que incluya una buena regulación de voltaje y bajo rizado para evitar interferencias en la operación de los drivers.

\begin{itemize}
	\item \textbf{Motor NEMA 34} – Alimentado a través del driver HSS86.
	\item \textbf{Motor NEMA 23} – Alimentado mediante el driver HSS57.
	\item \textbf{Fuente conmutada} – 48V, mínimo 10A, con protecciones integradas.
\end{itemize}

El diseño del cableado de esta etapa debe considerar calibres adecuados de conductores, conectores robustos, disipación térmica suficiente para los drivers, y la correcta separación de líneas de potencia y señal para evitar acoplamientos indeseados.\\
Para cumplir con estos requerimientos se propone el uso de la fuente conmutada comercial LRS-600-48 [Fig. \ref{fig:fuente_conmutada}].
\begin{figure}[h!]
	\centering
	\includegraphics[angle =0, width=0.45\textwidth]{figure/s4_energía/fuente_conmutada.png}
	\caption[Fuente conmutada MEAN WELL LRS-600-48.]{Fuente conmutada MEAN WELL LRS-600-48. Las dimensiones de la fuente se encuentran en el anexo \nameref{Anexo_sistema de energia}, Fig. \ref{fig:ds_fuente_conmutada_dimensiones}.}
	\label{fig:fuente_conmutada}
\end{figure}
%https://www.meanwellshop.mx/shop/lrs-600-48-lrs-600-48-1580?srsltid=AfmBOoqqdPFpOqWXtVKLjRkz8Ju6RSBeIW2y7aexAoHGZEGHl9_jKR1l#attribute_values=1

La fuente de alimentación MEAN WELL LRS-600-48 está diseñada para proporcionar una salida de 48 VDC. Esta tensión se encuentra dentro del rango de operación recomendado para ambos drivers (24-80 VDC para el HSS86 y 20-50 VDC para el HSS57). Utilizar una tensión cercana al límite superior del rango de operación del driver HSS57 (50 VDC) y bien dentro del rango del HSS86 (80 VDC) permite que los motores operen con el par y la velocidad máximos posibles, aprovechando sus capacidades sin exceder las especificaciones de los drivers.\\ En la Fig. \ref{fig:diagrama_fuente_conmutada} se presenta el diagrama de bloques de la fuente MEAN WELL LRS-600-48.
\begin{figure}[h!]
	\centering
	\includegraphics[angle =0, width=1\textwidth]{figure/s4_energía/diagrama_fuente_comnutada.png}
	\caption[Diagrama de bloques de fuente MEAN WELL LRS-600-48.]{Diagrama de bloques de fuente MEAN WELL LRS-600-48.}
	\label{fig:diagrama_fuente_conmutada}
\end{figure}

Sobre la corriente, este es un factor importante, para asegurar que los motores reciban la potencia necesaria para funcionar correctamente, especialmente bajo carga o durante arranques.
\begin{itemize}
	\item El driver HSS86 (para NEMA 34) puede requerir una corriente pico de hasta 8 A.
	\item El driver HSS57 (para NEMA 23) puede requerir una corriente pico de hasta 5 A.
\end{itemize}
Considerando las necesidades de corriente máxima de ambos drivers, la corriente combinada requerida sería de: $$8\ A+5\ A=13\ A$$ La fuente MEAN WELL LRS-600-48 posee una corriente nominal de salida de 12.5 A. Aunque la suma directa de las corrientes pico de los drivers (13A) es ligeramente superior a la corriente nominal de la fuente (12.5A), es fundamental considerar el comportamiento real de los motores y drivers, y es que, con base en la programación detallada en la subsección \nameref{S6_HMI}, ambos motores no van a operar simultáneamente, por lo que la demanda máxima de corriente de la fuente estará dictada por el driver HSS86, con un pico de hasta 8 A.
La fuente MEAN WELL LRS-600-48 provee una corriente nominal de salida de 12.5 A. Al comparar esta capacidad con el requisito máximo de 8 A (para el HSS86), se obtiene que la fuente LRS-600-48 posee un amplio margen de corriente disponible, incluso aplicando un margen de seguridad del 20\% sobre el consumo pico del driver más grand:$$8\ A\times1.20 = 9.6\ A$$ La fuente de 12.5 A tiene una margen significativo de aproximadamente 2.9 A, lo que asegura mayor fiabilidad bajo las condiciones operativas más demandantes del motor individual. Este margen es importante para absorber picos de corriente transitorios durante el arranque o cambios rápidos de dirección del motor sin provocar caídas de tensión que puedan afectar la estabilidad del sistema durante la ejecución de los movimientos.

\paragraph{M7. Módulo de acondicionamiento de energía}\mbox{}\\

El módulo de acondicionamiento de energía se encarga de proporcionar una distribución adecuada y segura de las distintas tensiones requeridas por los componentes electrónicos y de control del sistema. A diferencia del módulo de etapa de potencia, este módulo opera con cargas de baja corriente y requiere tensiones estabilizadas y con bajo nivel de ruido.

El sistema cuenta con una Raspberry Pi 4 como unidad de procesamiento central, la cual requiere una alimentación estable de 5 V y al menos 3 A a través de conector USB-C. Adicionalmente, se alimentan dispositivos como una pantalla táctil Waveshare de 7", que también opera a 5 V, así como el sensor óptico VL5310X y un módulo relé de 1 canal (5 V / 10 A), junto con su respectivo botón de paro de emergencia normalmente cerrado:

\begin{itemize}
	\item Raspberry Pi 4: 5 V / 3 A.
	\item Pantalla Táctil Waveshare de 7": 5 V (consumo típico < 1 A).
	\item Sensor Óptico VL5310X: 5 V (bajo consumo en mA).
	\item Módulo Relé de 1 Canal: 5 V (la bobina consume mA, el 10 A es para los contactos).
	\item Botón de Paro de Emergencia: No requiere alimentación activa.
\end{itemize}

Para cubrir estos requerimientos, se propone utilizar una segunda fuente conmutada independiente de 5V y al menos 5A, preferentemente con salidas múltiples o borneras de fácil conexión. Esta fuente será la encargada de alimentar el microcontrolador, sensores, la interfaz HMI y los elementos de seguridad de baja potencia. Para satisfacer estos requerimientos se propone el uso de la fuente conmutada comercial MEAN WELL LRS-50-5 [Fig. \ref{fig:fuente_conmutada_5V}].

\begin{figure}[h!]
	\centering
	\includegraphics[angle =0, width=0.45\textwidth]{figure/s4_energía/fuente_conmutada_5V.png}
	\caption[Fuente conmutada MEAN WELL LRS-50-5.]{Fuente conmutada MEAN WELL LRS-50-5. Las dimensiones de la fuente se encuentran en el anexo \nameref{Anexo_sistema de energia}, Fig. \ref{fig:ds_fuente_conmutada_5V_dimensiones}.}
	\label{fig:fuente_conmutada_5V}
\end{figure}

La fuente MEAN WELL LRS-50-5 está diseñada para proporcionar una salida estable y regulada de 5 VDC, eliminando la necesidad de reguladores de voltaje adicionales para cada componente. Esto simplifica el diseño, reduce la complejidad del circuito y mejora la fiabilidad general. En la Fig. \ref{fig:diagrama_fuente_conmutada5V} se presenta el diagrama de bloques de la fuente MEAN WELL LRS-50-5.
\begin{figure}[h!]
	\centering
	\includegraphics[angle =0, width=1\textwidth]{figure/s4_energía/diagrama_fuente_comnutada5v.png}
	\caption[Diagrama de bloques de fuente MEAN WELL LRS-50-5.]{Diagrama de bloques de fuente MEAN WELL LRS-50-5.}
	\label{fig:diagrama_fuente_conmutada5V}
\end{figure}

El requerimiento definido de corriente para el módulo es de un mínimo de 5 A. La fuente MEAN WELL LRS-50-5 tiene una capacidad nominal de salida de 10 A. Esta capacidad de 10 A es el doble del requisito mínimo de 5 A y proporciona un margen de seguridad extremadamente alto para el sistema, lo cual también permite que se pueden integrar futuros componentes o funcionalidades que puedan aumentar el consumo sin necesidad de reemplazar la fuente de alimentación.

Otra de las ventajas de usar esta fuente es que, aunque las fuentes conmutadas generan cierto nivel de ruido inherente, los diseños de MEAN WELL están optimizados para minimizarlo. La hoja de datos del LRS-50-5 especifica un rizado y ruido de 80mVp-p para la salida de 5V. 