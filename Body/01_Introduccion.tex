% CREATED BY MAGNUS GUSTAVER, 2020
\section*{Introducción}
\addcontentsline{toc}{section}{Introducción}
La hemiplejia es una parálisis que afecta un lado del cuerpo, generalmente causada por un accidente cerebrovascular (ACV), también llamado ICTUS , lo que impacta significativamente la función motora y puede provocar déficits sensoriales, cognitivos y de coordinación, dependiendo de la gravedad y localización del daño cerebral \cite{01}. Según datos del Instituto Nacional de Estadística y Geografía (INEGI) del año 2022, las enfermedades cerebrovasculares ocuparon el sexto lugar entre las principales causas de muerte en México, con un total de 18,632 personas fallecidas por esta causa \cite{02}.\\

En casos de hemiplejia por accidente cerebrovascular, la fisioterapia es esencial para evitar la atrofia muscular de la movilidad en la zona afectada, así como para evitar complicaciones adicionales, como contracturas o desvíos posturales. Uno de los principales riesgos asociados con la inmovilidad prolongada es el desarrollo del síndrome de inmovilidad, que deteriora la capacidad del paciente para interactuar con su entorno, generando dependencia de otras personas u objetos para realizar actividades cotidianas \cite{03}. La fisioterapia, por lo tanto, juega un papel crucial en la reducción de las secuelas, como caídas y complicaciones musculoesqueléticas, cardiovasculares y, especialmente, neurológicas.\\

Durante el proceso de rehabilitación intervienen dos actores principales: el paciente, quien padece las consecuencias de la hemiplejia, y el fisioterapeuta, encargado de llevar a cabo el tratamiento. Aunque se espera que el fisioterapeuta realice su labor con la máxima eficiencia, siendo humano, puede enfrentar limitaciones físicas y fatiga al asistir continuamente a los pacientes. Por ello, surge la necesidad de herramientas complementarias que puedan optimizar este proceso.\\

Este enfoque no solo mejorará la calidad de vida de los pacientes al apoyarlos en su proceso de rehabilitación, sino que también beneficiará a los profesionales de la salud, permitiéndoles ofrecer una atención más efectiva. En particular, este proyecto se enfoca en la optimización del proceso de rehabilitación de la extremidad inferior derecha.\\

La importancia de la robótica en la rehabilitación es el aumento de intensidad y frecuencia de la terapia, de este modo se fomenta la neuroplasticidad, donde la capacidad del cerebro se adapta a nuevos ambientes por medio de estimulaciones sensoriales \cite{03_01}. Realizando movimientos repetitivos continuos, los sistemas robóticos ayudan a mejorar la fuerza, resistencia y equilibrio de los pacientes, aumentando la motivación y esperanza de recuperación. Una de las ventajas de la robótica en la fisioterapia es apoyar el tratamiento convencional con un tratamiento asistido, haciendo mejoría de las condiciones osteoarticulares.\\

La terapia robótica aporta beneficios a las secuelas de la enfermedad cerebrovascular, los que la han recibido se muestran más alegres, optimistas, realizan un mejor análisis de la actividad y su secuencia, mostrando mayor rapidez en actividades cognitivas y motoras \cite{03_02}.

\subsection*{Enfoque mecatrónico}
\addcontentsline{toc}{subsection}{Enfoque mecatrónico}

El presente proyecto se contempla desde un enfoque mecatrónico debido a que integra conocimientos de múltiples dominios, los cuales van desde la mecánica (sujeción del paciente, transmisión de movimiento, estructura), electrónica (etapa de potencia para distribución de energía, adquisición de señales, activación de actuadores) y el uso de software para programación y comunicación del sistema con el usuario (interfaz humano-máquina, lógica de control) con el fin de lograr el diseño de la ortesis robótica integrada en una cama que permita de forma controlada realizar ejercicios de rehabilitación para la extremidad inferior derecha mediante flexión, extensión, abducción y aducción, con monitoreo, control programado, y protección ante posibles fallos.\\

La sinergia de estos dominios permite que las instrucciones que un usuario declare a través de la interfaz humano-máquina y la lógica de control sean llevadas a la estructura mecánica a través de los elementos y componentes que conformen al dominio electrónico. Para realizar esta tarea se cuenta con los siguientes sistemas:

\begin{enumerate}
	\item Sistema estructural, cuya función general es soportar y ajustar posición del paciente.
	\item Sistema de seguridad eléctrico, empleado para proteger al sistema contra sobrecorrientes y sobrevoltajes.
	\item	Sistema de seguridad mecánico que limita físicamente el movimiento de la ortesis más allá de rangos establecidos.
	\item	Sistema de energía que controla los flujos de energía a los diversos sistemas, así como paro de emergencia.
	\item Sistema de movimiento encargado de ejecutar los movimientos articulares mediante actuadores controlados.
	\item Sistema de comunicación humano-máquina que permite la interacción directa con el usuario, en este caso, el fisioterapeuta. El intercambio de información es bidireccional. 
	\item Sistema de control en el cual se encuentra la ley de control que gobierna al sistema, sensado, y procesamiento de datos de control para poder realizar los movimientos que el usuario declare desde la interfaz humano-máquina.
\end{enumerate}

El enfoque mecatrónico adoptado para el desarrollo de esta ortesis robótica se justifica en la necesidad de integrar los distintos dominios del sistema: mecánico, electrónico y de software. Se optó por una arquitectura modular, en la que cada sistema (estructura, energía, control, movimiento, seguridad y comunicación) responde a funciones específicas definidas, buscando de esta forma un diseño escalable, mantenible y que permita su validación por etapas. Además, la integración entre sensores, microcontroladores y actuadores posibilita una operación sincronizada y precisa, importante para lograr movimientos terapéuticos controlados. Se incorporaron mecanismos de seguridad tanto eléctricos como mecánicos, y a esto se suma la capacidad de personalizar rutinas y parámetros de sesión de acuerdo con las condiciones del paciente. Esta combinación de robustez, adaptabilidad y control hace que el enfoque mecatrónico sea apropiado para cumplir con los objetivos del proyecto.

