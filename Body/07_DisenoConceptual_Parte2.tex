\subsubsection{Propuestas solución}
Para poder desarrollar un diseño conceptual se plantearon posibles soluciones para los sistemas y módulos establecidos en la arquitectura física de la Fig. \ref{fig:Arquitectura_Fisica}. Para esto se desarrolló un diagrama morfológico por cada sistema donde se establecen alternativas para sus principales módulos. 
\paragraph*{Sistema estructural\\}
El sistema estructural consta de la cama que soporta a la ortesis robótica como a los demás sistemas de energía, control, seguridad eléctrica y mecánico, así como de comunicación HMI. Además, es el sistema sobre el cual el paciente es posicionado. En la tabla \ref{dm_S1} se tiene el diagrama morfológico de este sistema.
\begin{table}[h!]
	\centering
	\caption{Diagrama morfológico del sistema estructural\label{dm_S1}.}
	\begin{tabular}{|c|c|c|}
		\hline
		\textbf{Característica} & \textbf{Alternativa 1} & \textbf{Alternativa 2}   \\ \hline
		Plataforma base         & Cama metálica fija     & Cama metálica con ruedas \\ \hline
	\end{tabular}
\end{table}
\paragraph*{ Sistema de seguridad eléctrico\\}
Este sistema se conforma por módulos de sobreprotección contra corrientes y voltajes. En la tabla \ref{dm_S2} se muestra su análisis morfológico que incluye la alternativa más simple y efectiva encontrada para este sistema.
\begin{table}[h!]
	\centering
	\caption{Diagrama morfológico del sistema de seguridad eléctrico\label{dm_S2}.}
	\begin{tabular}{|c|c|}
		\hline
		\textbf{Característica}       & \textbf{Alternativa 1} \\ \hline
		Protección por sobrecorriente & Fusible rápido         \\ \hline
	\end{tabular}
\end{table}
\paragraph*{Sistema de seguridad mecánico\\}
Dentro del sistema de seguridad mecánico se contemplan dos módulos principales, los cuales se encuentran definidos por los requerimientos del proyecto presentados, por lo que su análisis morfológico cuenta con una alternativa, tal y como se observa en la tabla \ref{dm_S3}.
\begin{table}[h!]
	\centering
	\caption{Diagrama morfológico del sistema de seguridad mecánico.\label{dm_S3}}
	\begin{tabular}{|c|c|}
		\hline
		\textbf{Característica} & \textbf{Alternativa 1} \\ \hline
		Topes mecánicos         & Tope mecánico          \\ \hline
		Sujeción y ajuste       & Cintas de velcro       \\ \hline
	\end{tabular}
\end{table}

\paragraph*{Sistema de movimiento\\}
Es uno de los sistemas más importantes al ser el encargado de generar los movimientos que requiere el paciente posicionado durante sus sesiones. En la tabla \ref{dm_S5} se muestra el diagrama morfológico correspondiente a dicho sistema.
\begin{table}[h!]
	\centering
	\caption{Diagrama morfológico del sistema de movimiento.\label{dm_S5}}
	\begin{tabular}{|c|ccc|}
		\hline
		\textbf{Característica} & \multicolumn{1}{c|}{\textbf{Alternativa 1}} & \multicolumn{1}{c|}{\textbf{Alternativa 2}} & \textbf{Alternativa 3} \\ \hline
		Actuadores lineales     & \multicolumn{1}{c|}{Un motor lineal}        & \multicolumn{1}{c|}{Dos motores lineales}   & Actuador neumático     \\ \hline
		Actuador rotativo       & \multicolumn{3}{c|}{Un motor rotativo}                                                                             \\ \hline
	\end{tabular}
\end{table}

\paragraph*{Sistema de energía\\}
En la tabla \ref{dm_S4} se muestra el análisis morfológico correspondiente al sistema de energía, donde se plantean diversas alternativas para el paro de emergencia, etapa de potencia y acondicionamiento de energía. De este sistema dependen todos la mayoría de sistemas para poder operar.
\begin{table}[h!]
	\centering
	\caption{Diagrama morfológico del sistema de energía.\label{dm_S4}}
	\begin{tabular}{|c|c|cc|}
		\hline
		\textbf{Característica}      & \textbf{Alternativa 1} & \multicolumn{1}{c|}{\textbf{Alternativa 2}} & \textbf{Alternativa 3} \\ \hline
		Paro de emergencia           & Botón mecánico         & \multicolumn{1}{c|}{Interruptor de palanca} & Pedal                  \\ \hline
		Etapa de potencia            & Drivers de motores     & \multicolumn{2}{c|}{Puentes H}                                       \\ \hline
		\begin{tabular}[c]{@{}c@{}}Acondicionamiento\\ de energía\end{tabular} & Fuente conmutada       & \multicolumn{2}{c|}{Reguladores}                                     \\ \hline
	\end{tabular}
\end{table}

\paragraph*{Sistema de comunicación Humano–Máquina\\}
A través de este sistema el usuario puede comunicar sus instrucciones y también recibir información del mismo. Para este sistema se presenta el siguiente diagrama morfológico mostrado en la tabla \ref{dm_S6}.
\begin{table}[h!]
	\centering
	\caption{Diagrama morfológico del sistema de comunicación Humano–Máquina.\label{dm_S6}}
	\begin{tabular}{|c|ccc|}
		\hline
		\textbf{Característica} & \multicolumn{1}{c|}{\textbf{Alternativa 1}} & \multicolumn{1}{c|}{\textbf{Alternativa 2}} & \textbf{Alternativa 3} \\ \hline
		Módulo entrada/salida   & \multicolumn{1}{c|}{Pantalla táctil}        & \multicolumn{1}{c|}{Botonera con display}   & Computadora            \\ \hline
		Almacenamiento          & \multicolumn{3}{c|}{Memoria interna MC}                                                                            \\ \hline
	\end{tabular}
\end{table}
\paragraph*{Sistema de control\\}
En la tabla \ref{dm_S7} se muestra el diagrama morfológico del sistema de control.
\begin{table}[h!]
	\centering
	\caption{Diagrama morfológico del sistema de control.\label{dm_S7}}
	\begin{tabular}{|c|cc|}
		\hline
		\textbf{Característica}                                                                 & \multicolumn{1}{c|}{\textbf{Alternativa 1}}                                                                                                     & \textbf{Alternativa 2}                                                                       \\ \hline
		Sensores                                                                                & \multicolumn{1}{c|}{\begin{tabular}[c]{@{}c@{}}Encoder, \\ sensores de distancia, \\ potenciómetros lineales\end{tabular}} & \begin{tabular}[c]{@{}c@{}}Encoder, \\ sensor de distancia\end{tabular} \\ \hline
		\multirow{2}{*}{\begin{tabular}[c]{@{}c@{}}Acondicionamiento\\ de señales\end{tabular}} & \multicolumn{2}{c|}{Filtros digitales}                                                                                                                                                                                                         \\ \cline{2-3} 
		& \multicolumn{2}{c|}{Amplificador operacional}                                                                                                                                                                                                  \\ \hline
		Procesamiento                                                                           & \multicolumn{1}{c|}{Raspberry}                                                                                                                  & Arduino                                                                                      \\ \hline
	\end{tabular}
\end{table}

\subsubsection{Propuestas de diseño conceptual}
Con base en las propuestas solución planteadas se realizaron diseños conceptuales tomando en cuenta los requerimientos del proyecto. En la tabla \ref{conceptos_solución} se muestran los dos enfoques identificados como posibles soluciones. 
\begin{table}[h!]
	\centering
	\caption{Conceptos solución generados. \label{conceptos_solución}}
	\begin{tabular}{|ccc|cc|}
		\hline
		\multicolumn{3}{|c|}{\textbf{Característica}}                                                                                                    & \multicolumn{1}{c|}{\textbf{Concepto solución 1}}                                                                        & \textbf{Concepto solución 2}                                            \\ \hline
		\multicolumn{1}{|c|}{S1}                  & \multicolumn{1}{c|}{C1}  & Plataforma base                                                           & \multicolumn{1}{c|}{Cama metálica fija}                                                                                  & Cama metálica con ruedas                                                \\ \hline
		\multicolumn{1}{|c|}{S2}                  & \multicolumn{1}{c|}{C2}  & \begin{tabular}[c]{@{}c@{}}Protección por\\  sobrecorriente\end{tabular} & \multicolumn{2}{c|}{Fusible rápido}                                                                                                                                                                \\ \hline
		\multicolumn{1}{|c|}{\multirow{2}{*}{S3}} & \multicolumn{1}{c|}{C3}  & Topes mecánicos                                                           & \multicolumn{2}{c|}{Tope mecánico}                                                                                                                                                                 \\ \cline{2-5} 
		\multicolumn{1}{|c|}{}                    & \multicolumn{1}{c|}{C4}  & Sujeción y ajuste                                                         & \multicolumn{2}{c|}{Cintas de velcro}                                                                                                                                                              \\ \hline
		\multicolumn{1}{|c|}{\multirow{3}{*}{S4}} & \multicolumn{1}{c|}{C5}  & Paro de emergencia                                                        & \multicolumn{1}{c|}{Botón mecánico}                                                                                      & Pedal                                                                   \\ \cline{2-5} 
		\multicolumn{1}{|c|}{}                    & \multicolumn{1}{c|}{C6}  & Etapa de potencia                                                         & \multicolumn{1}{c|}{Drivers de motores}                                                                                  & Puentes H                                                               \\ \cline{2-5} 
		\multicolumn{1}{|c|}{}                    & \multicolumn{1}{c|}{C7}  & \begin{tabular}[c]{@{}c@{}}Acondicionamiento\\ de energía\end{tabular}    & \multicolumn{1}{c|}{Fuente conmutada}                                                                                    & Reguladores                                                             \\ \hline
		\multicolumn{1}{|c|}{\multirow{2}{*}{S5}} & \multicolumn{1}{c|}{C8}  & Actuadores lineales                                                       & \multicolumn{1}{c|}{Un motor lineal}                                                                                     & Dos motores lineales                                                    \\ \cline{2-5} 
		\multicolumn{1}{|c|}{}                    & \multicolumn{1}{c|}{C9}  & Actuador rotativo                                                         & \multicolumn{2}{c|}{Un motor rotativo}                                                                                                                                                             \\ \hline
		\multicolumn{1}{|c|}{\multirow{2}{*}{S6}} & \multicolumn{1}{c|}{C10} & Módulo entrada/salida                                                     & \multicolumn{1}{c|}{Pantalla táctil}                                                                                     & Botonera con display                                                    \\ \cline{2-5} 
		\multicolumn{1}{|c|}{}                    & \multicolumn{1}{c|}{C11} & Almacenamiento                                                            & \multicolumn{2}{c|}{Memoria interna MC}                                                                                                                                                            \\ \hline
		\multicolumn{1}{|c|}{\multirow{2}{*}{S7}} & \multicolumn{1}{c|}{C12} & Sensores                                                                  & \multicolumn{1}{c|}{\begin{tabular}[c]{@{}c@{}}Encoder,\\ sensores de distancia\end{tabular}} & \begin{tabular}[c]{@{}c@{}}Encoder, \\ sensor de distancia\\potenciómetros\end{tabular} \\ \cline{2-5} 
		\multicolumn{1}{|c|}{}                    & \multicolumn{1}{c|}{C13} & Procesamiento                                                             & \multicolumn{1}{c|}{Raspberry}                                                                                           & Arduino                                                                 \\ \hline
	\end{tabular}
\end{table}

\paragraph{Concepto solución 1.}
El diseño del primer concepto contempla una cama metálica cuyo soporte para la pierna afectada no tiene ruedas [Fig. \ref{fig:cs1}]. Se incluye el fusible rápido para protección por sobrecorriente, topes mecánicos y cintas de velcro para sujeción y ajuste. Para el paro de emergencia incluye un botón mecánico, su etapa de potencia se trabaja con drivers de motores, y su acondicionamiento de energía a través de una fuente conmutada. En cuanto a sus actuadores, se incluye  un motor lineal y un motor rotativo. Su módulo de entrada y salida se trabaja a través de una pantalla táctil y una tarjeta Raspberry Pi para recibir información de sensores como encoders y sensores de distancia [Fig. \ref{fig:cs1_hmi}].
\begin{figure}[h!]
	\centering
	\subcaptionbox{Estructura de la cama.\label{fig:cs_11}}
	{\includegraphics [trim = 0 0cm 0 0, clip,width=8cm]{figure/Imagenes conceptos/7.png}}
	\subcaptionbox{Soporte sin ruedas.\label{fig:cs_12}}
	{\includegraphics [trim = 0 0cm 0 0, clip,width=8cm]{figure/Imagenes conceptos/3.jpg}}
	\caption[Concepto solución 1 - Estructura.]{Concepto solución 1 - Estructura.\label{fig:cs1}}
\end{figure}
 \begin{figure}[h!]
 	\centering
 	\subcaptionbox{\label{fig:cs_13}}
 	{\includegraphics [angle = -90, trim = 0 0cm 0 0, clip,width=8cm]{figure/Imagenes conceptos/9.jpg}}
 	\subcaptionbox{\label{fig:cs_14}}
 	{\includegraphics [trim = 0 0cm 0 0, clip,width=8cm]{figure/Imagenes conceptos/10.jpg}}
 	\caption[Concepto solución 1 - HMI.]{Concepto solución 1 - HMI.\label{fig:cs1_hmi}}
 \end{figure}
  \paragraph{Concepto solución 2.}
  El diseño del segundo concepto contempla una cama metálica cuyo soporte para la pierna afectada sí tiene ruedas [Fig. \ref{fig:cs2}]. Se incluye el fusible rápido para protección por sobrecorriente, topes mecánicos y cintas de velcro para sujeción y ajuste. Para realizar paros de emergencia incluye un pedal, su etapa de potencia se trabaja a través de puentes H, y su acondicionamiento de energía se maneja con reguladores. Este concepto implica un mayor diseño tanto para los puentes H como para los reguladores que permitan trabajar la etapa de potencia y el acondicionamiento de energía. \\
  En cuanto a sus actuadores, se incluyen dos motores lineales y un motor rotativo. Su módulo de entrada y salida se trabaja a través de una botonera con display y microcontrolador Arduino para recibir información de sensores como encoders, potenciómetros y sensores de distancia [Fig. \ref{fig:cs2_hmi}].
  \begin{figure}[h!]
  	\centering
  	\subcaptionbox{Estructura de la cama.\label{fig:cs_21}}
  	{\includegraphics [trim = 0 0cm 0 0, clip,width=8cm]{figure/Imagenes conceptos/5 - Copy.jpg}}
  	\subcaptionbox{Soporte con ruedas.\label{fig:cs_22}}
  	{\includegraphics [trim = 0 0cm 0 0, clip,width=8cm]{figure/Imagenes conceptos/4.png}}
  	\caption[Concepto solución 2 - Estructura.]{Concepto solución 2 - Estructura.\label{fig:cs2}}
  \end{figure}
  \begin{figure}[h!]
  	\centering
  	\includegraphics [trim = 0 0cm 0 0, clip, width=8cm]{figure/Imagenes conceptos/11.jpg}
  	\caption[Concepto solución 2 - HMI.]{Concepto solución 2 - HMI.\label{fig:cs2_hmi}}
  \end{figure}
  
 
 
\subsubsection{Selección de diseño conceptual}
Para elegir uno de los diseños conceptuales se utilizaron tres matrices, primero la matriz de selección subjetiva, después la matriz binaria, y por último la matriz de ponderación. Estas fueron aplicadas para aquellas características en las que se tenían dos alternativas. 

\paragraph*{Sistema estructural\\}
Para el sistema estructural se realizó la matriz de selección subjetiva mostrada en la tabla \ref{sdc_01}, tomando en cuenta los criterios de modularidad, costo, facilidad de integración y disponibilidad.
\begin{table}[h!]
	\centering
	\caption{Matriz de selección subjetiva. S1 - Plataforma base.\label{sdc_01}}
	\begin{tabular}{|cc|cc|}
		\hline
		\multicolumn{2}{|c|}{\multirow{2}{*}{\textbf{Criterios}}} & \multicolumn{2}{c|}{\textbf{Conceptos}}                                              \\ \cline{3-4} 
		\multicolumn{2}{|c|}{}                                    & \multicolumn{1}{c|}{\textbf{Cama metálica fija}} & \textbf{Cama metálica con ruedas} \\ \hline
		\multicolumn{1}{|c|}{A}     & Modularidad                 & \multicolumn{1}{c|}{Regular}                     & Regular                           \\ \hline
		\multicolumn{1}{|c|}{B}     & Costo                       & \multicolumn{1}{c|}{Bueno}                       & Regular                           \\ \hline
		\multicolumn{1}{|c|}{C}     & Facilidad de integración    & \multicolumn{1}{c|}{Excelente}                   & Bueno                             \\ \hline
		\multicolumn{1}{|c|}{D}     & Disponibilidad              & \multicolumn{1}{c|}{Bueno}                       & Bueno                             \\ \hline
	\end{tabular}
\end{table}

Posteriormente, en la matriz binaria de la tabla \ref{sdc_02}, se jerarquizó la importancia relativa de los criterios.
\begin{table}[h!]
	\centering
	\caption{Matriz binaria. S1 - Plataforma base.\label{sdc_02}}
	\begin{tabular}{|cc|c|c|c|c|c|c|}
		\hline
		\multicolumn{2}{|c|}{\textbf{Criterios}}             & \textbf{A}               & \textbf{B}               & \textbf{C}               & \textbf{D}               & \textbf{Total} & \textbf{Rango} \\ \hline
		\multicolumn{1}{|c|}{A} & Modularidad                & \cellcolor[HTML]{BFBFBF} & 0                        & 0                        & 0                        & 0              & 1              \\ \hline
		\multicolumn{1}{|c|}{B} & Costo                      & 1                        & \cellcolor[HTML]{BFBFBF} & 1                        & 1                        & 3              & 2              \\ \hline
		\multicolumn{1}{|c|}{C} & Facilidad de   integración & 1                        & 0                        & \cellcolor[HTML]{BFBFBF} & 0                        & 1              & 3              \\ \hline
		\multicolumn{1}{|c|}{D} & Disponibilidad             & 1                        & 0                        & 1                        & \cellcolor[HTML]{BFBFBF} & 2              & 4              \\ \hline
	\end{tabular}
\end{table}

Finalmente, la tabla \ref{sdc_03} muestra la matriz de ponderación que determinó como alternativa preferible la cama metálica fija.
\begin{table}[h!]
	\centering
	\caption{Matriz de ponderación. S1 - Plataforma base.\label{sdc_03}}
	\begin{tabular}{cccc|c|c|}
		\hline
		\multicolumn{2}{|c|}{\textbf{Criterios}}                                  & \multicolumn{1}{c|}{\textbf{Total}} & \textbf{Ponderación} & \begin{tabular}[c]{@{}c@{}}\textbf{Cama}\\ \textbf{metálica fija}\end{tabular} & \begin{tabular}[c]{@{}c@{}}\textbf{Cama metálica}\\\textbf{con ruedas}\end{tabular} \\ \hline
		\multicolumn{1}{|c|}{B} & \multicolumn{1}{c|}{Costo}                      & \multicolumn{1}{c|}{3}              & 0.5                  & 4                             & 3                                   \\ \hline
		\multicolumn{1}{|c|}{D} & \multicolumn{1}{c|}{Disponibilidad}             & \multicolumn{1}{c|}{2}              & 0.33                 & 4                             & 3                                   \\ \hline
		\multicolumn{1}{|c|}{C} & \multicolumn{1}{c|}{Facilidad de integración} & \multicolumn{1}{c|}{1}              & 0.17                 & 5                             & 4                                   \\ \hline
		\multicolumn{1}{|c|}{A} & \multicolumn{1}{c|}{Modularidad}                & \multicolumn{1}{c|}{0}              & 0                    & 3                             & 3                                   \\ \hline
		&                                                 &                                     &                      & 4.17                          & 3.17                                \\ \cline{5-6} 
	\end{tabular}
\end{table}

\paragraph*{Sistema de energía\\}
La tabla \ref{sdc_04} presenta la comparación subjetiva entre el botón mecánico y el pedal como opciones de paro de emergencia, considerando modularidad, costo, integración y disponibilidad.
\begin{table}[h!]
	\centering
	\caption{Matriz de selección subjetiva. S4 - Paro de emergencia.\label{sdc_04}}
	\begin{tabular}{|cc|cc|}
		\hline
		\multicolumn{2}{|c|}{\multirow{2}{*}{\textbf{Criterios}}} & \multicolumn{2}{c|}{\textbf{Conceptos}}                       \\ \cline{3-4} 
		\multicolumn{2}{|c|}{}                                    & \multicolumn{1}{c|}{\textbf{Botón mecánico}} & \textbf{Pedal} \\ \hline
		\multicolumn{1}{|c|}{A}    & Modularidad                  & \multicolumn{1}{c|}{Bueno}                   & Bueno          \\ \hline
		\multicolumn{1}{|c|}{B}    & Costo                        & \multicolumn{1}{c|}{Bueno}                   & Regular        \\ \hline
		\multicolumn{1}{|c|}{C}    & Facilidad de   integración   & \multicolumn{1}{c|}{Excelente}               & Bueno          \\ \hline
		\multicolumn{1}{|c|}{D}    & Disponibilidad               & \multicolumn{1}{c|}{Bueno}                   & Regular        \\ \hline
	\end{tabular}
\end{table}

En la tabla \ref{sdc_05} se evaluó la prioridad relativa de cada criterio mediante una matriz binaria.
\begin{table}[h!]
	\centering
	\caption{Matriz binaria. S4. - Paro de emergencia.\label{sdc_05}}
	\begin{tabular}{|cc|c|c|c|c|c|c|}
		\hline
		\multicolumn{2}{|c|}{\textbf{Criterios}}             & \textbf{A}               & \textbf{B}               & \textbf{C}               & \textbf{D}               & \textbf{Total} & \textbf{Rango} \\ \hline
		\multicolumn{1}{|c|}{A} & Modularidad                & \cellcolor[HTML]{BFBFBF} & 0                        & 0                        & 0                        & 0              & 1              \\ \hline
		\multicolumn{1}{|c|}{B} & Costo                      & 1                        & \cellcolor[HTML]{BFBFBF} & 1                        & 1                        & 3              & 2              \\ \hline
		\multicolumn{1}{|c|}{C} & Facilidad de   integración & 1                        & 0                        & \cellcolor[HTML]{BFBFBF} & 0                        & 1              & 3              \\ \hline
		\multicolumn{1}{|c|}{D} & Disponibilidad             & 1                        & 0                        & 1                        & \cellcolor[HTML]{BFBFBF} & 2              & 4              \\ \hline
	\end{tabular}
\end{table}

La matriz de ponderación resultante en la tabla \ref{sdc_06} identificó al botón mecánico como la opción más viable.
\begin{table}[h!]
	\centering
	\caption{Matriz de ponderación. S4 - Paro de emergencia.\label{sdc_06}}
	\begin{tabular}{cccc|c|c|}
		\hline
		\multicolumn{2}{|c|}{\textbf{Criterios}}                                  & \multicolumn{1}{c|}{\textbf{Total}} & \textbf{Ponderación} & \textbf{Botón mecánico} & \textbf{Pedal} \\ \hline
		\multicolumn{1}{|c|}{B} & \multicolumn{1}{c|}{Costo}                      & \multicolumn{1}{c|}{3}              & 0.5                  & 4                       & 3              \\ \hline
		\multicolumn{1}{|c|}{D} & \multicolumn{1}{c|}{Disponibilidad}             & \multicolumn{1}{c|}{2}              & 0.33                 & 4                       & 3              \\ \hline
		\multicolumn{1}{|c|}{C} & \multicolumn{1}{c|}{Facilidad de   integración} & \multicolumn{1}{c|}{1}              & 0.17                 & 5                       & 4              \\ \hline
		\multicolumn{1}{|c|}{A} & \multicolumn{1}{c|}{Modularidad}                & \multicolumn{1}{c|}{0}              & 0                    & 4                       & 4              \\ \hline
		&                                                 &                                     &                      & 4.17                    & 3.17           \\ \cline{5-6} 
	\end{tabular}
\end{table}
\clearpage
Para seleccionar la etapa de potencia, se evaluaron los conceptos de drivers de motores y puentes H. La tabla \ref{sdc_07} expone la matriz subjetiva inicial.
\begin{table}[h!]
	\centering
	\caption{Matriz de selección subjetiva. S4 - Etapa de potencia.\label{sdc_07}}
	\begin{tabular}{|cc|cc|}
		\hline
		\multicolumn{2}{|c|}{\multirow{2}{*}{\textbf{Criterios}}} & \multicolumn{2}{c|}{\textbf{Conceptos}}                               \\ \cline{3-4} 
		\multicolumn{2}{|c|}{}                                    & \multicolumn{1}{c|}{\textbf{Drivers de motores}} & \textbf{Puentes H} \\ \hline
		\multicolumn{1}{|c|}{A}    & Diseño                       & \multicolumn{1}{c|}{Bueno}                       & Regular            \\ \hline
		\multicolumn{1}{|c|}{B}    & Costo                        & \multicolumn{1}{c|}{Regular}                     & Bueno              \\ \hline
		\multicolumn{1}{|c|}{C}    & Facilidad de   integración   & \multicolumn{1}{c|}{Excelente}                   & Regular            \\ \hline
		\multicolumn{1}{|c|}{D}    & Disponibilidad               & \multicolumn{1}{c|}{Bueno}                       & Regular            \\ \hline
	\end{tabular}
\end{table}

En la tabla \ref{sdc_08} se muestra la priorización de criterios con una matriz binaria.
\begin{table}[h!]
	\centering
	\caption{Matriz binaria. S4 - Etapa de potencia.\label{sdc_08}}
	\begin{tabular}{|cc|c|c|c|c|c|c|}
		\hline
		\multicolumn{2}{|c|}{\textbf{Criterios}}             & \textbf{A}               & \textbf{B}               & \textbf{C}               & \textbf{D}               & \textbf{Total} & \textbf{Rango} \\ \hline
		\multicolumn{1}{|c|}{A} & Diseño                     & \cellcolor[HTML]{BFBFBF} & 0                        & 0                        & 0                        & 0              & 1              \\ \hline
		\multicolumn{1}{|c|}{B} & Costo                      & 1                        & \cellcolor[HTML]{BFBFBF} & 1                        & 1                        & 3              & 2              \\ \hline
		\multicolumn{1}{|c|}{C} & Facilidad de   integración & 1                        & 0                        & \cellcolor[HTML]{BFBFBF} & 0                        & 1              & 3              \\ \hline
		\multicolumn{1}{|c|}{D} & Disponibilidad             & 1                        & 0                        & 1                        & \cellcolor[HTML]{BFBFBF} & 2              & 4              \\ \hline
	\end{tabular}
\end{table}

La tabla \ref{sdc_09} concluye el análisis ponderado, favoreciendo los drivers de motores por su mayor facilidad de integración.
\begin{table}[h!]
	\centering
	\caption{Matriz de ponderación. S4 - Etapa de potencia.\label{sdc_09}}
	\begin{tabular}{cccc|c|c|}
		\hline
		\multicolumn{2}{|c|}{\textbf{Criterios}}                                  & \multicolumn{1}{c|}{\textbf{Total}} & \textbf{Ponderación} & \begin{tabular}[c]{@{}c@{}}\textbf{Drivers de}\\\textbf{motores}\end{tabular} & \textbf{Puentes H} \\ \hline
		\multicolumn{1}{|c|}{B} & \multicolumn{1}{c|}{Costo}                      & \multicolumn{1}{c|}{3}              & 0.5                  & 3                            & 4                  \\ \hline
		\multicolumn{1}{|c|}{D} & \multicolumn{1}{c|}{Disponibilidad}             & \multicolumn{1}{c|}{2}              & 0.33                 & 4                            & 3                  \\ \hline
		\multicolumn{1}{|c|}{C} & \multicolumn{1}{c|}{Facilidad de   integración} & \multicolumn{1}{c|}{1}              & 0.17                 & 5                            & 3                  \\ \hline
		\multicolumn{1}{|c|}{A} & \multicolumn{1}{c|}{Diseño}                     & \multicolumn{1}{c|}{0}              & 0                    & 4                            & 3                  \\ \hline
		\multicolumn{4}{c|}{}                                                                                                                  & 3.67                         & 3.5                \\ \cline{5-6} 
	\end{tabular}
\end{table}

La comparación entre fuente conmutada y reguladores lineales se realizó en la matriz subjetiva de la tabla \ref{sdc_10}.
\begin{table}[h!]
	\centering
	\caption{Matriz de selección subjetiva. S4 - Acondicionamiento de energía.\label{sdc_10}}
	\begin{tabular}{|cc|cc|}
		\hline
		\multicolumn{2}{|c|}{\multirow{2}{*}{\textbf{Criterios}}} & \multicolumn{2}{c|}{\textbf{Conceptos}}                               \\ \cline{3-4} 
		\multicolumn{2}{|c|}{}                                    & \multicolumn{1}{c|}{\textbf{Fuente conmutada}} & \textbf{Reguladores} \\ \hline
		\multicolumn{1}{|c|}{A}    & Diseño                       & \multicolumn{1}{c|}{Excelente}                 & Bueno                \\ \hline
		\multicolumn{1}{|c|}{B}    & Costo                        & \multicolumn{1}{c|}{Regular}                   & Bueno                \\ \hline
		\multicolumn{1}{|c|}{C}    & Facilidad de   integración   & \multicolumn{1}{c|}{Excelente}                 & Regular              \\ \hline
		\multicolumn{1}{|c|}{D}    & Disponibilidad               & \multicolumn{1}{c|}{Bueno}                     & Bueno                \\ \hline
	\end{tabular}
\end{table}
\clearpage
Luego, en la tabla \ref{sdc_11} se jerarquizaron los criterios mediante una matriz binaria.
\begin{table}[h!]
	\centering
	\caption{Matriz binaria. S4 - Acondicionamiento de energía.\label{sdc_11}}
	\begin{tabular}{|cc|c|c|c|c|c|c|}
		\hline
		\multicolumn{2}{|c|}{\textbf{Criterios}}             & \textbf{A}               & \textbf{B}               & \textbf{C}               & \textbf{D}               & \textbf{Total} & \textbf{Rango} \\ \hline
		\multicolumn{1}{|c|}{A} & Diseño                     & \cellcolor[HTML]{BFBFBF} & 1                        & 0                        & 0                        & 1              & 1              \\ \hline
		\multicolumn{1}{|c|}{B} & Costo                      & 0                        & \cellcolor[HTML]{BFBFBF} & 1                        & 1                        & 2              & 2              \\ \hline
		\multicolumn{1}{|c|}{C} & Facilidad de   integración & 1                        & 0                        & \cellcolor[HTML]{BFBFBF} & 0                        & 1              & 3              \\ \hline
		\multicolumn{1}{|c|}{D} & Disponibilidad             & 1                        & 0                        & 1                        & \cellcolor[HTML]{BFBFBF} & 2              & 4              \\ \hline
	\end{tabular}
\end{table}

La matriz de ponderación en la tabla \ref{sdc_12} concluyó que la fuente conmutada representa una mejor alternativa.
\begin{table}[h!]
	\centering
	\caption{Matriz de ponderación. S4 - Acondicionamiento de energía.\label{sdc_12}}
	\begin{tabular}{cccc|c|c|}
		\hline
		\multicolumn{2}{|c|}{\textbf{Criterios}}                                  & \multicolumn{1}{c|}{\textbf{Total}} & \textbf{Ponderación} & \begin{tabular}[c]{@{}c@{}}\textbf{Fuente}\\\textbf{conmutada}\end{tabular} & \textbf{Reguladores} \\ \hline
		\multicolumn{1}{|c|}{B} & \multicolumn{1}{c|}{Costo}                      & \multicolumn{1}{c|}{2}              & 0.33                 & 3                         & 4                    \\ \hline
		\multicolumn{1}{|c|}{D} & \multicolumn{1}{c|}{Disponibilidad}             & \multicolumn{1}{c|}{2}              & 0.33                 & 4                         & 4                    \\ \hline
		\multicolumn{1}{|c|}{C} & \multicolumn{1}{c|}{Facilidad de   integración} & \multicolumn{1}{c|}{1}              & 0.17                 & 5                         & 3                    \\ \hline
		\multicolumn{1}{|c|}{A} & \multicolumn{1}{c|}{Diseño}                     & \multicolumn{1}{c|}{1}              & 0.17                 & 5                         & 4                    \\ \hline
		&                                                 &                                     &                      & 4                         & 3.83                 \\ \cline{5-6} 
	\end{tabular}
\end{table}

\paragraph*{Sistema de movimiento\\}
La tabla \ref{sdc_13} compara un motor lineal frente a dos motores lineales, considerando volumen, costo, integración y disponibilidad.
\begin{table}[h!]
	\centering
	\caption{Matriz de selección subjetiva. S5 - Actuadores lineales.\label{sdc_13}}
	\begin{tabular}{|cc|cc|}
		\hline
		\multicolumn{2}{|c|}{\multirow{2}{*}{\textbf{Criterios}}} & \multicolumn{2}{c|}{\textbf{Conceptos}}                                       \\ \cline{3-4} 
		\multicolumn{2}{|c|}{}                                    & \multicolumn{1}{c|}{\textbf{Un motor lineal}} & \textbf{Dos motores lineales} \\ \hline
		\multicolumn{1}{|c|}{A}    & Volumen ocupado              & \multicolumn{1}{c|}{Bueno}                    & Regular                       \\ \hline
		\multicolumn{1}{|c|}{B}    & Costo                        & \multicolumn{1}{c|}{Bueno}                    & Malo                          \\ \hline
		\multicolumn{1}{|c|}{C}    & Facilidad de   integración   & \multicolumn{1}{c|}{Bueno}                    & Regular                       \\ \hline
		\multicolumn{1}{|c|}{D}    & Disponibilidad               & \multicolumn{1}{c|}{Bueno}                    & Bueno                         \\ \hline
	\end{tabular}
\end{table}

La matriz binaria en la tabla \ref{sdc_14} estableció la prioridad de criterios.
\begin{table}[h!]
	\centering
	\caption{Matriz binaria. S5 - Actuadores lineales.\label{sdc_14}}
	\begin{tabular}{|cc|c|c|c|c|c|c|}
		\hline
		\multicolumn{2}{|c|}{\textbf{Criterios}}             & \textbf{A}               & \textbf{B}               & \textbf{C}               & \textbf{D}               & \textbf{Total} & \textbf{Rango} \\ \hline
		\multicolumn{1}{|c|}{A} & Volumen ocupado            & \cellcolor[HTML]{BFBFBF} & 1                        & 0                        & 0                        & 1              & 1              \\ \hline
		\multicolumn{1}{|c|}{B} & Costo                      & 0                        & \cellcolor[HTML]{BFBFBF} & 1                        & 1                        & 2              & 2              \\ \hline
		\multicolumn{1}{|c|}{C} & Facilidad de   integración & 1                        & 0                        & \cellcolor[HTML]{BFBFBF} & 0                        & 1              & 3              \\ \hline
		\multicolumn{1}{|c|}{D} & Disponibilidad             & 1                        & 0                        & 1                        & \cellcolor[HTML]{BFBFBF} & 2              & 4              \\ \hline
	\end{tabular}
\end{table}

Finalmente, la matriz de ponderación de la tabla \ref{sdc_15} favoreció la opción de un solo motor lineal por su menor volumen y costo.
\begin{table}[h!]
	\centering
	\caption{Matriz de ponderación. S5 - Actuadores lineales.\label{sdc_15}}
	\scalebox{0.9}{\begin{tabular}{cccc|c|c|}
			\hline
			\multicolumn{2}{|c|}{\textbf{Criterios}}                                  & \multicolumn{1}{c|}{\textbf{Total}} & \textbf{Ponderación} & \begin{tabular}[c]{@{}c@{}}\textbf{Un motor}\\\textbf{lineal}\end{tabular} & \begin{tabular}[c]{@{}c@{}}\textbf{Dos motores}\\\textbf{lineales}\end{tabular} \\ \hline
			\multicolumn{1}{|c|}{B} & \multicolumn{1}{c|}{Costo}                      & \multicolumn{1}{c|}{2}              & 0.33                 & 4                         & 1                    \\ \hline
			\multicolumn{1}{|c|}{D} & \multicolumn{1}{c|}{Disponibilidad}             & \multicolumn{1}{c|}{2}              & 0.33                 & 4                         & 4                    \\ \hline
			\multicolumn{1}{|c|}{C} & \multicolumn{1}{c|}{Facilidad de   integración} & \multicolumn{1}{c|}{1}              & 0.17                 & 4                         & 3                    \\ \hline
			\multicolumn{1}{|c|}{A} & \multicolumn{1}{c|}{Volumen ocupado}            & \multicolumn{1}{c|}{1}              & 0.17                 & 4                         & 3                    \\ \hline
			\multicolumn{4}{c|}{}                                                                                                                  & 4                         & 2.67                 \\ \cline{5-6} 
	\end{tabular}}
\end{table}
\paragraph*{Sistema HMI\\}
En la tabla \ref{sdc_16} se contrastaron la pantalla táctil y la botonera con display en cuanto a manipulabilidad, costo, integración y disponibilidad. La tabla \ref{sdc_17} presenta la comparación binaria entre estos criterios, mientras que en la matriz de ponderación final (tabla \ref{sdc_18}), la pantalla táctil resultó ser la mejor opción para la interfaz usuario-sistema.
\begin{table}[h!]
	\centering
	\caption{Matriz de selección subjetiva. S6 - Módulo entrada/salida.\label{sdc_16}}
	\scalebox{0.9}{\begin{tabular}{|cc|cc|}
			\hline
			\multicolumn{2}{|c|}{\multirow{2}{*}{\textbf{Criterios}}} & \multicolumn{2}{c|}{\textbf{Conceptos}}                                       \\ \cline{3-4} 
			\multicolumn{2}{|c|}{}                                    & \multicolumn{1}{c|}{\textbf{Pantalla táctil}} & \textbf{Botonera con display} \\ \hline
			\multicolumn{1}{|c|}{A}    & Manipulabilidad              & \multicolumn{1}{c|}{Bueno}                    & Bueno                         \\ \hline
			\multicolumn{1}{|c|}{B}    & Costo                        & \multicolumn{1}{c|}{Bueno}                    & Bueno                         \\ \hline
			\multicolumn{1}{|c|}{C}    & Facilidad de   integración   & \multicolumn{1}{c|}{Excelente}                & Regular                       \\ \hline
			\multicolumn{1}{|c|}{D}    & Disponibilidad               & \multicolumn{1}{c|}{Bueno}                    & Regular                       \\ \hline
	\end{tabular}}
\end{table}

\begin{table}[h!]
	\centering
	\caption{Matriz binaria. S6 - Módulo entrada/salida.\label{sdc_17}}
	\scalebox{0.9}{\begin{tabular}{|cc|c|c|c|c|c|c|}
			\hline
			\multicolumn{2}{|c|}{\textbf{Criterios}}             & \textbf{A}               & \textbf{B}               & \textbf{C}               & \textbf{D}               & \textbf{Total} & \textbf{Rango} \\ \hline
			\multicolumn{1}{|c|}{A} & Manipulabilidad            & \cellcolor[HTML]{BFBFBF} & 1                        & 0                        & 0                        & 1              & 1              \\ \hline
			\multicolumn{1}{|c|}{B} & Costo                      & 0                        & \cellcolor[HTML]{BFBFBF} & 0                        & 0                        & 0              & 2              \\ \hline
			\multicolumn{1}{|c|}{C} & Facilidad de   integración & 1                        & 1                        & \cellcolor[HTML]{BFBFBF} & 0                        & 2              & 3              \\ \hline
			\multicolumn{1}{|c|}{D} & Disponibilidad             & 1                        & 1                        & 1                        & \cellcolor[HTML]{BFBFBF} & 3              & 4              \\ \hline
	\end{tabular}}
\end{table}

\begin{table}[h!]
	\centering
	\caption{Matriz de ponderación. S6 - Módulo entrada/salida.\label{sdc_18}}
	\scalebox{0.9}{\begin{tabular}{cccc|c|c|}
			\hline
			\multicolumn{2}{|c|}{\textbf{Criterios}}                                  & \multicolumn{1}{c|}{\textbf{Total}} & \textbf{Ponderación} & \begin{tabular}[c]{@{}c@{}}\textbf{Pantalla}\\\textbf{táctil}\end{tabular} & \begin{tabular}[c]{@{}c@{}}\textbf{Botonera}\\\textbf{con display}\end{tabular} \\ \hline
			\multicolumn{1}{|c|}{D} & \multicolumn{1}{c|}{Disponibilidad}             & \multicolumn{1}{c|}{3}              & 0.5                  & 4                         & 3                    \\ \hline
			\multicolumn{1}{|c|}{C} & \multicolumn{1}{c|}{Facilidad de   integración} & \multicolumn{1}{c|}{2}              & 0.33                 & 5                         & 3                    \\ \hline
			\multicolumn{1}{|c|}{A} & \multicolumn{1}{c|}{Manipulabilidad}            & \multicolumn{1}{c|}{1}              & 0.17                 & 4                         & 4                    \\ \hline
			\multicolumn{1}{|c|}{B} & \multicolumn{1}{c|}{Costo}                      & \multicolumn{1}{c|}{0}              & 0                    & 4                         & 4                    \\ \hline
			&                                                 &                                     &                      & 4.33                      & 3.17                 \\ \cline{5-6} 
	\end{tabular}}
\end{table}
\clearpage
\paragraph*{Sistema de control\\}
La tabla \ref{sdc_19} compara sensores con y sin potenciómetros desde criterios técnicos y de integración.
\begin{table}[h!]
	\centering
	\caption{Matriz de selección subjetiva. S7 - Sensores.\label{sdc_19}}
	\scalebox{0.9}{\begin{tabular}{|cc|cc|}
			\hline
			\multicolumn{2}{|c|}{\multirow{2}{*}{\textbf{Criterios}}} & \multicolumn{2}{c|}{\textbf{Conceptos}}                                        \\ \cline{3-4} 
			\multicolumn{2}{|c|}{}                                    & \multicolumn{1}{c|}{\textbf{Sin potenciómetros}} & \textbf{Con potenciómetros} \\ \hline
			\multicolumn{1}{|c|}{A}    & Manipulabilidad              & \multicolumn{1}{c|}{Bueno}                       & Regular                     \\ \hline
			\multicolumn{1}{|c|}{B}    & Costo                        & \multicolumn{1}{c|}{Bueno}                       & Bueno                       \\ \hline
			\multicolumn{1}{|c|}{C}    & Facilidad de   integración   & \multicolumn{1}{c|}{Excelente}                   & Regular                     \\ \hline
			\multicolumn{1}{|c|}{D}    & Disponibilidad               & \multicolumn{1}{c|}{Excelente}                   & Regular                     \\ \hline
	\end{tabular}}
\end{table}

Posteriormente, se priorizaron dichos criterios en la tabla binaria \ref{sdc_20}.
\begin{table}[h!]
	\centering
	\caption{Matriz binaria. S7 - Sensores.\label{sdc_20}}
	\scalebox{0.9}{\begin{tabular}{|cc|c|c|c|c|c|c|}
			\hline
			\multicolumn{2}{|c|}{\textbf{Criterios}}             & \textbf{A}               & \textbf{B}               & \textbf{C}               & \textbf{D}               & \textbf{Total} & \textbf{Rango} \\ \hline
			\multicolumn{1}{|c|}{A} & Manipulabilidad            & \cellcolor[HTML]{BFBFBF} & 1                        & 0                        & 0                        & 1              & 1              \\ \hline
			\multicolumn{1}{|c|}{B} & Costo                      & 0                        & \cellcolor[HTML]{BFBFBF} & 0                        & 0                        & 0              & 2              \\ \hline
			\multicolumn{1}{|c|}{C} & Facilidad de   integración & 1                        & 1                        & \cellcolor[HTML]{BFBFBF} & 0                        & 2              & 3              \\ \hline
			\multicolumn{1}{|c|}{D} & Disponibilidad             & 1                        & 1                        & 1                        & \cellcolor[HTML]{BFBFBF} & 3              & 4              \\ \hline
	\end{tabular}}
\end{table}

En la tabla \ref{sdc_21}, la matriz de ponderación muestra que los sensores sin potenciómetros ofrecen mayor compatibilidad general.
\begin{table}[h!]
	\centering
	\caption{Matriz de ponderación. S7 - Sensores.\label{sdc_21}}
	\scalebox{0.96}{\begin{tabular}{cccc|c|c|}
			\hline
			\multicolumn{2}{|c|}{\textbf{Criterios}}                                  & \multicolumn{1}{c|}{\textbf{Total}} & \textbf{Ponderación} & \begin{tabular}[c]{@{}c@{}}\textbf{Sin}\\\textbf{potenciómetros}\end{tabular} & \begin{tabular}[c]{@{}c@{}}\textbf{Con}\\\textbf{potenciómetros}\end{tabular} \\ \hline
			\multicolumn{1}{|c|}{D} & \multicolumn{1}{c|}{Disponibilidad}             & \multicolumn{1}{c|}{3}              & 0.5                  & 5                             & 3                             \\ \hline
			\multicolumn{1}{|c|}{C} & \multicolumn{1}{c|}{Facilidad de   integración} & \multicolumn{1}{c|}{2}              & 0.33                 & 5                             & 3                             \\ \hline
			\multicolumn{1}{|c|}{A} & \multicolumn{1}{c|}{Manipulabilidad}            & \multicolumn{1}{c|}{1}              & 0.17                 & 4                             & 3                             \\ \hline
			\multicolumn{1}{|c|}{B} & \multicolumn{1}{c|}{Costo}                      & \multicolumn{1}{c|}{0}              & 0                    & 4                             & 4                             \\ \hline
			\multicolumn{4}{c|}{}                                                                                                                  & 4.83                          & 3                             \\ \cline{5-6} 
	\end{tabular}}
\end{table}

Se realizó una evaluación comparativa entre Raspberry y Arduino en la tabla \ref{sdc_22} considerando facilidad de programación, costo e integración.
\begin{table}[h!]
	\centering
	\caption{Matriz de selección subjetiva. S7 - Procesamiento.\label{sdc_22}}
	\scalebox{0.9}{\begin{tabular}{|cc|cc|}
			\hline
			\multicolumn{2}{|c|}{\multirow{2}{*}{\textbf{Criterios}}} & \multicolumn{2}{c|}{\textbf{Conceptos}}                    \\ \cline{3-4} 
			\multicolumn{2}{|c|}{}                                    & \multicolumn{1}{c|}{\textbf{Raspberry}} & \textbf{Arduino} \\ \hline
			\multicolumn{1}{|c|}{A}    & Programación                 & \multicolumn{1}{c|}{Excelente}          & Excelente        \\ \hline
			\multicolumn{1}{|c|}{B}    & Costo                        & \multicolumn{1}{c|}{Bueno}              & Bueno            \\ \hline
			\multicolumn{1}{|c|}{C}    & Facilidad de   integración   & \multicolumn{1}{c|}{Excelente}          & Regular          \\ \hline
			\multicolumn{1}{|c|}{D}    & Disponibilidad               & \multicolumn{1}{c|}{Excelente}          & Excelente        \\ \hline
	\end{tabular}}
\end{table}
\clearpage
La tabla \ref{sdc_23} organizó la importancia de los criterios mediante comparación binaria.
\begin{table}[h!]
	\centering
	\caption{Matriz binaria. S7 - Procesamiento.\label{sdc_23}}
	\begin{tabular}{|cc|c|c|c|c|c|c|}
		\hline
		\multicolumn{2}{|c|}{\textbf{Criterios}}             & \textbf{A}               & \textbf{B}               & \textbf{C}               & \textbf{D}               & \textbf{Total} & \textbf{Rango} \\ \hline
		\multicolumn{1}{|c|}{A} & Programación               & \cellcolor[HTML]{BFBFBF} & 1                        & 1                        & 1                        & 3              & 1              \\ \hline
		\multicolumn{1}{|c|}{B} & Costo                      & 0                        & \cellcolor[HTML]{BFBFBF} & 0                        & 0                        & 0              & 2              \\ \hline
		\multicolumn{1}{|c|}{C} & Facilidad de   integración & 0                        & 1                        & \cellcolor[HTML]{BFBFBF} & 0                        & 1              & 3              \\ \hline
		\multicolumn{1}{|c|}{D} & Disponibilidad             & 0                        & 1                        & 1                        & \cellcolor[HTML]{BFBFBF} & 2              & 4              \\ \hline
	\end{tabular}
\end{table}

Finalmente, en la tabla \ref{sdc_24}, la matriz de ponderación favoreció la Raspberry por su mayor capacidad de integración y disponibilidad.
\begin{table}[h!]
	\centering
	\caption{Matriz de ponderación. S7 - Procesamiento.\label{sdc_24}}
	\begin{tabular}{cccc|c|c|}
		\hline
		\multicolumn{2}{|c|}{\textbf{Criterios}}                                  & \multicolumn{1}{c|}{\textbf{Total}} & \textbf{Ponderación} & \textbf{Raspberry} & \textbf{Arduino} \\ \hline
		\multicolumn{1}{|c|}{A} & \multicolumn{1}{c|}{Programación}               & \multicolumn{1}{c|}{3}              & 0.5                  & 5                  & 5                \\ \hline
		\multicolumn{1}{|c|}{D} & \multicolumn{1}{c|}{Disponibilidad}             & \multicolumn{1}{c|}{2}              & 0.33                 & 5                  & 5                \\ \hline
		\multicolumn{1}{|c|}{C} & \multicolumn{1}{c|}{Facilidad de   integración} & \multicolumn{1}{c|}{1}              & 0.17                 & 5                  & 3                \\ \hline
		\multicolumn{1}{|c|}{B} & \multicolumn{1}{c|}{Costo}                      & \multicolumn{1}{c|}{0}              & 0                    & 4                  & 4                \\ \hline
		&                                                 &                                     &                      & 5                  & 4.67             \\ \cline{5-6} 
	\end{tabular}
\end{table}

Con base en estas matrices es que se determina que la propuesta del concepto solución 1 es el que satisface de mejor forma nuestros requerimientos. Sin embargo, no se descartan modificaciones que sean necesarias en la realización del diseño detallado.