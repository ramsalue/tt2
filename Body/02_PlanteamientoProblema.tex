\clearpage
\subsection*{Definición del problema}
\addcontentsline{toc}{subsection}{Planteamiento del problema}
Con base en una entrevista realizada al Licenciado en Terapia Física Mario Sánchez Aguilar, de la Unidad de Medicina Física y Rehabilitación del Norte – IMSS, ubicado en la alcaldía Gustavo A. Madero de la Ciudad de México, una de las fases cruciales en el proceso de rehabilitación de un paciente de hemiplejia es la fase de flacidez, en la cual el paciente no puede mover el hemicuerpo afectado, ya que los músculos se encuentran caídos y sin fuerza. En esta fase deben realizarse sesiones de rehabilitación que implican movimientos de flexión, extensión, abducción y aducción en las articulaciones de coxofemoral y rodilla. Estos movimientos se tienen que realizar de acuerdo con la sesión de rehabilitación y las necesidades de cada paciente, por esta razón surge la problemática de implementar un sistema mecatrónico que ayude al fisioterapeuta a realizar adecuadamente los movimientos mencionados, debido a que son esenciales para prevenir complicaciones en el paciente como la atrofia muscular, desvíos posturales y contracturas.\\

Para atender esta problemática se implementará una ortesis\footnote{Una ortesis es un dispositivo mecánico para sostener, corregir o asistir el movimiento de una parte del cuerpo, mejorando su funcionalidad \cite{20}.} robótica que asista al fisioterapeuta en las sesiones de rehabilitación, en la cual se ingresen los parámetros por el fisioterapeuta en cada inicio de sesión a través de una interfaz humano-máquina, además, se propone un diseño modular que permita adaptar la ortesis a diferentes pacientes, esto representa un gran desafío, ya que, las ortesis tienen que personalizarse según las necesidades individuales de cada paciente, lo que puede generar una gran variedad de configuraciones y diseños. Existen diversos retos que se involucran en el desarrollo de la ortesis, entre los cuales destacan:

\begin{itemize}
	\item	Implementar una ortesis robótica modular que se adapte a distintas estaturas promedio de personas adultas en México, que se encuentran en un rango de 150 cm a 170 cm.
    \item	Diseñar y fabricar una estructura estable, que sea capaz de soportar un peso máximo de 80 kg de una persona adulta.
    \item	Implementar sensores de retroalimentación, para ajustar los rangos de movimiento, la fuerza aplicada y la velocidad inducida durante las sesiones de rehabilitación.
    \item	Diseñar un sistema de sujeción para retirar y montar la región inferior derecha sin complicaciones durante el proceso.
    \item	Implementar un diseño ergonómico que sea cómodo para el paciente, protegiendo sus articulaciones y evitando lesiones por uso prolongado.
    \item	Desarrollar una interfaz humano-máquina que permita ingresar y visualizar los diferentes parámetros de movimiento.
\end{itemize}
