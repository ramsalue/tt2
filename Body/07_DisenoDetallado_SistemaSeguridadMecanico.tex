\subsubsection{S3. Sistema de seguridad mecánico}
El sistema de seguridad mecánico (S3) se compone de dos módulos: el módulo de topes mecánicos (M4) y el módulo de sujeción y ajuste (M5). Este sistema es de mucha importancia ya que su función principal es proteger al paciente de movimientos que sobrepasen los límites establecidos, en caso de que falle el control o la programación del sistema, o que ocurra algún problema con los movimientos que ejecuten los mecanismos de flexión-extensión y abducción-aducción. De igual manera asegura que la pierna del paciente se encuentre correctamente posicionada y no se mueva durante las sesiones de rehabilitación para que el paciente este cómodo.
\paragraph{M3. Módulo de topes mecánicos}
Este módulo es el encargado de limitar físicamente el movimiento de los mecanismos de flexión-extensión y abducción-aducción mediante topes mecánicos implementados en la estructura de la ortesis y en la estructura de la cama, evitando que el paciente tenga alguna molestia o sufra una lesión cuando falle el control del sistema general o cuando se presente alguna falla en la programación que cause que la pierna del paciente se vaya a flexionar o extender más allá de los límites definidos.
\paragraph*{Topes mecanismo de flexión-extensión}\mbox{}\\
Cuando ocurre el movimiento de flexión de coxofemoral y rodilla la pierna adopta una posición como la mostrada en la Fig. \ref{fig:Pina_Figura17}, y el ángulo que se forma entre la base del mecanismo y la parte inferior del muslo no debe sobrepasar el valor máximo definido por el fisioterapeuta porque esto afectará al paciente. El ángulo máximo se podrá ajustar entre 0° y 50° aproximadamente y al momento de llegar al valor establecido el mecanismo se va a detener gracias al tope que se encuentra atravesado de un lado a otro, el cual se podrá desplazar a lo largo de la estructura para definir el ángulo deseado y se fija mediante unos tornillos de sujeción. Cuando la pierna se extienda del otro lado también se encuentra un tope que atraviesa la estructura de un lado a otro y este no va a permitir que la pierna se extienda más de lo establecido, y también este tope mecánico se puede mover a lo largo de la estructura para poder definir el límite de acuerdo con las características de cada paciente.
 \begin{figure}[h!]
 	\centering
 	\includegraphics [trim = 0 0cm 0 0, clip, width=0.8\textwidth]{figure/s1_estructural/Pina_Figura17.png}
 	\caption[Posición donde el mecanismo toca el tope mécanico al hacer flexión de coxofemoral.]{Posición donde el mecanismo toca el tope mécanico al hacer flexión de coxofemoral.\label{fig:Pina_Figura17}}
 \end{figure}

  \paragraph*{Aplicación de fuerzas y resultado de tensiones, desplazamientos y factor de seguridad\\}

 Para verificar que los topes se diseñaron correctamente se realizó un análisis estático en SolidWorks y se aplicó una carga de 150N sobre la cara mostrada en la Fig. \ref{fig:Pina_Figura18}, esta carga tiene esa magnitud porque se contempló que la pierna del paciente pesa 23Kg y el tope además de soportar ese peso debe soportar el peso de la estructura que es de 7Kg aproximadamente, lo cual da un total de 30Kg que se van a distribuir en 2 puntos (15Kg cada uno), el primero está en las uniones de los tubos del lado izquierdo y el segundo sobre el tope.

 Al ejecutar el análisis estático arrojó una tensión máxima de 110MPa y el límite elástico del material utilizado es de 275MPa un poco más del doble, lo que indica que el tope es capaz soportar la carga aplicada y no va a ceder [Fig. \ref{fig:Pina_Figura19}].

 El desplazamiento máximo es de 0.02mm y se encuentra en el centro del tope, este valor es muy pequeño y no afectará la geometría de la estructura, por lo tanto, el resultado es adecuado para este tope [Fig. \ref{fig:Pina_Figura20}].

 El tope mecánico tiene un factor de seguridad de 2.5, lo que proporciona seguridad al paciente y nos asegura de que puede soportar un poco más del doble de la carga aplicada y no va a fallar [Fig. \ref{fig:Pina_Figura21}].

 \begin{figure}[h!]
	\centering
    \subcaptionbox{Aplicación de fuerzas para estudio de tope mecánico.\label{fig:Pina_Figura18}}
	{\includegraphics [trim = 0 0cm 0 0, clip, width=0.4\textwidth]{figure/s1_estructural/Pina_Figura18.png}}
	\subcaptionbox{Resultado del estudio de tensiones.\label{fig:Pina_Figura19}}
	{\includegraphics [trim = 0 0cm 0 0, clip, width=0.49\textwidth]{figure/s1_estructural/Pina_Figura19.png}}
	\subcaptionbox{Resultado del estudio de desplazamientos.\label{fig:Pina_Figura20}}
	{\includegraphics [trim = 0 0cm 0 0, clip, width=0.49\textwidth]{figure/s1_estructural/Pina_Figura20.png}}
    \subcaptionbox{Resultado del estudio de factor de seguridad.\label{fig:Pina_Figura21}}
	{\includegraphics [trim = 0 0cm 0 0, clip, width=0.49\textwidth]{figure/s1_estructural/Pina_Figura21.png}}
	\caption[Resultados de un estudio de tensiones, desplazamientos, y factor seguridad en SolidWorks de acuerdo a las fuerzas de la Fig. \ref{fig:Pina_Figura18}]{Resultados de un estudio de tensiones, desplazamientos, y factor seguridad en SolidWorks de acuerdo a las fuerzas de la Fig. \ref{fig:Pina_Figura18}\label{fig:Pina_Figura192021}}
\end{figure}

 Además de los topes diseñados para definir la flexión y extensión máxima de acuerdo con las características de cada paciente, se diseñó otro tope para que la flexión máxima permitida en la ortesis sea de 50° aproximadamente y la extensión máxima sea muy cercana a los 0° sin importar el tamaño de la pierna del paciente que vaya a utilizar la ortesis. En la Fig. \ref{fig:Pina_Figura22} se puede observar que este tope es una pequeña placa con una ranura sobre la cual sobresale un perno, y al momento de llegar a los límites establecidos, este tope no permitirá sobrepasarlos.

 El tope mecánico mencionado anteriormente se podrá montar y desmontar gracias a unos tornillos de fijación, idealmente, este tope deberá estar montado en todo momento sobre la ortesis pero en dado caso de que el fisioterapeuta lo requiera se podrá retirar fácilmente. Para verificar que el tope sea seguro se implementó un análisis estático y los resultados obtenidos podemos observarlos en las Figs. \ref{fig:Pina_Figura23}, \ref{fig:Pina_Figura24} y \ref{fig:Pina_Figura25}, los cuales fueron los siguientes: Al aplicarle una carga de 150N como en el tope que atraviesa la estructura, se presenta una tensión máxima de 40.5MPa y el límite elástico se encuentra arriba de los 200MPa, también se presenta un desplazamiento máximo de 0.018mm y un factor de seguridad de 5.1, de tal manera que al analizar los resultados obtenidos comprobamos que este tope es capaz de soportar la carga aplicada y no va a fallar, lo que representa seguridad para el paciente.
  \begin{figure}[h!]
 	\centering
 	\includegraphics [trim = 0 0cm 0 0, clip, width=0.5\textwidth]{figure/s1_estructural/Pina_Figura22.png}
 	\caption[Ensamble de tope mecánico para flexión y extensión máximas]{Ensamble de tope mecánico para flexión y extensión máximas\label{fig:Pina_Figura22}}
 \end{figure}
\begin{figure}[h!]
	\centering
	\subcaptionbox[Resultados de un estudio de tensiones en SolidWorks para una flexión/extensión máxima]{Resultados de un estudio de tensiones en SolidWorks para una flexión/extensión máxima\label{fig:Pina_Figura23}}
	{\includegraphics [trim = 0 0cm 0 0, clip, width=0.49\textwidth]{figure/s1_estructural/Pina_Figura23.png}}
	\subcaptionbox[Resultados de un estudio de desplazamientos en SolidWorks para una flexión/extensión máxima]{Resultados de un estudio de desplazamientos en SolidWorks para una flexión/extensión máxima\label{fig:Pina_Figura24}}
	{\includegraphics [trim = 0 0cm 0 0, clip, width=0.49\textwidth]{figure/s1_estructural/Pina_Figura24.png}}
    \subcaptionbox[Resultados de un estudio para conocer el Factor de Seguridad en SolidWorks para una flexión/extensión máxima]{Resultados de un estudio para conocer el Factor de Seguridad en SolidWorks para una flexión/extensión máxima\label{fig:Pina_Figura25}}
	{\includegraphics [trim = 0 0cm 0 0, clip, width=0.49\textwidth]{figure/s1_estructural/Pina_Figura25.png}}
	\caption[Resultados de estudios en SolidWorks para el tope de flexión/extensión máxima]{Resultados de estudios en SolidWorks para el tope de flexión/extensión máxima\label{fig:Pina_Figura232425}}
\end{figure}

\paragraph*{ Topes mecanismo de abducción-aducción\\}
El mecanismo de abducción-aducción requiere que el paciente no pueda hacer un movimiento de abducción con un ángulo mayor a 50°, de tal manera que se necesita colocar un tope mecánico para evitar que el paciente sufra algún daño si llega a fallar el control del sistema. En la Fig. \ref{fig:Pina_Figura26} se observa la forma del tope y la posición en donde se coloca, y en la Fig. \ref{fig:Pina_Figura27} se muestra que para verificar la seguridad de este tope se realizó un análisis estático en donde se aplicó una carga de 8 N sobre la cara larga del tope.
\begin{figure}[h!]
	\centering
	\subcaptionbox[Ensamble de tope mecánico para abducción y aducción]{Ensamble de tope mecánico para abducción y aducción\label{fig:Pina_Figura26}}
	{\includegraphics [trim = 0 0cm 0 0, clip, width=0.4\textwidth]{figure/s1_estructural/Pina_Figura26.png}}\\
	\subcaptionbox[Aplicación de fuerzas sobre el tope mecánico para un análisis]{Aplicación de fuerzas sobre el tope mecánico para un análisis\label{fig:Pina_Figura27}}
	{\includegraphics [trim = 0 0cm 0 0, clip, width=0.69\textwidth]{figure/s1_estructural/Pina_Figura27.png}}
	\caption[Ensamble de tope mecánico sobre estructura de la cama]{Ensamble de tope mecánico sobre estructura de la cama\label{fig:Pina_Figura2627}}
\end{figure}

Los resultados obtenidos muestran que existe una tensión máxima de 0.09364 MPa y el límite elástico es de 300 MPa aproximadamente, de igual forma se identifica un desplazamiento máximo de 4.349E-4 mm y un factor de seguridad de 3.2E3, por lo tanto, la implementación de este tope es aceptable y no va a fallar [Fig. \ref{fig:Pina_Figura28}, Fig. \ref{fig:Pina_Figura29}, Fig. \ref{fig:Pina_Figura30}].
\begin{figure}[h!]
	\centering
	\subcaptionbox[Resultados de estudio de tensiones en SolidWorks para el tope de abducción y aducción]{Resultados de estudio de tensiones en SolidWorks para el tope de abducción y aducción\label{fig:Pina_Figura28}}
	{\includegraphics [trim = 0 0cm 0 0, clip, width=0.49\textwidth]{figure/s1_estructural/Pina_Figura28.png}}\\
	\subcaptionbox[Resultados de estudio de desplazamientos en SolidWorks para el tope de abducción y aducción.]{Resultados de estudio de desplazamientos en SolidWorks para el tope de abducción y aducción.\label{fig:Pina_Figura29}}
	{\includegraphics [trim = 0 0cm 0 0, clip, width=0.49\textwidth]{figure/s1_estructural/Pina_Figura29.png}}\\
    \subcaptionbox[Resultados de estudio para conocer el Factor de Seguridad en SolidWorks para el tope de abducción y aducción.]{Resultados de estudio para conocer el Factor de Seguridad en SolidWorks para el tope de abducción y aducción.\label{fig:Pina_Figura30}}
	{\includegraphics [trim = 0 0cm 0 0, clip, width=0.49\textwidth]{figure/s1_estructural/Pina_Figura30.png}}
	\caption[Resultados de estudios en SolidWorks para el tope de abducción y aducción]{Resultados de estudios en SolidWorks para el tope de abducción y aducción\label{fig:Pina_Figura282930}}
\end{figure}

\paragraph{M4. Módulo de sujeción y ajuste}\mbox{}\\
Este módulo se encarga de mantener la pierna sujetada firmemente a la ortesis para que no se desvíe o haga movimientos no deseados, además se encarga de que la ortesis se pueda adaptar a la pierna del paciente para garantizar comodidad.

Para implementar el ajuste a diferentes tamaños de pierna se utilizaron tubos telescópicos en la parte del fémur y de la tibia, en la Fig. \ref{fig:Pina_Figura35} podemos observar la ortesis con las medidas mínimas que tiene, y en la Fig. \ref{fig:Pina_Figura36} se observa la ortesis con las medidas máximas. El tamaño de los tubos para el fémur y el tamaño de los tubos para la tibia una vez ajustados al tamaño de la pierna del paciente se fijan mediante 4 tornillos uno en cada tubo y de esta manera mantenemos la estructura rígida para que los movimientos se realicen correctamente.
\begin{figure}[h!]
	\centering
	\subcaptionbox[Configuración mínima de los tubos telescópicos.]{Configuración mínima de los tubos telescópicos.\label{fig:Pina_Figura35}}
	{\includegraphics [trim = 0 0cm 0 0, clip, width=0.6\textwidth]{figure/s1_estructural/Pina_Figura35.png}}
	\subcaptionbox[Configuración máxima de los tubos telescópicos.]{Configuración máxima de los tubos telescópicos.\label{fig:Pina_Figura36}}
	{\includegraphics [trim = 0 0cm 0 0, clip, width=0.6\textwidth]{figure/s1_estructural/Pina_Figura36.png}}
	\caption[Configuraciones de los tubos telescópicos para el ensamble.]{Configuraciones de los tubos telescópicos para el ensamble.\label{fig:Pina_Figura3536}}
\end{figure}

Por otro lado, para sujetar la pierna del paciente a la ortesis se usarán cintas de velcro que se colocarán alrededor de los soportes para el fémur y para la tibia y abrazarán la pierna de la persona, estas cintas no van unidas a la ortesis y se colocan en el momento de la sesión de rehabilitación, en la Fig. \ref{fig:Pina_Figura37} observamos los tipos de cintas que se planean utilizar.

De igual manera para sujetar el pie del paciente se va a colocar otra cinta que será más pequeña, pero también será de velcro como la que se observa en la Fig. \ref{fig:Pina_Figura38}.

\begin{figure}[h!]
	\centering
	\subcaptionbox[Cinta que envolverán los soportes.]{Cinta que envolverán los soportes.\label{fig:Pina_Figura37}}
	{\includegraphics [angle=90,trim = 0 0cm 0 0, clip, width=0.4\textwidth]{figure/s1_estructural/Pina_Figura37.png}}
	\subcaptionbox[Cinta que envolverán las piernas.]{Cinta que envolverán las piernas.\label{fig:Pina_Figura38}}
	{\includegraphics [trim = 0 0cm 0 0, clip, width=0.4\textwidth]{figure/s1_estructural/Pina_Figura38.png}}
	\caption[Cintas de velcro para sujeción de piernas.]{Cintas de velcro para sujeción de piernas.\label{fig:Pina_Figura3738}}
\end{figure}


Para brindar una mayor comodidad al momento de sujetar la pierna del paciente, entre el soporte de PVC y la pierna se colocará un pequeño colchón como el que se observa en la Fig. \ref{fig:cama}, este tipo de colchones se pueden recortar y ajustar a la medida del soporte y con diferentes tamaños de grosor, en este caso se usarán colchones de 2cm de grosor aproximadamente, esto permitirá que el paciente no repose su pierna directamente sobre el soporte de PVC y de esta manera las sesiones de rehabilitación serán más cómodas.
 \begin{figure}[h!]
 	\centering
 	\includegraphics [trim = 0 0cm 0 0, clip, width=0.5\textwidth]{figure/s1_estructural/Camilla.png}
 	\caption[Colchón propuesto para comodidad del paciente.]{Colchón propuesto para comodidad del paciente.\label{fig:cama}}
 \end{figure}