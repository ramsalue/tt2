% ==============================================================================
% ANEXO A (o 1, según configuración)
% ==============================================================================

\chapter{Hoja de datos}
% Este será "Anexo A" o "Anexo 1" automáticamente

Una ficha técnica, hoja técnica u hoja de datos (datasheet en inglés), también 
ficha de características u hoja de características, es un documento que resume 
el funcionamiento y otras características de un componente (por ejemplo, un 
componente electrónico) o subsistema (por ejemplo, una fuente de alimentación) 
con el suficiente detalle para ser utilizado por un ingeniero de diseño y 
diseñar el componente en un sistema.

Comienza típicamente con una página introductoria que describe el resto del 
documento, seguido por los listados de componentes específicos, con la 
información adicional sobre la conectividad de los dispositivos. En caso de 
que haya código fuente relevante a incluir, se une cerca del extremo del 
documento o se separa generalmente en otro archivo.

Las fichas técnicas no se limitan solo a componentes electrónicos, sino que 
también se dan en otros campos de la ciencia, como por ejemplo compuestos 
químicos o alimentos.

% Ejemplo de tabla en anexo
\begin{table}[H]
	\centering
	\caption{Especificaciones del componente}
	\label{tab:anexo-especificaciones}
	\begin{tabular}{|l|c|}
		\hline
		\textbf{Parámetro} & \textbf{Valor} \\
		\hline
		Voltaje & 12V \\
		Corriente & 2A \\
		\hline
	\end{tabular}
\end{table}